%----------------------------------------------------------------------------------------
%   NUMBERING
%----------------------------------------------------------------------------------------
\numberwithin{equation}{section}
\numberwithin{table}{section}
\numberwithin{figure}{section}

%----------------------------------------------------------------------------------------
%	SHORTCUTS
%----------------------------------------------------------------------------------------
\newcommand{\B}[1]{\mathbb{#1}}
\newcommand{\Cl}[1]{\ensuremath{\mathcal{#1}}}

\newcommand{\sse}{\Leftrightarrow}
\newcommand{\so}{\Rightarrow}
\newcommand{\se}{\Leftarrow}
\newcommand{\rec}{\leftarrow}

\newcommand{\tdots}{\,.\,.\,}

\DeclarePairedDelimiter{\ceil}{\lceil}{\rceil}
\DeclarePairedDelimiter{\floor}{\lfloor}{\rfloor}

%----------------------------------------------------------------------------------------
%	ALGPSEUDOCODE AND ALGORITHMS
%----------------------------------------------------------------------------------------
% language
\algrenewtext{Function}[2]{\textbf{função} \textsc{#1}$(#2)$}
\algrenewtext{For}[1]{\textbf{para} #1 \textbf{faça}}
\algrenewtext{If}[1]{\textbf{se} #1 \textbf{então}}
\algrenewtext{ElsIf}[1]{\textbf{senão se} #1 \textbf{então}}
\algrenewtext{Else}{\textbf{senão}}
\algrenewcommand\Return{\textbf{devolve} }
\floatname{algorithm}{Algoritmo}

% number togheter
\makeatletter
\let\c@algorithm\c@equation
\let\c@table\c@equation
\let\c@figure\c@equation
\makeatother

%----------------------------------------------------------------------------------------
%	THEOREMS
%----------------------------------------------------------------------------------------
\newtheorem{prop}[equation]{Proposição}
\newtheorem{theo}[equation]{Teorema}
\newtheorem{lema}[equation]{Lema}

\newtheorem{defi}[equation]{Definição}

\newtheorem{prob}[equation]{Problema}
