\section{Cronograma}
Durante os meses de Janeiro e Fevereiro foi realizado um estudo inicial da bibliografia, em Fevereiro e Março foi escrita a Seção~\ref{DivConq} sobre a técnica da divisão e conquista, durante Março e Abril foi desenvolvida a Seção~\ref{Monge} sobre matrizes Monge e convexidade e a Seção~\ref{SMAWK} sobre o algoritmo SMAWK foi produzida durante o mês de Abril.  

De Maio até Junho, estudaremos buscas de máximos e mínimos em matrizes online~\cite{Galil:1992}~\cite{Brucker:1995} e, depois disso, ainda em Junho será explorada a otimização de Knuth-Yao~\cite{Bein:2009}~\cite{Knuth:1971}~\cite{Yao:1980}. Após estudar estes algoritmos e suas aplicações, serão estudadas as possíveis generalizações deles. Em Julho e Agosto, olharemos para os algoritmos sobre matrizes online já explorados sob o ponto de vista da dualidade ponto-reta~\cite{Berg:2000}, em Agosto e Setembro exploramos condições mais fracas para a busca em matrizes online e em Setembro e Outubro revisitamos a otimização de Knuth-Yao buscando, também, enfraquecer as condições para sua aplicação. 

Em paralelo com quase todo o desenvolvimento do trabalho, de Fevereiro até Outubro, serão implementados e testados os algoritmos discutidos. Finalmente, o mês de Novembro será reservado para a finalizar e preparar a apresentação do trabalho.
