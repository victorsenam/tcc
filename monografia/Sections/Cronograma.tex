\section{Cronograma}
Durante os meses de janeiro e fevereiro foi realizado um estudo inicial da literatura relacionada ao tema, em fevereiro e março foi escrita a Seção~\ref{DivConq}, sobre a técnica da divisão e conquista, durante março e abril foi desenvolvida a Seção~\ref{Monge}, sobre matrizes Monge e convexidade, e a Seção~\ref{SMAWK} sobre o algoritmo SMAWK foi produzida durante o mês de abril.  

Em maio, estudaremos a otimização de Knuth-Yao~\cite{Bein:2009,Knuth:1971,Yao:1980} e, depois disso, em junho e julho, exploraremos as buscas de máximos e mínimos em matrizes online~\cite{Galil:1992,Brucker:1995}. Após estudar estes algoritmos e suas aplicações, serão estudadas as possíveis generalizações deles. Em julho e agosto, olharemos para os algoritmos sobre matrizes online já explorados sob o ponto de vista da dualidade ponto-reta~\cite{Berg:2000}, em agosto e setembro, exploraremos condições mais fracas para a busca em matrizes online, e em outubro e novembro, revisitaremos a otimização de Knuth-Yao buscando, também, enfraquecer as condições para sua aplicação. Havendo tempo, dependendo do andamento do trabalho, outros tópicos relacionados ao tema poderão ser estudados. 

Em paralelo com quase todo o desenvolvimento do trabalho, de fevereiro até novembro, serão implementados e testados os algoritmos discutidos. Finalmente, o mês de dezembro será reservado para a finalizar o texto e fazer ajustes finais nas implementações. 

% Durante os meses de Janeiro e Fevereiro foi realizado um estudo inicial da bibliografia, em Fevereiro e Março foi escrita a Seção~\ref{DivConq} sobre a técnica da divisão e conquista, durante Março e Abril foi desenvolvida a Seção~\ref{Monge} sobre matrizes Monge e convexidade e a Seção~\ref{SMAWK} sobre o algoritmo SMAWK foi produzida durante o mês de Abril.  

% Em Maio, estudaremos a otimização de Knuth-Yao~\cite{Bein:2009,Knuth:1971,Yao:1980} e, depois disso, em Junho e Julho, exploraremos as buscas de máximos e mínimos em matrizes online~\cite{Galil:1992,Brucker:1995}. Após estudar estes algoritmos e suas aplicações, serão estudadas as possíveis generalizações deles. Em Julho e Agosto, olharemos para os algoritmos sobre matrizes online já explorados sob o ponto de vista da dualidade ponto-reta~\cite{Berg:2000}, em Agosto e Setembro exploramos condições mais fracas para a busca em matrizes online e em Setembro e Outubro revisitamos a otimização de Knuth-Yao buscando, também, enfraquecer as condições para sua aplicação. 

% Em paralelo com quase todo o desenvolvimento do trabalho, de Fevereiro até Outubro, serão implementados e testados os algoritmos discutidos. Finalmente, o mês de Novembro será reservado para a finalizar e preparar a apresentação do trabalho.
