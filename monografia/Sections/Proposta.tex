\begin{center}
\LARGE{\scshape{projeto de pesquisa}} \\
\vspace{.5cm}
\large{\scshape{Algoritmos em matrizes monótonas e Monge convexas}} \\
\end{center}

\vspace{1cm}

\noindent
\textsc{Aluno:} Victor Sena Molero \\
\textsc{Orientadora:} Profa. Cristina G. Fernandes

\vspace{1cm}


Este é o projeto de pesquisa referente ao pedido de bolsa PIBIC de Victor Sena Molero, aluno de Bacharelado em Ciência da Computação do IME-USP.

\section{Introdução}

As matrizes Monge têm propriedades como a monotonicidade total e a monotonicidade, que são exploradas por uma série de algoritmos para a busca de máximos e mínimos em linhas e colunas destas matrizes. Algumas destas técnicas são populares pois são utilizadas para a agilização de problemas clássicos e aparecem com crescente frequência em competições de programação por sua utilidade na solução eficiente de problemas de programação dinâmica.

Busca-se neste trabalho formalizar, compilar e aprofundar os conhecimentos adquiridos sobre estes algoritmos por meio do estudo da literatura já existente~\cite{Galil:1992,Bein:2009} a fim de facilitar o aprendizado destas técnicas. Além disso, procura-se generalizar os algoritmos inspirando-se em desafios propostos em competições, explorando suas limitações e modificando as formalizações encontradas na literatura para ampliar os escopos tradicionais. Serão disponibilizadas, também, implementações eficientes dos algoritmos estudados que sejam utilizáveis em diversos cenários. Serão criados e fornecidos testes para comparar as velocidades delas quando possível.

No que segue, apresentaremos parte da pesquisa já realizada. O nosso objetivo neste projeto é continuar o estudo nesta linha. Ao final do texto, apresentamos o cronograma planejado para o desenvolvimento da pesquisa.
