\section{Proposta}

As matrizes Monge têm propriedades como a monotonicidade total e a monotonicidade, que são exploradas por uma série de algoritmos para a busca de máximos e mínimos em linhas e colunas destas matrizes. Algumas destas técnicas são populares pois são utilizadas para a agilização de problemas clássicos e aparecem com crescente frequência em competições de programação por sua utilidade na solução eficiente de problemas de programação dinâmica.

Busca-se neste trabalho formalizar, compilar e aprofundar os conhecimentos adquiridos sobre estes algoritmos por meio do estudo da literatura já existente~\cite{Galil:1992}~\cite{Bein:2009} a fim de facilitar o aprendizado destas técnicas. Além disso, procura-se generalizar os algoritmos inspirando-se em desafios propostos em competições, explorando suas limitações e modificando as formalizações encontradas na literatura para ampliar os escopos tradicionais. Serão disponibilizadas, também, implementações eficientes dos algoritmos estudados que sejam utilizáveis em diversos cenários. Serão criados e fornecidos testes para comparar as velocidades delas quando possível.

\subsection{Cronograma}
Algumas das atividades planejadas já foram realizadas. Durante os meses de Janeiro e Fevereiro foi realizado um estudo inicial da bibliografia, em Fevereiro e Março foi escrita a Seção~\ref{DivConq} sobre a técnica da divisão e conquista, durante Março e Abril foi desenvolvida a Seção~\ref{Monge} sobre matrizes Monge e convexidade e a Seção~\ref{SMAWK} sobre o algoritmo SMAWK foi produzida durante o mês de Abril.  
Durante os meses de Maio e Junho, o estudaremos buscas de máximos e mínimos em matrizes online~\cite{Galil:1992}~\cite{Brucker:1995} e, depois disso, ainda em Junho será explorada a otimização de Knuth-Yao~\cite{Bein:2009}~\cite{Knuth:1971}~\cite{Yao:1980}. Após estudar estes algoritmos e suas aplicações, serão estudadas as possíveis generalizações deles. Em Julho e Agosto, olharemos para os algoritmos sobre matrizes online já explorados sob o ponto de vista da dualidade ponto-reta~\cite{Berg:2000}, em Agosto e Setembro exploramos condições mais fracas para a busca em matrizes online e em Setembro e Outubro revisitamos a otimização de Knuth-Yao buscando, também, enfraquecer as condições para sua aplicação. Durante quase todo o desenvolvimento do trabalho, de Fevereiro até Outubro, serão implementados e testados os algoritmos discutidos. Finalmente, o mês de Novembro será reservado para a finalizar e preparar a apresentação do trabalho.

\subsection{Texto atual} \label{Intro:impl}
No que segue, apresentaremos parte da pesquisa já realizada. O nosso objetivo neste projeto é continuar o estudo nesta linha.

:q
cllear

:visual

