\chapter{Exemplos implementados}
\label{Exemplos}

Neste capítulo descrevemos os exemplos implementados disponíveis em~\texttt{implementacao/exemplos}. O objetivo de implementar exemplos é mostrar possibilidades de aplicações simples dos algoritmos discutidos e mostrar maneiras facilmente adaptáveis de utilizar as implementações desenvolvidas neste trabalho.

Cada exemplo pode ser implementado por mais de um dos algoritmos estudados. Uma implementação ``trivial`` também é fornecida para cada exemplo, demonstrando uma forma simples e pouco eficiente de resolver o problema proposto que não conta com os algoritmos discutidos. Aqui serão descritos os problemas resolvidos em cada exemplo, o formato de entrada e saída e serão listadas as soluções implementadas por cada um deles. Um eventual comentário sobre a modelagem feita pode ser realizado aqui.

Cada implementação de um exemplo leva um nome do formato~\texttt{NomeDoExemplo.Algoritmo.cpp} e é compilado, a partir do utilitário~\texttt{make} com comportamente descrito pelo~\texttt{Makefile} presente em~\texttt{implementacao}, para o arquivo~\texttt{NomeDoExemplo.Algoritmo.out} onde~\texttt{NomeDoExemplo} é especificado nas descrições abaixo.

%-----------------------------------------------

\section{Exemplo Monge} \label{ExemploMonge}

Este é um exemplo simples da aplicação dos algoritmos em matrizes Monge. Ele resolve o Problema~\ref{ Monge:example } dado o vetor~$w \in \B{Q}^{n}_+$ e um inteiro~$k$. A implementação leva o nome~\texttt{ExemploMonge}.

O exemplo foi implementado com a otimização da Divisão e Conquista (\texttt{DivConq}), com o algoritmo SMAWK (\texttt{SMAWK}) e da maneira trivial (\texttt{Trivial}).

\subsection{Descrição da entrada}
Na primeira linha da entrada são dados dois inteiros~$k$ e~$n$ separados por um espaço. Na segunda linha são dados os valores do vetor~$w$, separados por espaços.

\subsection{Descrição da saída}
Na única linha da saída, deve ser impresso o valor mínimo da função otimizada pelo problema proposto.

\subsection{Exemplo}
Entrada:
\begin{verbatim}
3 7
13 7 4 9 14 2 5
\end{verbatim}
Saída:
\begin{verbatim}
1010
\end{verbatim}

\section{Monge simples} \label{SimpleMonge}

Este exemplo demonstra a aplicação direta dos algoritmos estudados neste trabalho. Recebe uma matriz~$A \in \B{Q}^{n \times m}$ Monge convexa e retorna os valores dos mínimos de cada uma das linhas desta matriz. A implementação leva o nome~\texttt{SimpleMonge}.

O exemplo foi implementado com a otimização da Divisão e Conquista (\texttt{DivConq}), com o algoritmo SMAWK (\texttt{SMAWK}) e da maneira trivial (\texttt{Trivial}).

\subsection{Descrição da entrada}

Na primeira linha da entrada são dadas as dimensões~$n$ e~$m$ da matriz~$A$. As próximas~$n$ linhas contém~$m$ valores separados por espaços cada. O~$j$-ésimo valor da~$i$-ésima linha é a entrada~$A[i][j]$ da matriz.

\subsection{Descrição da saída}

O programa deve imprimir~$n$ valores, um em cada linha. A~$i$-ésima linha da saída deve conter o valor de mínimo da~$i$-ésima linha de~$A$.

\subsection{Exemplo}

Entrada:
\begin{verbatim}
3 4
3 7 10 5
2 5 7 2
9 5 6 0
\end{verbatim}
Saída:
\begin{verbatim}
3
2
0
\end{verbatim}

\section{Internet Trouble simplificado} \label{InternetTroubleSimples}

Este é um exemplo simples da aplicação dos algoritmos em matrizes Monge. Ele resolve o Problema~\ref{DivConq:InternetTrouble} dado o vetor~$v \in \B{Q}^{n}_+$ e um inteiro~$k$. A implementação leva o nome~\texttt{InternetTroubleSimples}.

O exemplo foi implementado com a otimização da Divisão e Conquista (\texttt{DivConq}), com o algoritmo SMAWK (\texttt{SMAWK}) e da maneira trivial (\texttt{Trivial}).

\subsection{Descrição da entrada}

Na primeira linha da entrada são dados os valores~$k$ e~$n$ separados por um espaço. Na próxima linha são dados~$n$ valores separados por espaços, as entradas do vetor~$v$.

\subsection{Descrição da saída}

O programa deve imprimir um único valor na saída, a resposta do problema proposto.

\subsection{Exemplo}

Entrada:
\begin{verbatim}
3 7
13 7 4 9 14 2 5
\end{verbatim}
Saída:
\begin{verbatim}
13
\end{verbatim}

%-----------------------------------------------

\section{Problema online convexo} \label{ProblemaOnlineConvexo}

Resolve o Problema~\ref{Online:prob} no caso convexo dado um vetor~$w \in \B{Q}^n_+$ onde a matriz~$C \in \B{Q}^{n \times n}$ é tal que, para todo~$i,j\in[n]$, vale que~$C[i][j] = (\sum_{k=1}^j w_k - \sum_{k=1}^{i-1} w_i)^2$. A implementação leva o nome~\texttt{ProblemaOnlineConvexo}.

O exemplo foi implementado com o envelope convexo (\texttt{EnvelopeConvexo}) e da maneira trivial (\texttt{Trivial}).

\subsection{Descrição da entrada}
Na primeira linha da entrada é dado o inteiro~$n$. Na próxima linha são dados~$n$ valores separados por espaços. O~$i$-ésimo deles (a partir do 1) é a entrada~$w_i$.

\subsection{Descrição da saída}
A saída consiste de~$n$ valores em uma linha, o~$i$-ésimo valor (à partir do 1) deve ser o valor correto de~$E[i]$.

\subsection{Exemplo}
Entrada:
\begin{verbatim}
7
13 7 4 9 14 2 5
\end{verbatim}
Saída:
\begin{verbatim}
1315.000000 1315.000000 915.000000 794.000000 625.000000 256.000000 0.000000
\end{verbatim}

%-----------------------------------------------

\section{Problema online côncavo} \label{ProblemaOnlineConcavo}

Resolve o Problema~\ref{Online:prob} no caso côncavo dado um vetor~$w \in \B{Q}^n_+$ onde a matriz~$C \in \B{Q}^{n \times n}$ é tal que, para todo~$i,j\in[n]$, vale que~$C[i][j] = -(\sum_{k=1}^j w_k - \sum_{k=1}^{i-1} w_i)^2$. A implementação leva o nome~\texttt{ProblemaOnlineConcavo}.

O exemplo foi implementado com o envelope convexo (\texttt{EnvelopeConcavo}) e da maneira trivial (\texttt{Trivial}).

\subsection{Descrição da entrada}
Na primeira linha da entrada é dado o inteiro~$n$. Na próxima linha são dados~$n$ valores separados por espaços. O~$i$-ésimo deles (a partir do 1) é a entrada~$w_i$.

\subsection{Descrição da saída}
A saída consiste de~$n$ valores em uma linha, o~$i$-ésimo valor (à partir do 1) deve ser o valor correto de~$E[i]$.

\subsection{Exemplo}
Entrada:
\begin{verbatim}
7
13 7 4 9 14 2 5
\end{verbatim}
Saída:
\begin{verbatim}
-2634.000000 -2465.000000 -1412.000000 -985.000000 -785.000000 -256.000000 0.000000
\end{verbatim}

%-----------------------------------------------

\section{NKLEAVES} \label{NKLEAVES}

Resolve o problema NKLEAVES disponível em http://www.spoj.com/problems/NKLEAVES/.

O exemplo foi implementado com o envelope convexo (\texttt{EnvelopeConvexo}), o envelope convexo linear (\texttt{EnvelopeConvexoLinear}) e da maneira trivial (\texttt{Trivial}).
