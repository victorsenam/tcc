\section{Monotonicidade, convexidade e matrizes Monge}
\label{MatrizMonge}

%----------------------------------------------------------------------------------------

Aqui serão apresentados e explorados os conceitos de monotonicidade, convexidade e matrizes Monge, além disso, alguns resultados referentes a estes conceitos serão demonstrados. Estes conceitos são fundamentais para o desenvolvimento do restante do trabalho.

\begin{defi}[Vetor Monótono]
Seja $a \in \B{Q}^n$ um vetor, $a$ é dito monótono quando vale uma das propriedades abaixo.
    \begin{enumerate}
        \item Se para todo $i,j \in [n]$, $i < j \so a_i \leq a_j$, $a$ é dito monótono crescente (ou só crescente).
        \item Se para todo $i,j \in [n]$, $i < j \so a_i \geq a_j$, $a$ é dito monótono decrescente (ou só decrescente).
    \end{enumerate}
\end{defi}

Sabemos que a monotonicidade de vetores pode ser aproveitada para agilizar alguns algoritmos importantes, por exemplo, a busca binária pode ser interpretada como uma otimização da busca linear para vetores monótonos. 

\begin{defi}[Vetor Convexo]
Seja $a \in \B{Q}^n$ um vetor,
    \begin{enumerate}
        \item se para todo $i,j,k \in [n]$, $i < j < k \so a_j \leq \frac{(j-k)a_i + (i-j)a_k}{i-k}$, $a$ é dito convexo e
        \item se para todo $i,j,k \in [n]$, $i < j < k \so a_j \geq \frac{(j-k)a_i + (i-j)a_k}{i-k}$, $a$ é dito côncavo.
    \end{enumerate}
\end{defi}

Vale notar que a definição dada acima é específica para vetores porém é compatível com a definição usual de funções convexas, isto é, $a$ é convexa se e somente se existe uma função $f$ convexa tal que para todo $i \in [n]$, $a_i = f(i)$. A afirmação análoga para a concavidade também é verdadeira.  

Assim como a monotonicidade, a convexidade também é usualmente explorada para agilizar algoritmos, por exemplo, se um vetor é estritamente convexo (basta substituir $\leq$ por $<$ na definição) podemos encontrar o mínimo (que é único) com uma busca ternária ao invés de percorrer todo o vetor.  

\begin{defi}
Seja $A \in \B{Q}^{n \times m}$, definimos
\todo{Talvez essa definição vá para a introdução. Prometo não usar essa notação estranha sem aviso.}
    \begin{enumerate}
        \item $\hat{j_A} \in \B{Q}^n : i \mapsto \min\{j \in [m] \mid A[i][j] \geq A[i][j'] \text{ para todo } j' \in [m]\}$, o vetor de índices de máximos das linhas de $A$,
        \item $\check{j_A} \in \B{Q}^n : i \mapsto \min\{j \in [m] \mid A[i][j] \leq A[i][j'] \text{ para todo } j' \in [m]\}$, o vetor de índices de mínimos das linhas de $A$,
        \item $\hat{i_A} \in \B{Q}^m : j \mapsto \min\{i \in [n] \mid A[i][j] \geq A[i'][j] \text{ para todo } i' \in [n]\}$, o vetor de índices de máximos das colunas de $A$ e 
        \item $\check{i_A} \in \B{Q}^m : j \mapsto \min\{i \in [n] \mid A[i][j] \leq A[i'][j] \text{ para todo } i' \in [n]\}$, o vetor de índices de mínimos das colunas de $A$.
    \end{enumerate}
\end{defi}

Calcular estes vetores para dadas matrizes é um problema interessante. Algumas propriedades das matrizes podem ser exploradas a fim de agilizar este cálculo. A monotonicidade, convexidade ou concavidade de matrizes indicam propriedades interessantes sobre estes problemas, portanto, vamos explorar estas definições.

\begin{defi}[Matriz Monótona]
Seja $A \in \B{Q}^{n \times m}$ uma matriz, $A$ é dita monótona crescente (ou decrescente) nos máximos (ou mínimos) das linhas (ou colunas) se o vetor de índices de máximos (ou mínimos) das linhas (ou colunas) de $A$ é crescente (ou decrescente). Formalmente,
    \begin{enumerate}
        \item Se o vetor de índices de máximos das linhas de $A$ é crescente, $A$ é dita monótona crescente nos máximos das linhas.
        \item Se o vetor de índices de máximos das linhas de $A$ é decrescente, $A$ é dita monótona decrescente nos máximos das linhas.
        \item Se o vetor de índices de máximos das colunas de $A$ é crescente, $A$ é dita monótona crescente nos máximos das colunas.
        \item Se o vetor de índices de máximos das colunas de $A$ é decrescente, $A$ é dita monótona decrescente nos máximos das colunas.
        \item Se o vetor de índices de mínimos das linhas de $A$ é crescente, $A$ é dita monótona crescente nos mínimos das linhas.
        \item Se o vetor de índices de mínimos das linhas de $A$ é decrescente, $A$ é dita monótona decrescente nos mínimos das linhas.
        \item Se o vetor de índices de mínimos das colunas de $A$ é crescente, $A$ é dita monótona crescente nos mínimos das colunas.
        \item Se o vetor de índices de mínimos das colunas de $A$ é decrescente, $A$ é dita monótona decrescente nos mínimos das colunas.
    \end{enumerate}
\end{defi}

\begin{defi}[Matrizes Totalmente Monótonas]
\end{defi}
