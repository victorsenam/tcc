\section{Monotonicidade, convexidade e matrizes Monge}
\label{MatrizMonge}

%----------------------------------------------------------------------------------------

Aqui serão apresentados e explorados os conceitos de monotonicidade, convexidade e matrizes Monge, além disso, alguns resultados referentes a estes conceitos serão demonstrados. Estes conceitos são fundamentais para o desenvolvimento do restante do trabalho.

\begin{defi}[Vetor monótono]
Seja $a \in \B{Q}^n$ um vetor, $a$ é dito monótono quando vale uma das propriedades abaixo.
    \begin{itemize}
        \item Se para todo $i,j \in [n]$, $i < j \so a_i \leq a_j$, $a$ é dito monótono crescente (ou só crescente).
        \item Se para todo $i,j \in [n]$, $i < j \so a_i \geq a_j$, $a$ é dito monótono decrescente (ou só decrescente).
    \end{itemize}
\end{defi}

Sabemos que a monotonicidade de vetores pode ser aproveitada para agilizar alguns algoritmos importantes, por exemplo, a busca binária pode ser interpretada como uma otimização da busca linear para vetores monótonos. 

\begin{defi}[Função convexa]
Seja $g : \B{Q} \to \B{Q}$ uma função,
    \begin{itemize}
        \item se para todo par de pontos $x,y \in \B{Q}$ e $\lambda \in \B{Q}$ que respeita $0 \leq \lambda \leq 1$, vale~${g(\lambda x + (1 - \lambda)y) \leq \lambda g(x) + (1 - \lambda) g(y)}$, $g$ é dita convexa e 
        \item se para todo par de pontos $x,y \in \B{Q}$ e $\lambda \in \B{Q}$ que respeita $0 \leq \lambda \leq 1$, vale~${g(\lambda x + (1 - \lambda)y) \geq \lambda g(x) + (1 - \lambda) g(y)}$, $g$ é dita côncava.
    \end{itemize}
\end{defi}

\begin{prop} \label{prop:sqConv}
A função $g(x) = x^2$ é convexa. \todo{eu uso isso lá no final do capítulo, achei razoável colocar aqui pra deixar menos poluído lá na frente, a explicação de lá já é bem complexa.}
\end{prop}

\begin{proof}
Só fazer.
\end{proof}

É interessante definir convexidade também em termos de vetores.

\begin{defi}[Vetor convexo]
Seja $a \in \B{Q}^n$ um vetor,
    \begin{itemize}
        \item se para todo $i,j,k \in [n]$, $i < j < k \so a_j \leq \frac{(j-k)a_i + (i-j)a_k}{i-k}$, $a$ é dito convexo e
        \item se para todo $i,j,k \in [n]$, $i < j < k \so a_j \geq \frac{(j-k)a_i + (i-j)a_k}{i-k}$, $a$ é dito côncavo.
    \end{itemize}
\end{defi}

Assim como a monotonicidade, a convexidade também é usualmente explorada para agilizar algoritmos, por exemplo, se um vetor é convexo podemos definir o valor mínimo do vetor com uma busca ternária ao invés de percorrer todo o vetor.  

\begin{defi}
Seja $A \in \B{Q}^{n \times m}$, definimos quatro vetores a seguir.
    \begin{itemize}
        \item O vetor de índices de máximos das linhas de $A$ guarda na posição $i$ 
              o número $\max\{j \in [m] \mid A[i][j] \geq A[i][j'] \text{ para todo } j' \in [m]\}$. 
        \item O vetor de índices de mínimos das linhas de $A$ guarda na posição $i$ 
              o número $\min\{j \in [m] \mid A[i][j] \leq A[i][j'] \text{ para todo } j' \in [m]\}$. 
        \item O vetor de índices de máximos das colunas de $A$ guarda na posição $j$ 
              o número $\max\{i \in [n] \mid A[i][j] \geq A[i'][j] \text{ para todo } i' \in [n]\}$. 
        \item O vetor de índices de mínimos das colunas de $A$ guarda na posição $j$ 
              o número $\min\{i \in [n] \mid A[i][j] \leq A[i'][j] \text{ para todo } i' \in [n]\}$. 
    \end{itemize}
\end{defi}

Note que o máximo de uma linha (ou coluna) foi definido como o maior índice que atinge o máximo e o de mínimo foi definido como o menor índice que atinge o mínimo. Esta escolha foi feita para simplificar o Lema~\ref{lema:MonotoneTotallyMonotone}.  

Dada uma matriz, encontrar estes vetores é um problema central para este trabalho. Neste momento é interessante classificar algumas matrizes de acordo com propriedades que vão nos ajudar a calcular os vetores de mínimos e máximos de maneira especialmente eficiente.   

A Figura~\ref{table:ConvexMonotone} resume as relações de implicação da classificação que será realizada. Os conceitos ilustrados nela serão apresentados a seguir.

\begin{table}[t]
    \caption{Comportamento dos vetores de índices ótimos em relação à convexidade.} \label{table:ConvexMonotone}
    \begin{tabular}{ c | c | c }
        Se a matriz é & os índices de máximos são & e os de mínimos são \\
        \hline
        convexa       & decrescentes              & crescentes        \\
        \hline
        côncava       & crescentes                & decrescentes          \\
    \end{tabular}
\end{table}

\begin{defi}[Matriz monótona]
Seja $A \in \B{Q}^{n \times m}$ uma matriz. Se $A$ tiver o vetor de índices de máximos das linhas monótono, $A$ é dita monótona nos máximos das linhas. 

Valem também as definições análogas para mínimos ou colunas e pode-se especificar monotonicidade crescente ou decrescente.
\end{defi}

\begin{defi}[Matriz totalmente monótona]
Seja $A \in \B{Q}^{n \times m}$ uma matriz.
    \begin{itemize}
        \item Se $A[i'][j] \leq A[i'][j']$ implica $A[i][j] \leq A[i][j']$ para todo $1 \leq i < i' \leq n$ e $1 \leq j < j' \leq m$, $A$ é monótona convexa nas linhas.
        \item Se $A[i][j'] \leq A[i'][j']$ implica $A[i][j] \leq A[i'][j]$ para todo $1 \leq i < i' \leq n$ e $1 \leq j < j' \leq m$, $A$ é monótona convexa nas colunas.
        \item Se $A[i'][j] > A[i'][j']$ implica $A[i][j] > A[i][j']$ para todo $1 \leq i < i' \leq n$ e $1 \leq j < j' \leq m$, $A$ é monótona côncava nas linhas.
        \item Se $A[i][j'] > A[i'][j']$ implica $A[i][j] > A[i'][j]$ para todo $1 \leq i < i' \leq n$ e $1 \leq j < j' \leq m$, $A$ é monótona côncava nas colunas.
    \end{itemize}
\end{defi}

O motivo do uso dos termos ``convexa'' e ``côncava'' em relação a matrizes durante o texto são justificados pelo Teorema~\ref{theo:MongeConvex_and_Convex}. Note que se uma matriz é totalmente monótona, todas as suas submatrizes são totalmente monótonas no mesmo sentido.

\begin{lema} \label{lema:MonotoneTotallyMonotone}
Se $A \in \B{Q}^{n \times m}$ é uma matriz totalmente monótona convexa nas linhas, toda submatriz de $A$ é monótona decrescente nos máximos das linhas e monótona crescente nos mínimos das linhas.  

Se $A$ é totalmente monótona côncava nas linhas, toda submatriz de $A$ é monótona decrescente nos mínimos das linhas e monótona crescente nos máximos das linhas.  

As afirmações valem identicamente em termos de colunas.
\end{lema}

\begin{proof}
Considere uma matriz $A$ totalmente monótona convexa nas linhas. Sejam $i$ e $i'$ índices de linhas de $A$ onde $i < i'$. Chamamos de $j$ o índice de máximo da linha $i$ e de $j'$ o índice de máximo da linha $i'$. Queremos provar que os máximos são decrescentes, portanto, vamos supor por absurdo que $j < j'$. Com isso, teremos $A[i][j'] < A[i][j]$ e $A[i'][j] \leq A[i'][j']$. Porém, já que $A$ é monótona convexa nas linhas, a segunda desigualdade implica em $A[i][j] \leq A[i][j']$, que contradiz a primeira. Portanto, os índices de máximos são decrescentes.  

Agora, considere novamente dois índices $i$ e $i'$ quaisquer de linhas de $A$ onde $i < i'$. Denotamos por $j$ o índice de mínimo da linha $i'$ e por $j'$ o índice de mínimo da linha $i$ (note e a inversão no uso de $'$). Vamos supor por absurdo que $j < j'$ e teremos $A[i'][j] \leq A[i'][j']$ e $A[i][j'] < A[i][j]$. E, novamente, usando o fato de que $A$ é monótona convexa nas linhas, obtivemos uma contradição.  

Finalmente, se $A'$ é uma submatriz de $A$, então $A'$ é totalmente monótona convexa nas linhas, portanto monótona crescente nos máximos das linhas e monótona decrescente nos mínimos das linhas.

As demonstrações no caso côncavo e nos casos relacionados a colunas são análogas.
\end{proof}

\begin{defi}[Monge Convexidade] \label{defi:MatrizMonge}
Seja $A \in \B{Q}^{n \times m}$.
    \begin{enumerate}
        \item Se vale $A[i][j] + A[i'][j'] \leq A[i][j'] + A[i'][j]$ para todo $1 \leq i < i' \leq n$ e $1 \leq j < j' \leq m$, $A$ é dita Monge convexa.
        \item Se vale $A[i][j] + A[i'][j'] \geq A[i][j'] + A[i'][j]$ para todo $1 \leq i < i' \leq n$ e $1 \leq j < j' \leq m$, $A$ é dita Monge côncava.
    \end{enumerate}
\end{defi}

A desigualdade que define as matrizes Monge é conhecida também por ``Condição de Monge'' ou ``Desigualdade Quadrangular''\todo{citação}.

\begin{lema}
Se~$A$ é Monge convexa,~$A$ é totalmente monótona convexa tanto nas linhas quanto nas colunas.  

Se~$A$ é Monge côncava,~$A$ é totalmente monótona côncava tanto nas linhas quanto nas colunas.
\end{lema}

\begin{proof}
Seja~$A$ uma matriz Monge convexa. Suponha que vale, para certos $i,i' \in [n]$ e $j,j' \in [m]$ onde $i < i'$ e $j < j'$, $A[i'][j] \leq A[i'][j']$, então, somamos esta desigualdade à definição de Monge convexa e obtemos $A[i][j] \leq A[i][j']$, ou seja, $A$ é totalmente monótona convexa nas linhas.  

Por outro lado, se vale, para certos $i,i' \in [n]$ e $j,j' \in [m]$ com $i < i'$ e $j < j'$, $A[i][j'] \leq A[i'][j']$, somamos esta desigualdade à definição de Monge convexa e obtemos $A[i][j] \leq A[i'][j]$, assim, $A$ é totalmente monótona convexa nas colunas.  

A prova para o caso côncavo é análoga.
\end{proof}

As matrizes Monge são usadas para resolver uma série de problemas que serão explorados aqui. A condição de Monge é a mais forte apresentada aqui. Alguns dos algoritmos apresentados não dependem dela, apenas da monotonicidade ou total monotonicidade, ainda assim, ela leva a resultados úteis que nos permitem provar a pertinência dos algoritmos a alguns problemas, mesmo que o algoritmo usado não se utilize da condição diretamente.  

Como consequência desta utilidade, iremos discutir um problema que será resolvido com um algoritmo apresentado somente na Seção~\ref{SMAWK}, o algoritmo SMAWK. Ele não será explicado neste momento, utilizamos ele como caixa preta. Isto é motivado pelo fato de que o pensamento apresentado aqui não é útil somente para o algoritmo SMAWK, ele é útil também em vários dos outros momentos deste trabalho.  

\begin{prob} \label{prob:Monge_ex}
Dados dois inteiros~$k$ e~$n$ com~$k \leq n$, um vetor de pesos~$a \in \B{Q}^n$ e uma matriz de custos~${A[i][j] = \left(\sum\limits_{k=i+1}^{j} a_k\right)^2}$. Queremos particionar o vetor~$a$ em~$k$ partes não-vazias de forma a maximizar a soma dos custos das partes, isto é, queremos escolher um vetor~${r \in \B{N}^{[0 \tdots k]}}$ de índices tal que~${1 = r_0 < r_1 < r_2 < \dots < r_{k-1} < r_k = n}$ de forma que $\sum\limits_{i=1}^{k} A[r_{i-1}][r_i]$ seja máximo.
\end{prob}

Podemos resolver este problema com programação dinâmica. Vamos preencher a matriz $E \in \B{Q}^{k \times n}$ definida recursivamente para todo $k' \in [k]$ e $n' \in [n]$: 
$$ E[k'][n'] = \begin{cases}
    A[1][n']                                               & \text{, se } k'= 1 \text{,} \\[2pt]
    \max\limits_{i=k'-1}^{n'-1} E[k'-1][i] + A[i][n']     & \text{, se } k' \leq n' \text{,} \\[2pt]
    \text{indefinida }                                    & \text{, caso contrário.}
\end{cases} $$

A recorrência acima nos dá em cada entrada~$E[k'][n']$ o maior valor possível alcançado particionando o vetor~$a[1 \tdots n']$ em~$k'$ partes não vazias. Podemos preencher esta tabela trivialmente em tempo~$O(kn^2)$, basta iterarar primeiro pelos índices~$k'$ crescentemente. O caso onde~$k' = 1$ é resolvido trivialmente e os casos maiores podem ser resolvidos, um por vez, testando todas as possibilidades de máximo para todo~$n'$.  

Fixamos um~$k' > 1$. Vamos utilizar o algoritmo SMAWK para agilizar a solução deste subproblema. Este algoritmo é capaz de resolver o Problema~\ref{prob:Monge_SMAWK} (descrito abaixo) em tempo $O(n)$. Precisamos provar, então, que o subproblema resolvido para cada~$k'$ é equivalente ao Problema~\ref{prob:Monge_SMAWK}.

\begin{prob} \label{prob:Monge_SMAWK}
Dada uma matriz $A \in \B{A}^{n \times n}$ totalmente monótona convexa nas colunas, encontrar o vetor de máximos das colunas de $A$.
\end{prob}

Definimos a matriz $B_{k'}$ para todo $i,n' \in \B{N}$ onde $k'-1 \leq i < n' \leq n$ como $B_{k'}[i][n'] = E[k'-1][i] + A[i][n']$. Note que já que $k'$ é fixo, já descobrimos os valores da entrada da matriz $E$ na linha $k'-1$. Encontrar o índice $i$ que atinge o máximo em $E[k'][n']$ é exatamente encontrar o índice de máximo da coluna $n'$ na matriz $B_{k'}$, formalmente, 

$$\max\limits_{i=k'-1}^{n'-1} B_{k'}[i][n'] = \max\limits_{i=k'-1}^{n'-1} E[k'-1][i] + A[i][n'] \text{.}$$

Portanto, basta mostrar que $B_{k'}$ é monótona convexa nas colunas. Para isso, vamos mostrar, com a ajuda dos resultados abaixo, que $B_{k'}$ é Monge convexa.
    

\begin{lema}
Sejam~$A,B \in \B{Q}^{n \times m}$ matrizes e~$c \in \B{Q}^n$ um vetor tais que para todo~$i \in [n]$ e~$j \in [m]$,~${B[i][j] = A[i][j] + c[i]}$. Se~$A$ é Monge convexa,~$B$ é Monge convexa.  

O mesmo resultado vale se~$c \in \B{Q}^m$ e~${B[i][j] = A[i][j] + c[j]}$.
\end{lema}

\begin{proof}
Sejam $A,B$ e $b$ definidos como no enunciado do teorema. Suponha que $A$ é Monge convexa. Vale, para quaisquer~${1 \leq i < i' \leq n}$ e~${1 \leq j < j' \leq m}$,~${A[i][j] + A[i'][j'] \leq A[i'][j] + A[i][j']}$, logo, vale~${A[i][j] + b[i] + A[i'][j'] + b[i'] \leq A[i'][j] + b[i'] + A[i][j'] + b[i]}$ que é~${B[i][j] + B[i'][j'] \leq B[i'][j] + B[i][j']}$. A prova para o caso onde~$c \in \B{Q}^m$ e~${B[i][j] = A[i][j] + c[j]}$ é análoga.
\end{proof}

Com este resultado, é fácil ver que, desde que $A$ seja Monge convexa, $B_1$ será Monge convexa e a Monge convexidade será mantida para todo $B_{k'}$ com $1 \leq k' \leq k$. O teorema a seguir nos ajuda a mostrar que $A$ é Monge convexa.

\begin{theo} \label{theo:MongeConvex_and_Convex}
Sejam~$A \in \B{Q}^{n \times n}$ uma matriz, $b \in \B{Q}^n$ um vetor e~${g : \B{Q} \to \B{Q}}$ uma fução tais que, para todo $i \in [n]$ e $j \in [m]$, $A[i][j] = g\left(\sum\limits_{k=1}^j b_k - \sum\limits_{k=1}^i b_k\right)$. $A$ é convexa se e somente se $g$ é convexa. Similarmente, $A$ é côncava se e somente se $b$ é côncava.
\end{theo}

Antes de apresentar uma prova para o teorema acima, vamos mostrar a utilidade dele no nosso problema atual, o que deve ajudar na compreensão de seu enunciado.  

Queremos mostrar que $A$ é Monge convexa, porém, para um certo vetor $a \in \B{Q}^n$, $A[i][j] = \left(\sum\limits_{k=(i+1)}^{j} a_k \right)^2 = \left(\sum\limits_{k=1}^{j} a_k - \sum\limits_{k=1}^{i} a_k \right)^2$, por definição. Assim, precisamos mostrar apenas que a função $g(x) = x^2$ é convexa, o que segue da Proposição~\ref{prop:sqConv}. Agora, como explicado acima, todas as matrizes $B_{k'}$ são Monge convexas e podemos aplicar o algoritmo SMAWK para todo $k'$, resolvendo o Problema~\ref{prob:Monge_ex} em tempo $O(kn)$.

Agora, nos resta provar o Teorema~\ref{theo:MongeConvex_and_Convex}.

\begin{proof}
Só fazer.
\end{proof}
