\section{Matrizes Monge e monotonicidade}
\label{Monge}

%----------------------------------------------------------------------------------------

Nesta seção serão apresentados e explorados os conceitos de monotonicidade, convexidade e matrizes Monge, além disso, alguns resultados referentes a estes conceitos serão demonstrados. As definições e os resultados desta seção são fundamentais para o desenvolvimento do restante do trabalho.  

\begin{defi}[Vetor monótono]
Seja $a \in \B{Q}^n$ um vetor, $a$ é dito monótono quando vale uma das propriedades abaixo.
    \begin{itemize}
        \item Se para todo $i,j \in [n]$, $i < j \so a_i \leq a_j$, $a$ é dito monótono crescente (ou só crescente).
        \item Se para todo $i,j \in [n]$, $i < j \so a_i \geq a_j$, $a$ é dito monótono decrescente (ou só decrescente).
    \end{itemize}
\end{defi}

Sabemos que a monotonicidade de vetores pode ser aproveitada para agilizar alguns algoritmos importantes, por exemplo, a busca binária pode ser interpretada como uma otimização da busca linear para vetores monótonos. 

\begin{defi}[Função convexa]
Seja $g : \B{Q} \to \B{Q}$ uma função,
    \begin{itemize}
        \item se para todo par de pontos $x,y \in \B{Q}$ e $\lambda \in \B{Q}$ que respeita $0 \leq \lambda \leq 1$, vale~${g(\lambda x + (1 - \lambda)y) \leq \lambda g(x) + (1 - \lambda) g(y)}$, $g$ é dita convexa e 
        \item se para todo par de pontos $x,y \in \B{Q}$ e $\lambda \in \B{Q}$ que respeita $0 \leq \lambda \leq 1$, vale~${g(\lambda x + (1 - \lambda)y) \geq \lambda g(x) + (1 - \lambda) g(y)}$, $g$ é dita côncava.
    \end{itemize}
\end{defi}

\begin{prop} \label{Monge:sqConv}
A função $g(x) = x^2$ é convexa. 
\end{prop}

\begin{proof}
Sejam~$x,y,\lambda \in \B{Q}$ onde vale~$0 \leq \lambda \leq 1$. Queremos provar~$(\lambda x + (1 - \lambda)y)^2 \leq \lambda x^2 + (1 - \lambda) y^2$, isso equivale a
$$\lambda^2x^2 + (1 - \lambda)^2y^2 + 2\lambda(1-\lambda)xy \leq \lambda x^2 + (1 - \lambda)y^2 \text{, ou seja}$$
$$(\lambda^2 - \lambda)(x^2) + ((1-\lambda)^2 - (1 - \lambda))y^2 + 2(\lambda - \lambda^2)xy \leq 0 \text{, que é}$$
$$(\lambda^2 - \lambda)(x^2 + y^2 - 2xy) = (\lambda^2 - \lambda)(x + y)^2 \leq 0 \text{.}$$
\end{proof}

É interessante definir convexidade também em termos de vetores.

\begin{defi}[Vetor convexo]
Seja $a \in \B{Q}^n$ um vetor,
    \begin{itemize}
        \item se para todo $i,j,k \in [n]$, $i < j < k \so a_j \leq \frac{(j-k)a_i + (i-j)a_k}{i-k}$, $a$ é dito convexo e
        \item se para todo $i,j,k \in [n]$, $i < j < k \so a_j \geq \frac{(j-k)a_i + (i-j)a_k}{i-k}$, $a$ é dito côncavo.
    \end{itemize}
\end{defi}

Assim como a monotonicidade, a convexidade também é usualmente explorada para agilizar algoritmos, por exemplo, se um vetor é convexo podemos definir o valor mínimo do vetor com uma busca ternária ao invés de percorrer todo o vetor.  

\begin{defi}
Seja $A \in \B{Q}^{n \times m}$, definimos quatro vetores a seguir.
    \begin{itemize}
        \item O vetor de índices de máximos das linhas de $A$ guarda na posição $i$ 
              o número $\max\{j \in [m] \mid A[i][j] \geq A[i][j'] \text{ para todo } j' \in [m]\}$. 
        \item O vetor de índices de mínimos das linhas de $A$ guarda na posição $i$ 
              o número $\min\{j \in [m] \mid A[i][j] \leq A[i][j'] \text{ para todo } j' \in [m]\}$. 
        \item O vetor de índices de máximos das colunas de $A$ guarda na posição $j$ 
              o número $\max\{i \in [n] \mid A[i][j] \geq A[i'][j] \text{ para todo } i' \in [n]\}$. 
        \item O vetor de índices de mínimos das colunas de $A$ guarda na posição $j$ 
              o número $\min\{i \in [n] \mid A[i][j] \leq A[i'][j] \text{ para todo } i' \in [n]\}$. 
    \end{itemize}
\end{defi}

Note que o máximo de uma linha (ou coluna) foi definido como o maior índice que atinge o máximo e o de mínimo foi definido como o menor índice que atinge o mínimo. Esta escolha foi feita para simplificar o Lema~\ref{lema:MonotoneTotallyMonotone}, porém, os algoritmos e resultados discutidos neste trabalho funcionam (com pequenas adaptações) para diversas definições distintas destes vetores.  

Dada uma matriz, encontrar estes vetores é um problema central para este trabalho. Neste momento é interessante classificar algumas matrizes de acordo com propriedades que vão nos ajudar a calcular os vetores de mínimos e máximos de maneira especialmente eficiente.   

A Figura~\ref{figure:ConvexMonotone} resume as relações de implicação da classificação que será realizada. Os conceitos ilustrados nela serão apresentados a seguir.

\begin{figure}[t]
    \centering
    \tikzset{
  basic/.style  = {draw, text width=2cm, drop shadow, font=\sffamily, rectangle},
  root/.style   = {basic, rounded corners=2pt, thin, align=center,
                   fill=green!30},
  level 2/.style = {basic, rounded corners=6pt, thin,align=center, fill=green!60,
                   text width=8em},
  level 3/.style = {basic, thin, align=left, fill=pink!60, text width=6.5em}
}

\begin{tikzpicture}[
  level 1/.style={sibling distance=40mm},
  edge from parent/.style={->,draw},
  >=latex]

% root of the the initial tree, level 1
\node[root] {Monge convexa}
% The first level, as children of the initial tree
  child {node[level 2] (c1) {Totalmente monótona convexa por linhas}}
  child {node[level 2] (c2) {Totalmente monótona convexa por colunas}};

% The second level, relatively positioned nodes
\begin{scope}[every node/.style={level 3}]
\node [below of = c1, xshift=15pt, yshift=-26pt] (c11) {Monótona decrescente nos máximos das linhas};
\node [below of = c11, yshift=-32pt] (c12) {Monótona crescente nos mínimos das linhas};

\node [below of = c2, xshift=15pt, yshift=-26pt] (c21) {Monótona decrescente nos máximos das colunas};
\node [below of = c21, yshift=-32pt] (c22) {Monótona crescente nos mínimos das colunas};
\end{scope}

% lines from each level 1 node to every one of its "children"
\foreach \value in {1,2}
  \draw[->] (c1.195) |- (c1\value.west);

\foreach \value in {1,2}
  \draw[->] (c2.195) |- (c2\value.west);
\end{tikzpicture}

    \caption{Comportamento dos vetores de índices ótimos em relação à convexidade.} \label{figure:ConvexMonotone}
\end{figure}

\begin{defi}[Matriz monótona]
Seja $A \in \B{Q}^{n \times m}$ uma matriz. Se $A$ tiver o vetor de índices de mínimos das linhas monótono, $A$ é dita monótona nos mínimos das linhas. 

Valem também as definições análogas para máximos ou colunas e pode-se especificar monotonicidade crescente ou decrescente.
\end{defi}

\begin{defi}[Matriz totalmente monótona]
Seja $A \in \B{Q}^{n \times m}$ uma matriz.
    \begin{itemize}
        \item Se~$A[i'][j] \leq A[i'][j']$ implica~$A[i][j] \leq A[i][j']$ para todo~${i,i' \in [n]}$ e~${j,j' \in [m]}$ onde~${i<i'}$ e~${j<j'}$,~$A$ é monótona convexa nas linhas.
        \item Se~$A[i][j'] \leq A[i'][j']$ implica~$A[i][j] \leq A[i'][j]$ para todo~${i,i' \in [n]}$ e~${j,j' \in [m]}$ onde~${i<i'}$ e~${j<j'}$,~$A$ é monótona convexa nas colunas.
        \item Se~$A[i'][j] > A[i'][j']$ implica~$A[i][j] > A[i][j']$ para todo~${i,i' \in [n]}$ e~${j,j' \in [m]}$ onde~${i<i'}$ e~${j<j'}$,~$A$ é monótona côncava nas linhas.
        \item Se~$A[i][j'] > A[i'][j']$ implica~$A[i][j] > A[i'][j]$ para todo~${i,i' \in [n]}$ e~${j,j' \in [m]}$ onde~${i<i'}$ e~${j<j'}$,~$A$ é monótona côncava nas colunas.
    \end{itemize}
\end{defi}

O motivo do uso dos termos ``convexa'' e ``côncava'' em relação a matrizes durante o texto são justificados pelo Teorema~\ref{theo:MongeConvex_and_Convex}. Note que se uma matriz é totalmente monótona, todas as suas submatrizes são totalmente monótonas no mesmo sentido.

\begin{lema} \label{lema:MonotoneTotallyMonotone}
Se $A \in \B{Q}^{n \times m}$ é uma matriz totalmente monótona convexa nas linhas, toda submatriz de $A$ é monótona crescente nos mínimos das linhas e monótona decrescente nos máximos das linhas.  

Se $A$ é totalmente monótona côncava nas linhas, toda submatriz de $A$ é monótona decrescente nos mínimos das linhas e monótona crescente nos máximos das linhas.  

As afirmações valem identicamente em termos de colunas.
\end{lema}

\begin{proof}
Considere uma matriz $A$ totalmente monótona convexa nas linhas. Sejam $i$ e $i'$ índices de linhas de $A$ onde $i < i'$. Chamamos de $j$ o índice de máximo da linha $i$ e de $j'$ o índice de máximo da linha $i'$. Queremos provar que os máximos são decrescentes, portanto, vamos supor por absurdo que $j < j'$. Com isso, teremos $A[i][j'] < A[i][j]$ e $A[i'][j] \leq A[i'][j']$. Porém, já que $A$ é monótona convexa nas linhas, a segunda desigualdade implica em $A[i][j] \leq A[i][j']$, que contradiz a primeira. Portanto, os índices de máximos são decrescentes.  

Agora, considere novamente dois índices $i$ e $i'$ quaisquer de linhas de $A$ onde $i < i'$. Denotamos por $j$ o índice de mínimo da linha $i'$ e por $j'$ o índice de mínimo da linha $i$ (note e a inversão no uso de $'$). Vamos supor por absurdo que $j < j'$ e teremos $A[i'][j] \leq A[i'][j']$ e $A[i][j'] < A[i][j]$. E, novamente, usando o fato de que $A$ é monótona convexa nas linhas, obtivemos uma contradição.  

Finalmente, se $A'$ é uma submatriz de $A$, então $A'$ é totalmente monótona convexa nas linhas, portanto monótona crescente nos mínimos das linhas e monótona decrescente nos máximos das linhas.

As demonstrações no caso côncavo e nos casos relacionados a colunas são análogas.
\end{proof}

\begin{defi}[Monge Convexidade] \label{defi:MatrizMonge}
Seja $A \in \B{Q}^{n \times m}$.
    \begin{enumerate}
        \item Se vale $A[i][j] + A[i'][j'] \leq A[i][j'] + A[i'][j]$ para todo~${i,i' \in [n]}$ e~${j,j' \in [m]}$ onde~${i<i'}$ e~${j<j'}$ então~$A$ é dita Monge convexa.
        \item Se vale $A[i][j] + A[i'][j'] \geq A[i][j'] + A[i'][j]$ para todo~${i,i' \in [n]}$ e~${j,j' \in [m]}$ onde~${i<i'}$ e~${j<j'}$ então~$A$ é dita Monge côncava.
    \end{enumerate}
\end{defi}

A desigualdade que define as matrizes Monge é conhecida pelos nomes ``Propriedade de Monge'' (em inglês, ``Monge Property'')~\cite{Burkard:1996} ou ``Desigualdade Quadrangular'' (em inglês, ``Quadrangle Inequality'')~\cite{Yao:1980,Bein:2009}.

\begin{lema} \label{Monge:MCtoTM}
Se~$A$ é Monge convexa,~$A$ é totalmente monótona convexa tanto nas linhas quanto nas colunas.  

Se~$A$ é Monge côncava,~$A$ é totalmente monótona côncava tanto nas linhas quanto nas colunas.
\end{lema}

\begin{proof}
Seja~$A$ uma matriz Monge convexa. Suponha que vale, para certos $i,i' \in [n]$ e $j,j' \in [m]$ onde $i < i'$ e $j < j'$, $A[i'][j] \leq A[i'][j']$, então, somamos esta desigualdade à definição de Monge convexa e obtemos $A[i][j] \leq A[i][j']$, ou seja, $A$ é totalmente monótona convexa nas linhas.  

Por outro lado, se vale, para certos $i,i' \in [n]$ e $j,j' \in [m]$ com~${i < i'}$ e~${j < j'}$, $A[i][j'] \leq A[i'][j']$, somamos esta desigualdade à definição de Monge convexa e obtemos $A[i][j] \leq A[i'][j]$, assim, $A$ é totalmente monótona convexa nas colunas.  

A prova para o caso côncavo é análoga.
\end{proof}

\begin{theo} \label{Monge:theo+1}
Seja~$A \in \B{Q}^{n \times m}$.

Vale~${A[i][j] + A[i+1][j+1] \leq A[i][j+1] + A[i+1][j]}$ para todo~${i \in [n-1]}$ e~${j \in [m-1]}$ se e somente se~$A$ é Monge convexa.

Vale~${A[i][j] + A[i+1][j+1] \geq A[i][j+1] + A[i+1][j]}$ para todo~${i \in [n-1]}$ e~${j \in [m-1]}$ se e somente se~$A$ é Monge côncava.
\end{theo}

\begin{proof}
Seja~$A \in \B{Q}^{n \times m}$. Se~$A$ é Monge convexa, vale~${A[i][j] + A[i+1][j+1] \leq A[i][j+1] + A[i+1][j]}$ para todo~${i \in [n-1]}$ e~${j \in [m-1]}$. Vamos mostrar o outro lado desta implicação.

Sejam~${i \in [n-1]}$ e~${j \in [m-1]}$ quaisquer, vamos provar que~${A[i][j] + A[i+a][j+1] \leq A[i][j+1] + A[i+a][j]}$ para todo~${0 < a \leq n - i}$ com indução em~$a$. A base, onde~${a = 1}$ vale pela hipótese. Quando~${1 < a \leq n-i}$ assumimos que a tese vale com~${a-1}$ e temos~${A[i][j] + A[i+a-1][j+1] \leq A[i][j+1] + A[i+a-1][j]}$ e, já que~${i + a - 1 \in [n-1]}$, vale~$A[i+a-1][j] + A[i+a][j+1] \leq A[i+a-1][j+1] + A[i+a][j]$ e, somando as duas inequações, obtemos~${A[i][j] + A[i+a][j+1] \leq A[i][j+1] + A[i+a][j]}$. Isso conclui a prova proposta neste parágrafo.

Agora, sejam~${i \in [n-1]}$ e~${j \in [m-1]}$ quaisquer, vamos provar que~${A[i][j] + A[i+a][j+b] \leq A[i][j+b] + A[i+a][j+b]}$ para todo~${0 < a \leq n-i}$ e~${0 < b \leq m-j}$ por indução~$b$. A base desta indução, onde~${b = 1}$, foi provada no parágrafo anterior. Se~${1 < b \leq m-j}$ assumimos que a tese vale para~${b-1}$, escrevemos~${A[i][j] + A[i+a][j+b-1] \leq A[i][j+b-1] + A[i+a][j]}$, pela prova do parágrafo anterior, vale~$A[i][j+b-1] + A[i+a][j+b] \leq A[i][j+b] + A[i+a][j+b-1]$ e, mais uma vez, somando as duas equações provamos a desigualdade~${A[i][j] + A[i+a][j+b] \leq A[i][j+b] + A[i+a][j]}$. Com isso provamos que~$A$ é Monge convexa.

A prova para o caso côncavo segue análogamente.
\end{proof}

As matrizes Monge são usadas para resolver uma série de problemas que serão explorados aqui. A condição de Monge é a mais forte apresentada neste trabalho e alguns dos algoritmos apresentados não dependem dela, apenas da monotonicidade ou total monotonicidade, ainda assim, ela leva a resultados úteis que nos permitem provar a pertinência dos algoritmos a problemas, mesmo que o algoritmo usado não se utilize da condição diretamente.  

Como consequência desta utilidade, iremos discutir um problema que será resolvido com um algoritmo apresentado somente na Seção~\ref{SMAWK}, o algoritmo SMAWK. Ele não será explicado neste momento, utilizamos ele como caixa preta. Desta forma, poderemos introduzir estes resultados que são importantes em vários momentos deste texto e na aplicação prática dos conhecimentos discutidos aqui de forma suave e motivada.   

\begin{prob} \label{Monge:example}
Definimos a função de custo~$c$ de cada vetor~$v$ como~$c(v) = \left( \sum\limits_{i=1}^{|v|} v_i \right)^2$.  

Dados dois inteiros~$k$ e~$n$ e um vetor~$v \in \B{Q}_+^n$, queremos particionar o vetor~$v$ em~$k$ subvetores de forma a minimizar a soma dos custos das partes. Formalmente, queremos escolher um particionamento~$P_1,P_2,\dots,P_k$ de~$v$ em subvetores que minimize~$\sum \limits_{i=1}^k c(v_{P_i})$.
\end{prob}

Definimos a matriz~$A$ para todo~$i,j \in [0 \tdots n]$ onde~${A[i][j] = c(v[i+1 \tdots j])}$ para todo~$i \leq j$ e~$A[i][j] = +\infty$ caso contrário. A matriz não precisa ser explícitamente calculada, pré-calculamos em~$\Cl{O}(n)$ o vetor~$a$ tal que~${a_i = \sum_{k=1}^i v_k}$ para todo~${i \in [0 \tdots n]}$. Com o vetor~$a$ conseguimos calcular cada entrada da matriz~$A$ em~$\Cl{O}(1)$ quando necessário.

Podemos resolver o Problema~\ref{Monge:example} com programação dinâmica. Um subproblema de parâmetros~$i$ e~$\ell$ é da forma: Melhor particionamento do vetor~${v[i+1 \tdots n]}$ em~$\ell$ partes. Definimos a matriz~${E \in \B{Q}^{[k] \times [0 \tdots n]}}$ de respostas desses subproblemas, assim, se~$\ell$ e~$i$ definem um subproblema o seu valor ótimo é guardado em~$E[\ell][i]$. Todas as outras entradas da matriz~$E$ têm valor~$+\infty$. Escrevemos~$E$ como uma recorrência, para todo~$\ell \in [k]$ e~$i \in [0 \tdots n]$,

\begin{equation*}
E[\ell][i] = \begin{cases}
    A[i][n]                                                     & \text{, se } \ell = 1 \text{, }\\
    \min\limits_{j=i}^n A[i][j] + E[\ell-1][j]                  & \text{, se } \ell \leq k \text{ e }\\
    +\infty                                                     & \text{, caso contrário.}
\end{cases}
\end{equation*}

É fácil resolver a recorrência definida acima em tempo~$\Cl{O}(kn^2)$. Vamos simplificar a definição de~$E$. Fixados~${i \in [0 \tdots n]}$ e~${\ell \in [1 \tdots k]}$, se~$j < i$ então~${A[i][j] = +\infty}$ e podemos escrever~${E[\ell][i] = \min\limits_{j=0}^n A[i][j] + E[\ell-1][j]}$. Definimos a matriz~$B_\ell$ onde, para todo~${j \in [0 \tdots n]}$, a entrada~${B_\ell[i][j] = A[i][j] + E[\ell-1][j]}$. Além disso, definimos a matriz~$B_1$ onde, para todo~${i,j \in [0 \tdots n]}$ vale~${B_1[i][j] = A[i][j] + A[i][n]}$. Note que~${\min\limits_{j=0}^n B_1[i][j]} = {A[i][i] + A[i][n]} = {A[i][n]} = {E[1][i]}$. Desta forma, para todo~$\ell \in [k]$ e~$i \in [0 \tdots n]$ vale~${E[\ell][i] = \min\limits_{j=0}^n B_\ell[i][j]}$.

Com esta formulação, reduzimos o problema original a encontrar os mínimos das linhas de~$B_\ell$ para todo~$l \in [k]$. O algoritmo SMAWK encontra mínimos de linhas de matrizes~$n+1 \times n+1$ totalmente monótonas por linhas em tempo~$\Cl{O}(n)$. Vamos mostrar que as matrizes~$B_\ell$ são totalmente monótonas convexas por linhas para podermos aplicar o SMAWK.

\begin{lema} \label{Monge:keepConvex}
Sejam~$A,B \in \B{Q}^{n \times m}$ matrizes e~$c \in \B{Q}^m$ um vetor tais que para todo~$i \in [n]$ e~$j \in [m]$,~${B[i][j] = A[i][j] + c[j]}$. Se~$A$ é Monge convexa,~$B$ é Monge convexa.  

O mesmo resultado vale se~$c \in \B{Q}^n$ e~${B[i][j] = A[i][j] + c[i]}$. Os resultados análogos valem nos casos de concavidade.
\end{lema}

\begin{proof}
Sejam $A,B$ e $b$ definidos como no enunciado do teorema. Suponha que $A$ é Monge convexa. Vale, para quaisquer~${1 \leq i < i' \leq n}$ e~${1 \leq j < j' \leq m}$,~${A[i][j] + A[i'][j'] \leq A[i'][j] + A[i][j']}$, logo, vale~${A[i][j] + b[j] + A[i'][j'] + b[j'] \leq A[i'][j] + b[j'] + A[i][j'] + b[j]}$ que é~${B[i][j] + B[i'][j'] \leq B[i'][j] + B[i][j']}$. A prova para o caso onde~$c \in \B{Q}^n$ e~${B[i][j] = A[i][j] + c[i]}$ é análoga, bem como as provas para os casos côncavos.
\end{proof}

Suponha que~$A$ é Monge convexa. Todas as matrizes~$B_\ell$ se encaixam perfeitamente nas hipóteses do Lema~\ref{Monge:keepConvex} e, por isso, são Monge convexas, portanto, totalmente monótonas convexas por linhas. Basta provar que~$A$ é Monge convexa.

\begin{theo} \label{theo:MongeConvex_and_Convex}
Sejam~$A \in \B{Q}^{n \times n}$ uma matriz, $w \in \B{Q}_+^n$ um vetor e~${g : \B{Q} \to \B{Q}}$ uma fução tais que para todo~$i,j \in [n]$ vale~${A[i][j] = g\left(\sum\limits_{k=1}^j w_k - \sum\limits_{k=1}^i w_k\right)}$. Se~$g$ é convexa,~$A$ é Monge convexa. Similarmente, se~$g$ é côncava,~$A$ é Monge côncava.
\end{theo}

Antes de demonstrar este teorema, vamos provar que~$A$ é Monge convexa utilizando o resultado. Considere a função~$g$ tal que~$g(x) = x^2$ se~$x \geq 0$ e~$g(x) = +\infty$ caso contrário. Vale, para todo~$i,j \in [n]$,~${A[i][j] = g\left(\sum\limits_{k=1}^j v_k - \sum\limits_{k=1}^i v_k \right)}$. Vamos provar que~$g$ é convexa. Sejam~$x \leq y \in \B{Q}$ e~$\lambda \in [0,1]$, escrevemos~$z = \lambda x + (1-\lambda)y$. Se~$0 \leq x$, pela Proposição~\ref{Monge:sqConv}, sabemos~$g(z) \leq \lambda g(x) + (1-\lambda) g(y)$. Se~$x < 0$, vale~$\lambda g(x) + (1-\lambda)g(y) = +\infty \geq g(z)$, independente do valor de~$g(z)$. Assim,~$g$ é convexa e aplicamos o teorema para concluir que~$A$ é Monge convexa.

Com isso já que o nosso problema se reduziu a encontrar, para todos os~$\ell \in [k]$ os mínimos das linhas da matriz~$B_\ell$ e estas são Monge convexas, elas também são totalmente monótonas convexas por linhas e podemos encontrar seus máximos em~$\Cl{O}(n)$, resolvendo o problema todo em~$\Cl{O}(kn)$.

Nos falta provar o Teorema~\ref{theo:MongeConvex_and_Convex}.

\begin{proof}
Sejam $A$ e $g$ quaisquer que respeitem as condições do enunciado. Sejam ainda~$i,i',j,j' \in [n]$ onde~$i < i'$ e~$j < j'$. Escrevemos~$a = \sum \limits_{k=1}^{i'} w_k - \sum \limits_{k=1}^{i} w_k$,~$b = \sum \limits_{k=1}^{j'} w_k - \sum \limits_{k=1}^{j} w_k$ e~$z = \sum \limits_{k=1}^{j} w_k - \sum \limits_{k=1}^{i'} w_k$. Desta forma, temos~$g(z) = A[i'][j]$,~$g(z+a+b) = A[i][j']$,~$g(z+a) = A[i][j]$ e~$g(z+b) = A[i'][j']$, portanto, $g(z+a) + g(z+b) \leq g(z) + g(z+a+b)$ se e somente se $A[i][j] + A[i'][j'] \leq A[i][j'] + A[i'][j]$ ($A$ é Monge convexa).

Consideramos o caso onde~$0 < a \leq b$. Temos~$0 < a \leq b < a+b$, ou seja,~$z < z+a \leq z+b < z+a+b$. Definimos~$\lambda = \frac{a}{a+b}$. Já que~$z+a = \lambda z + (1-\lambda)(z+a+b)$ e~$z+b = (1-\lambda)z + \lambda(z+a+b)$, por convexidade de~$g$, obtemos~$g(z+a) \leq \lambda g(z) + (1 - \lambda) g(z+a+b)$ e~$g(z+b) \leq \lambda g(z+a+b) + (1 - \lambda) g(z)$. Somando, obtemos~$g(z+a) + g(z+b) \leq g(z) + g(z+a+b)$.  

Se considerarmos o caso onde~$0 < b \leq a$, seguimos o mesmo raciocínio e obtemos, novamente,~$g(z+a) + g(z+b) \leq g(z) + g(z+a+b)$. Falta considerar o caso onde~$0 = a = b$, neste caso,~$g(z) = g(z+a) = g(z+b) = g(z+a+b)$ e vale~$(z+a) + g(z+b) \leq g(z) + g(z+a+b)$. Portanto, $A$ é Monge convexa.
\end{proof}
