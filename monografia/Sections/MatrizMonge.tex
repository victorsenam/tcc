\section{Matrizes Monge e monotonicidade}
\label{Monge}

%----------------------------------------------------------------------------------------

Nesta seção serão apresentados e explorados os conceitos de monotonicidade, convexidade e matrizes Monge. Além disso, alguns resultados referentes a estes conceitos serão demonstrados. As definições e os resultados desta seção são fundamentais para o desenvolvimento do restante do trabalho.  Na Subseção~\ref{Monge:Basic} apresentamos vários destes conceitos básicos. Na Subseção~\ref{Monge:Monge} apresenta as matrizes Monge e as relaciona com os conceitos básicos já apresentados. A Subseção~\ref{Monge:Triangular} define e discute brevemente a ideia de matrizes triangulares.

%----------------------------------------------------------------------------------------

\subsection{Conceitos básicos} \label{Monge:Basic}

\begin{defi}[Vetor monótono]
Um vetor~${ a \in \B{Q}^n }$ é dito monótono quando vale uma das propriedades abaixo.
\begin{itemize}
    \item Se, para todo $i,j \in [n]$, $i < j$ implica~$a_i \leq a_j$, $a$ é dito monótono crescente (ou só crescente).
    \item Se, para todo $i,j \in [n]$, $i < j$ implica~$a_i \geq a_j$, $a$ é dito monótono decrescente (ou só decrescente).
\end{itemize}
\end{defi}

Sabemos que a monotonicidade de vetores pode ser aproveitada para agilizar alguns algoritmos importantes. Por exemplo, a busca binária pode ser interpretada como uma otimização da busca linear para vetores monótonos. 

\begin{defi}[Função convexa]
Seja $g : \B{Q} \to \B{Q}$ uma função.
\begin{itemize}
    \item Se, para todo par de pontos $x,y \in \B{Q}$ e $\lambda \in \B{Q}$ que respeita $0 \leq \lambda \leq 1$, vale a desigualdade~${g(\lambda x + (1 - \lambda)y) \leq \lambda g(x) + (1 - \lambda) g(y)}$, $g$ é dita convexa.
    \item Se, para todo par de pontos $x,y \in \B{Q}$ e $\lambda \in \B{Q}$ que respeita $0 \leq \lambda \leq 1$, vale a desigualdade~${g(\lambda x + (1 - \lambda)y) \geq \lambda g(x) + (1 - \lambda) g(y)}$, $g$ é dita côncava.
\end{itemize}
\end{defi}

\begin{prop} \label{Monge:sqConv}
A função $g(x) = x^2$ é convexa. 
\end{prop}

\begin{proof}
Tome quaisquer três valores~$x,y$ e~$\lambda \in \B{Q}$ de forma que~${ 0 \leq \lambda \leq 1 }$. Queremos mostrar que vale~${ (\lambda x + (1 - \lambda)y)^2 \leq \lambda x^2 + (1 - \lambda) y^2 }$. Expandindo o lado esquerdo da desigualdade obtemos~${ \lambda^2x^2 + (1 - \lambda)^2y^2 + 2\lambda(1-\lambda)xy \leq \lambda x^2 + (1 - \lambda)y^2 }$ o que equivale a~${ (\lambda^2 - \lambda)x^2 + ((1-\lambda)^2 - (1 - \lambda))y^2 + 2(\lambda - \lambda^2)xy \leq 0 }$ que é verdade uma vez que~${ (\lambda^2 - \lambda)(x^2 + y^2 - 2xy) = (\lambda^2 - \lambda)(x + y)^2 \leq 0 }$.
\end{proof}

É interessante definir convexidade também em termos de vetores.

\begin{defi}[Vetor convexo]
Seja $a \in \B{Q}^n$ um vetor,
\begin{itemize}
    \item Se, para todo~$i,j,k \in [n]$, $i < j < k$ implica~$a_j \leq \frac{(j-k)a_i + (i-j)a_k}{i-k}$, $a$ é dito convexo e
    \item Se, para todo $i,j,k \in [n]$, $i < j < k$ implica~$a_j \geq \frac{(j-k)a_i + (i-j)a_k}{i-k}$, $a$ é dito côncavo.
\end{itemize}
\end{defi}

Assim como a monotonicidade, a convexidade também é usualmente explorada para agilizar algoritmos. Por exemplo, se um vetor é convexo podemos encontrar o valor mínimo do vetor com uma busca ternária em vez de percorrer todo o vetor.  

\begin{defi}
Seja $A \in \B{Q}^{n \times m}$. Definimos quatro vetores a seguir.
\begin{itemize}
    \item O vetor de índices de máximos das linhas de $A$ guarda na posição $i$ 
          o número $\max\{j \in [m] \mid A[i][j] \geq A[i][j'] \text{ para todo } j' \in [m]\}$. 
    \item O vetor de índices de mínimos das linhas de $A$ guarda na posição $i$ 
          o número $\min\{j \in [m] \mid A[i][j] \leq A[i][j'] \text{ para todo } j' \in [m]\}$. 
    \item O vetor de índices de máximos das colunas de $A$ guarda na posição $j$ 
          o número $\max\{i \in [n] \mid A[i][j] \geq A[i'][j] \text{ para todo } i' \in [n]\}$. 
    \item O vetor de índices de mínimos das colunas de $A$ guarda na posição $j$ 
          o número $\min\{i \in [n] \mid A[i][j] \leq A[i'][j] \text{ para todo } i' \in [n]\}$. 
\end{itemize}
\end{defi}

Note que o máximo de uma linha (ou coluna) foi definido como o maior índice que atinge o máximo e o de mínimo foi definido como o menor índice que atinge o mínimo. Esta escolha foi feita para simplificar o Lema~\ref{lema:MonotoneTotallyMonotone} a frente, porém os algoritmos e resultados discutidos neste trabalho funcionam (com pequenas adaptações) para diversas definições distintas destes vetores.  

Dada uma matriz, encontrar estes vetores é um problema central para este trabalho. Neste momento é interessante classificar algumas matrizes de acordo com propriedades que vão nos ajudar a calcular os vetores de mínimos e máximos de maneira especialmente eficiente.   

A Figura~\ref{figure:ConvexMonotone} resume as relações de implicação da classificação que será realizada. Os conceitos ilustrados nela serão apresentados a seguir.

\begin{figure}[t]
    \centering
    \tikzset{
  basic/.style  = {draw, text width=2cm, drop shadow, font=\sffamily, rectangle},
  root/.style   = {basic, rounded corners=2pt, thin, align=center,
                   fill=green!30},
  level 2/.style = {basic, rounded corners=6pt, thin,align=center, fill=green!60,
                   text width=8em},
  level 3/.style = {basic, thin, align=left, fill=pink!60, text width=6.5em}
}

\begin{tikzpicture}[
  level 1/.style={sibling distance=40mm},
  edge from parent/.style={->,draw},
  >=latex]

% root of the the initial tree, level 1
\node[root] {Monge convexa}
% The first level, as children of the initial tree
  child {node[level 2] (c1) {Totalmente monótona convexa por linhas}}
  child {node[level 2] (c2) {Totalmente monótona convexa por colunas}};

% The second level, relatively positioned nodes
\begin{scope}[every node/.style={level 3}]
\node [below of = c1, xshift=15pt, yshift=-26pt] (c11) {Monótona decrescente nos máximos das linhas};
\node [below of = c11, yshift=-32pt] (c12) {Monótona crescente nos mínimos das linhas};

\node [below of = c2, xshift=15pt, yshift=-26pt] (c21) {Monótona decrescente nos máximos das colunas};
\node [below of = c21, yshift=-32pt] (c22) {Monótona crescente nos mínimos das colunas};
\end{scope}

% lines from each level 1 node to every one of its "children"
\foreach \value in {1,2}
  \draw[->] (c1.195) |- (c1\value.west);

\foreach \value in {1,2}
  \draw[->] (c2.195) |- (c2\value.west);
\end{tikzpicture}

    \caption{Comportamento dos vetores de índices ótimos em relação à convexidade.} \label{figure:ConvexMonotone}
\end{figure}

\begin{defi}[Matriz monótona]
Seja $A \in \B{Q}^{n \times m}$ uma matriz. Se $A$ tiver o vetor de índices de mínimos das linhas monótono, $A$ é dita monótona nos mínimos das linhas. Valem também as definições análogas para máximos ou colunas e pode-se especificar monotonicidade crescente ou decrescente.
\end{defi}

\begin{defi}[Matriz totalmente monótona]
Seja $A \in \B{Q}^{n \times m}$ uma matriz.
    \begin{itemize}
        \item Se~$A[i'][j] \leq A[i'][j']$ implica~$A[i][j] \leq A[i][j']$ para todo~${i,i' \in [n]}$ e~${j,j' \in [m]}$ onde~${i<i'}$ e~${j<j'}$,~$A$ é totalmente monótona convexa nas linhas.
        \item Se~$A[i][j'] \leq A[i'][j']$ implica~$A[i][j] \leq A[i'][j]$ para todo~${i,i' \in [n]}$ e~${j,j' \in [m]}$ onde~${i<i'}$ e~${j<j'}$,~$A$ é totalmente monótona convexa nas colunas.
        \item Se~$A[i'][j] > A[i'][j']$ implica~$A[i][j] > A[i][j']$ para todo~${i,i' \in [n]}$ e~${j,j' \in [m]}$ onde~${i<i'}$ e~${j<j'}$,~$A$ é totalmente monótona côncava nas linhas.
        \item Se~$A[i][j'] > A[i'][j']$ implica~$A[i][j] > A[i'][j]$ para todo~${i,i' \in [n]}$ e~${j,j' \in [m]}$ onde~${i<i'}$ e~${j<j'}$,~$A$ é totalmente monótona côncava nas colunas.
    \end{itemize}
\end{defi}

O uso dos termos ``convexa'' e ``côncava'' em relação a matrizes durante o texto é explicado pelo Teorema~\ref{theo:MongeConvex_and_Convex}. Note que se uma matriz é totalmente monótona todas as suas submatrizes são totalmente monótonas no mesmo sentido.

\begin{lema} \label{lema:MonotoneTotallyMonotone}
Se $A \in \B{Q}^{n \times m}$ é uma matriz totalmente monótona convexa nas linhas, toda submatriz de $A$ é monótona crescente nos mínimos das linhas e monótona decrescente nos máximos das linhas. Se $A$ é totalmente monótona côncava nas linhas, toda submatriz de $A$ é monótona decrescente nos mínimos das linhas e monótona crescente nos máximos das linhas. As afirmações valem similarmente em termos de colunas.
\end{lema}

\begin{proof}
Considere uma matriz $A$ totalmente monótona convexa nas linhas. Sejam $i$ e $i'$ índices de linhas de $A$ onde $i < i'$. Chamamos de $j$ o índice de máximo da linha $i$ e de $j'$ o índice de máximo da linha $i'$. Queremos provar que os índices de máximo são decrescentes, portanto, vamos supor por absurdo que $j < j'$. Com isso, teremos que~${ A[i][j'] < A[i][j] }$ e $A[i'][j] \leq A[i'][j']$. Porém, já que $A$ é monótona convexa nas linhas, a segunda desigualdade implica em $A[i][j] \leq A[i][j']$, que contradiz a primeira. Portanto, os índices de máximos são decrescentes.  

Agora, considere novamente dois índices $i$ e $i'$ quaisquer de linhas de $A$ onde $i < i'$. Denotamos por $j$ o índice de mínimo da linha $i'$ e por $j'$ o índice de mínimo da linha $i$ (note e inversão do nome dos índices). Vamos supor por absurdo que $j < j'$ e teremos que $A[i'][j] \leq A[i'][j']$ e $A[i][j'] < A[i][j]$. Novamente, usando o fato de que $A$ é monótona convexa nas linhas, obtivemos uma contradição.  

Finalmente, se $A'$ é uma submatriz de $A$, então $A'$ é totalmente monótona convexa nas linhas, portanto monótona crescente nos mínimos das linhas e monótona decrescente nos máximos das linhas.

As demonstrações no caso côncavo e nos casos relacionados a colunas são análogas.
\end{proof}

%----------------------------------------------------------------------------------------
\subsection{Matrizes Monge} \label{Monge:Monge}

\begin{defi}[Monge convexidade] \label{defi:MatrizMonge}
Seja $A \in \B{Q}^{n \times m}$.
    \begin{enumerate}
        \item Se vale $A[i][j] + A[i'][j'] \leq A[i][j'] + A[i'][j]$ para todo~${i,i' \in [n]}$ e~${j,j' \in [m]}$ onde~${i<i'}$ e~${j<j'}$, então~$A$ é dita Monge convexa.
        \item Se vale $A[i][j] + A[i'][j'] \geq A[i][j'] + A[i'][j]$ para todo~${i,i' \in [n]}$ e~${j,j' \in [m]}$ onde~${i<i'}$ e~${j<j'}$, então~$A$ é dita Monge côncava.
    \end{enumerate}
\end{defi}

A desigualdade que define as matrizes Monge é conhecida pelos nomes ``Propriedade de Monge'' (em inglês, ``Monge Property'')~\cite{Burkard:1996} ou ``Desigualdade Quadrangular'' (em inglês, ``Quadrangle Inequality'')~\cite{Yao:1980,Bein:2009}. O lema a seguir mostra que esta propriedade é mais forte do que a total monotonicidade. 

\begin{lema} \label{Monge:MCtoTM}
Se~$A$ é Monge convexa,~$A$ é totalmente monótona convexa tanto nas linhas quanto nas colunas. Se~$A$ é Monge côncava,~$A$ é totalmente monótona côncava tanto nas linhas quanto nas colunas.
\end{lema}

\begin{proof}
Seja~$A$ uma matriz Monge convexa. Suponha que vale, para certos~$i,i' \in [n]$ e~${ j,j' \in [m] }$ onde $i < i'$ e $j < j'$, que~$A[i'][j] \leq A[i'][j']$. Então, somamos esta desigualdade à definição de Monge convexa, obtendo que~$A[i][j] \leq A[i][j']$, ou seja, $A$ é totalmente monótona convexa nas linhas.  

Por outro lado se~${ A[i][j'] \leq A[i'][j] }$ para certos~${ i,i' \in [n] }$ e~${ j,j' \in [m] }$ com~${i < i'}$ e~${j < j'}$, somamos esta desigualdade à da definição de Monge convexa e obtemos que~${ A[i][j] \leq A[i'][j] }$, assim, $A$ é totalmente monótona convexa nas colunas.  

A prova para o caso côncavo é análoga.
\end{proof}

Os algoritmos estudados durante este trabalho não utilizam diretamente a condição de Monge, apenas a total monotonicidade, a monotonicidade ou condições parecidas. Apesar disso, a condição de Monge é útil uma vez que implica na total monotonicidade e tem propriedades interessantes que facilitam a modelagem de problemas ou a prova da pertinência de alguns problemas nos algoritmos que nos interessam. O Teorema~\ref{Monge:theo+1}, por exemplo, mostra uma condição simples e de fácil visualização que é equivalente à condição de Monge.

\begin{theo} \label{Monge:theo+1}
Seja~$A \in \B{Q}^{n \times m}$. Vale que

\begin{enumerate}[(a)]
\item Vale~${A[i][j] + A[i+1][j+1] \leq A[i][j+1] + A[i+1][j]}$ para todo~${i \in [n-1]}$ e~${j \in [m-1]}$ se e somente se~$A$ é Monge convexa. \label{Monge:theo+1:convex}

\item Vale~${A[i][j] + A[i+1][j+1] \geq A[i][j+1] + A[i+1][j]}$ para todo~${i \in [n-1]}$ e~${j \in [m-1]}$ se e somente se~$A$ é Monge côncava. \label{Monge:theo+1:concave}
\end{enumerate}
\end{theo}

\begin{proof}
Se~$A \in \B{Q}^{n \times m}$ é uma matriz Monge convexa, trivialmente vale, para todo par de~${ i \in [n-1] }$ e~${ j \in [m-1] }$, a desigualdade descrita pelo enunciado do teorema, isto é~${ A[i][j] + A[i+1][j+1] \leq A[i][j+1] + A[i+1][j] }$. Vamos mostrar o outro lado do Teorema.

Tomamos uma matriz~$A \in \B{Q}^{n \times m}$ e dois índices~${ i \in [n-1] }$ e~${ j \in [m-1] }$. Vamos provar, com indução em~$a$, que~${ A[i][j] + A[i+1][j+1] \leq A[i][j+1] + A[i+a][j] }$ para todo~${ a \in [1 \tdots n-i ] }$. A base é o caso em que~${ a = 1 }$ e ela vale por hipótese. Para~${ a \in [2 \tdots n-i] }$, assuma que a tese vale para~${ a-1 }$, ou seja,~${ A[i][j] + A[i + a - 1][j+1] \leq A[i + a - 1][j] }$. Já que~${ i + a - 1  \in [n-1] }$, temos que~${ A[i+a-1][j] + A[i+a][j+1] \leq A[i+a-1][j-1] + A[i+a][j] }$ e, somando as duas desigualdades, obtemos~${ A[i][j] + A[i+a][j+1] \leq A[i][j+1] + A[i+a][j] }$ o que conclui a prova proposta neste parágrafo.

Agora tomamos novamente dois índices~${ i \in [n-1] }$ e~${ j \in [m-1] }$. Vamos provar que vale~${ A[i][j] + A[i+a][j+b] \leq A[i][j+b] + A[i+a][j+b] }$ por indução em~$b$ para todo~${ a \in [1 \tdots n-i] }$ e~${ b \in [1 \tdots n-j] }$. A base desta indução é o caso em que~${ b = 1 }$ que foi provado no parágrafo anterior. Para~${ b \in [2 \tdots n-j] }$ assumimos que a tese vale para~${ b-1 }$, ou seja,~${ A[i][j] + A[i+a][j+b] \leq A[i][j+b] + A[i+a][j+b] }$. Então, pela prova do parágrafo anterior, vale que~${ A[i][j + b - 1] + A[i+a][j+b] \leq A[i][j+b] + A[i+a][j+b-1] }$ e, mais uma vez, somando as duas desigualdades, provamos que~${ A[i][j] + A[i+a][j+b] \leq A[i][j+b] + A[i+a][j] }$. Com isso concluímos que~$A$ é Monge convexa.

A prova de~\ref{Monge:theo+1:concave} segue analogamente.
\end{proof}

As matrizes Monge são usadas para resolver uma série de problemas que serão explorados aqui. A condição de Monge é a mais forte apresentada neste trabalho e alguns dos algoritmos apresentados não dependem dela, mas apenas da monotonicidade ou total monotonicidade. Ainda assim, ela leva a resultados úteis que nos permitem provar a adequação dos algoritmos a problemas, mesmo que o algoritmo usado não se utilize da condição diretamente.  

Como consequência desta utilidade, iremos discutir um problema que será resolvido com um algoritmo apresentado somente na Seção~\ref{SMAWK}, o algoritmo SMAWK. Ele não será explicado neste momento. Será utilizado como caixa preta. Desta forma, poderemos introduzir estes resultados que são importantes em vários momentos deste texto e na aplicação prática dos conhecimentos discutidos aqui de forma gradual e motivada.   

\begin{prob} \label{ Monge:example }
A função de custo~$c$ de cada vetor~$v$ é definida como~$c(v) = \left( \sum\limits_{i=1}^{|v|} v_i \right)^2$. Dados dois inteiros~$k$ e~$n$ e um vetor~${ v \in \B{Q}^n_+ }$, particionar o vetor~$v$ em~$k$ subvetores de forma a minimizar a soma dos custos das partes. Formalmente, escolher um particionamento~${ p_0 = 1 \leq p_1 \leq p_2 \leq \dots \leq p_k = n+1 }$ de~$v$ em subvetores que minimize~$\sum\limits_{i=1}^k c(v[p_{i-1} \tdots p_i - 1])$.
\end{prob}

Definimos a matriz~$A$ onde~${ A[i][j] = \left(\sum\limits_{\ell=1}^{j-1} v_\ell - \sum\limits_{\ell=1}^{i-1} v_\ell \right)^2 }$ para todo~${ i,j \in [n+1] }$. Quando~${ i \leq j }$, vale que~${ A[i][j] = c(v[i \tdots j-1]) }$. A matriz~$A$ não precisa ser explicitamente calculada. Pré calculamos em~$\Cl{O}(n)$ o vetor~$a$ tal que~${ a_i = \sum_{\ell=1}^{i-1} v_\ell }$ para todo~${ i \in [n+1] }$. Com este vetor~$a$, podemos calcular uma entrada~$i,j$ qualquer da matriz~$A$ em~$\Cl{O}(1)$, já que~${ A[i][j] = (a_j - a_i)^2 }$.

Podemos resolver o Problema~\ref{ Monge:example } com programação dinâmica. Um subproblema de parâmetros~$\ell$ e~$i$ é da forma: encontrar um melhor particionamento do vetor~${v[i \tdots n]}$ em~$\ell$ partes. Definimos a matriz~${E \in \B{Q}^{[k] \times [n+1]}}$ de respostas desses subproblemas, assim, se~$\ell$ e~$i$ definem um subproblema, o seu valor ótimo é guardado em~$E[\ell][i]$. Vale a seguinte recorrência para~$E$. Para todo~$\ell \in [k]$ e~$i \in [n+1]$,

\begin{equation*}
E[\ell][i] = \begin{cases}
    A[i][n+1]                                                   & \text{, se } \ell = 1 \text{ e } \\
    \min\{ A[i][j] + E[\ell-1][j] \mid j \in [i \tdots n+1] \}     & \text{, caso contrário. }
\end{cases}
\end{equation*}

Utilizando o fato de que sabemos calcular as entradas de~$A$ em~$\Cl{O}(1)$ fica fácil resolver a recorrência definida acima em tempo~${ \Cl{O}(kn^2) }$. Vamos simplificar a definição de~$E$. Para todo~${ \ell \in [2 \tdots k] }$ definimos a matriz~$B_\ell$ que guarda, em cada entrada~${ i,j \in [n+1] }$, o valor~${ A[i][j] + E[\ell-1][j] }$. Perceba que para todo~$\ell$ e~$i$ vale~${ E[\ell][i] = \min\{ B_\ell[i][j] \mid j \in [i \tdots n+1] \}}$. Se~${ j < i }$, é fácil mostrar que~${ E[j] \geq E[i] }$ e, já que~${ A[i][i] = 0 }$ e~${ A[j][i] \geq 0 }$, temos~${ B_\ell[i][j] \geq B_\ell[i][i] }$. Com isso, vale que~${ E[\ell][i] = \min\{ B_\ell[i][j] \mid j \in [n+1] \} }$. 

O parágrafo acima mostra que~$E[\ell][i]$ é o valor de mínimo da~$i$-ésima linha da matriz~$B_\ell$. Com esta formulação, reduzimos o problema original a encontrar os mínimos das linhas de cada uma das matrizes~$B_\ell$ com~$\ell \in [2 \tdots k]$. O algoritmo SMAWK encontra mínimos de linhas de matrizes~${ (n+1) \times (n+1) }$ totalmente monótonas por linhas em tempo~$\Cl{O}(n)$. Vamos mostrar que as matrizes~$B_\ell$ são totalmente monótonas convexas por linhas para podermos aplicar o SMAWK.

\begin{lema} \label{Monge:keepConvex}
Sejam~${ A,B \in \B{Q}^{n \times m} }$ matrizes e~${ c \in \B{Q}^m }$ um vetor tais que~${ B[i][j] = A[i][j] + c[j] }$ para todo~${ i \in [n] }$ e~${ j \in [m] }$. Se~$A$ é Monge convexa,~$B$ é Monge convexa. O mesmo vale se~$c \in \B{Q}^n$ e~${B[i][j] = A[i][j] + c[i]}$. Resultados análogos valem nos casos de concavidade.
\end{lema}

\begin{proof}
Sejam $A,B$ e $c$ definidos como no enunciado do teorema. Suponha que $A$ é Monge convexa. Vale, para quaisquer~${ i,i' \in [n]}$ e~${ j,j' \in [m] }$ onde~${ i < i' }$ e~${ j < j' }$ que~${A[i][j] + A[i'][j'] \leq A[i'][j] + A[i][j']}$. Somando~${ c[j] + c[j'] }$ nos dois lados obtemos a desigualdade~${A[i][j] + c[j] + A[i'][j'] + c[j'] \leq A[i'][j] + c[j'] + A[i][j'] + c[j]}$ que equivale a~${B[i][j] + B[i'][j'] \leq B[i'][j] + B[i][j']}$. A prova para o caso onde~$c \in \B{Q}^n$ e~${B[i][j] = A[i][j] + c[i]}$ é análoga, bem como as provas para os casos côncavos.
\end{proof}

Suponha que a matriz~$A$ do Problema~\ref{ Monge:example } é Monge convexa. Todas as matrizes~$B_\ell$ se encaixam perfeitamente nas hipóteses do Lema~\ref{Monge:keepConvex} e, por isso, são Monge convexas, portanto, totalmente monótonas convexas por linhas. Basta provar que~$A$ é Monge convexa.

\begin{theo} \label{theo:MongeConvex_and_Convex}
Sejam~$A \in \B{Q}^{n \times n}$ uma matriz, $w \in \B{Q}_+^n$ um vetor e~${g : \B{Q} \to \B{Q}}$ uma fução tais que para todo~$i,j \in [n]$ vale~${A[i][j] = g\left(\sum\limits_{k=1}^j w_k - \sum\limits_{k=1}^i w_k\right)}$. Se~$g$ é convexa,~$A$ é Monge convexa. Similarmente, se~$g$ é côncava,~$A$ é Monge côncava.
\end{theo}

Antes de demonstrar este teorema, vamos usá-lo para provar que a matriz~$A$ do Problema~\ref{ Monge:example } é Monge convexa. Considere a função~$g(x) = x^2$. Vale, para todo~${ i,j \in [n] }$, que~${A[i][j] = g\left(\sum\limits_{k=1}^j v_k - \sum\limits_{k=1}^i v_k \right)}$. Pela Proposição~\ref{Monge:sqConv},~$g$ é convexa. Aplicamos o Teorema~\ref{theo:MongeConvex_and_Convex} para concluir que~$A$ é Monge convexa.

Com isso, já que o nosso problema se reduziu a encontrar, para todos os~$\ell \in [k]$, os mínimos das linhas da matriz~$B_\ell$ e estas são Monge convexas, elas também são totalmente monótonas convexas por linhas e podemos encontrar seus máximos em~$\Cl{O}(n)$, resolvendo o problema todo em~$\Cl{O}(kn)$.

Resta provar o Teorema~\ref{theo:MongeConvex_and_Convex}.

\begin{proof}[Demonstração do Teorema~\ref{theo:MongeConvex_and_Convex}]
Sejam~$A$ e~$g$ uma matriz e uma função quaisquer que respeitem as condições do enunciado do Teorema~\ref{theo:MongeConvex_and_Convex}. Definimos, para todo par de índices~${ i,j \in [n] }$ a função~${ s(i,i') = \sum\limits_{k=1}^{i'} w_k - \sum\limits_{k=1}^i w_k }$. Note que desta maneira~${ A[i][j] = g(s(i,j)) }$. Sejam~${ i,i',j,j' \in [n] }$ tais que~${ i < i' }$ e~${ j < j' }$, vamos mostrar que vale a condição de Monge para estes índices, isto é,~${ A[i][j] + A[i'][j'] \leq A[i][j'] + A[i'][j] }$. Definimos os valores~${ a = s(i,i') }$,~${ b = s(j,j') }$ e~${ z = s(j,i') }$. É verdade que a condição de Monge para os índices fixados equivale à desigualdade~${ g(z+a) + g(z+b) \leq g(z) + g(z+a+b) }$, vamos provar que esta vale.

Consideramos o caso em que~${ 0 < a \leq b }$. Temos que~${ z < z+a \leq z + b < z + a + b }$. Definimos~${ \lambda = \frac{a}{a+b} }$. Já que~${ 0 \leq \lambda \leq 1 }$,~${ z + a = \lambda z + (1 - \lambda)(z+a+b) }$ e~${ z + b = (1 - \lambda)z + \lambda(z+a+b) }$, pela convexidade de~$g$, obtemos que~${ g(z+a) \leq \lambda g(z) + (1 - \lambda)g(z+a+b) }$ e também que~${ g(z+b) \leq (1-\lambda) g(z) + \lambda g(z+a+b) }$. Somando as duas desigualdades, concluímos que vale~${ g(z+a) + g(z+b) \leq g(z) + g(z+a+b) }$. No caso em que~${ 0 < b \leq a }$ o mesmo resultado segue de maneira análoga. Finalmente, no caso onde~${ 0 = a = b }$ vale que~${ g(z) = g(z+a) = g(z+b) = g(z+a+b) }$ e o resultado é trivial.
\end{proof}

%----------------------------------------------------------------------------------------

\subsection{Matrizes triangulares} \label{Monge:Triangular} 

Para alguns dos problemas e algoritmos discutidos neste trabalho, é interessante pensar em matrizes com entradas indefinidas. É possível definir uma matriz Monge com entradas arbitrárias indefinidas~\cite[Seção 9.4]{Burkard:1996}. Vamos limitar quais posições podem ser indefinidas em uma matriz de forma a mantê-las úteis e relacionadas com os resultados já obtidos nesta seção. 

\begin{defi}[Matriz triangular]
Seja~$c$ um inteiro. Uma matriz~$A \in \B{Q}^{n \times n}$ é triangular inferior em~$c$ se, para todo~${ i,j \in [n] }$,~$A[i][j]$ é indefinido se e somente se~${ i + c \leq j }$. Similarmente, se, para todo~${i,j \in [n]}$,~$A[i][j]$ é indefinido se e somente se~${ i + c \geq j }$, a matriz~$A$ é triangular superior em~$c$.
\end{defi}

Adaptaremos os conceitos e resultados apresentados nesta seção para matrizes triangulares. Por simplicidade, vamos realizar estas adaptações em termos de matrizes triangulares inferiores no caso da convexidade. É fácil recriar as mesmas adaptações para superiores e para o caso da concavidade. Várias das linhas ou colunas de uma matriz triangular podem ser totalmente indefinidas. Isso faz com que os vetores de índices ótimos da matriz sejam mal definidos.  Vamos considerar que os vetores de índices ótimos destas matrizes estão definidos apenas quando existe pelo menos um valor definido para o qual se otimizar. Se uma matriz triangular inferior não possui nenhum elemento definido nas linhas 1 e 2, por exemplo, consideramos que seu vetor de índices de mínimos das linhas está definido apenas para as posições em~$[3 \tdots n]$ onde~$n$ é a última linha da matriz.

Para que uma matriz~$A \in \B{Q}^{n \times n}$ triangular inferior em~$c$ seja totalmente monótona convexa nas linhas, é necessário que, para todo~${ i,i',j,j' \in [n] }$ onde~${ i + c \leq i' + c \leq j \leq j'}$, valha que~${ A[i'][j] \leq A[i'][j'] }$ implica em~${ A[i][j] \leq A[i][j'] }$. Isso quer dizer que toda submatriz de~$A$ que tem todas as entradas definidas é totalmente monótona convexas nas linhas. A definição análoga serve nos outros casos de total monotonicidade. Desta forma, o Lema~\ref{lema:MonotoneTotallyMonotone} 
vale para toda submatriz que tem todas as entradas definidas, também. É importante notar que os vetores de índices ótimos de matrizes totalmente monótonas triangulares \textbf{não} são necessariamente monótonos.

De maneira similar, dizemos que uma~$A$ triangular inferior em~$c$ de lado~$n$ é Monge convexa se a desigualdade quadrangular vale para todo~${ i,i',j,j' \in [n]}$ onde~${ i + c \leq i' + c \leq j \leq j' }$. O Lema~\ref{Monge:MCtoTM} também vale para toda submatriz inteiramente definida de~$A$. O Teorema~\ref{Monge:theo+1} neste contexto, no caso convexo, diz que, se~${ A[i][j] + A[i+1][j+1] \leq A[i][j+1] + A[i+1][j] }$ sempre que~${ i,j \in [n-1] }$ e~${ i + c + 1 \leq j }$, então~$A$ é Monge convexa. O caso côncavo vale analogamente.

É útil perceber que, se~$B \in \B{Q}^n$ é uma matriz Monge qualquer, uma matriz~$A \in \B{Q}^n$ triangular tal que~$A[i][j] = B[i][j]$ para toda entrada~$A[i][j]$ definida de~$A$ também é Monge no mesmo sentido. Esta observação prova, imediatamente, o Lema~\ref{Monge:keepConvex} e o Teorema~\ref{theo:MongeConvex_and_Convex} para matrizes triangulares.
