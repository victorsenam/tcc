\chapter{Estruturas de dados em programação dinâmica}
\label{EDPD}

Em certas ocasiões, ao aplicar programação dinâmica no cálculo de alguma recorrência, podemos manter algumas das entradas já calculadas em uma estrutura de dados que pode nos ajudar a calcular com mais eficiência as entradas seguintes. Nesta seção, exemplificaremos esta possibilidade. 

Na Seção~\ref{Online} apresentaremos um exemplo no qual a convexidade (ou concavidade) da recorrência calculada nos permite usar uma estrutura de dados de forma parecida com a apresentada aqui. Esta seção, portanto, é uma introdução à utilização de estruturas de dados em programação dinâmica que deve facilitar a compreensão da Seção~\ref{Online}.

%----------------------------------------------------------------------------------------

\section{O problema da subsequência crescente de peso máximo}

\begin{prob}[Subsequência crescente de peso máximo] \label{EDPD:HIS}
Dados um natural~$n$ e dois vetores~${ v,w \in \B{Q}^n }$, descobrir uma subsequência~$s$ de~$[1 \tdots n]$ crescente em~$v$ e de soma máxima em~$w$. Formalmente, a subsequência~$s$ deve respeitar~${ v_{s_i} < v_{s_j} }$ para todo par de índices~${ i,j \in [1 \tdots |s|] }$ tal que~${ i < j }$ e deve maximizar~${ \sum\limits_{i \in s} w_i }$.
\end{prob}

Podemos resolver este problema com tempo~$\Cl{O}(n^2)$ utilizando programação dinâmica na recorrência
\begin{equation} \label{EDPD:HIS:rec}
f(i) = \max(\{f(j) \mid j < i \text{ e } v_j < v_i\} \cup \{0\}) + w_j \text{,}
\end{equation}
onde, para todo~$i \in [1 \tdots n]$, o valor~$f(i)$ é o peso da melhor sequência válida que contém~$i$ como último elemento.

Um caso especial deste problema onde o vetor~$w$ é um vetor unitário é o problema clássico da maior subsequência crescente. É conhecido o fato de que este problema pode ser resolvido em tempo~$O(n\lg(n))$~\cite{Fredman:1975}. 

%----------------------------------------------------------------------------------------

\section{Fronteira de Pareto}

Apresentaremos uma estrutura de dados que vai nos ajudar a resolver o Problema~\ref{EDPD:HIS} em tempo~$\Cl{O}(n\lg(n))$, para isso, vamos utilizar algumas definições e resultados.

\begin{defi}[Relação de dominância]
Sejam~${ a = (a_1,a_2) }$ e~${ b = (b_1,b_2) }$ elementos de~$\B{Q}^2$. Se~${ a_1 \geq b_1 }$ e~${ a_2 \geq b_2 }$ escrevemos~${ a \succeq b }$ ($a$ domina~$b$). Além disso, se~${ S \subseteq \B{Q}^2 }$, um ponto~${ a \in \B{Q} }$ é dominado em~$S$ se existe um outro ponto~$b \in S$ que domina~$a$.
\end{defi}

\begin{prop}
A relação de dominância é uma ordem parcial~\cite[Apêndice B]{CLRS} em~$\B{Q}^2$.
\end{prop}
\begin{proof}
Se~$a = (a_1,a_2) \in \B{Q}^2$, vale~$a_1 \geq a_1$ e~$a_2 \geq a_2$, portanto~$a \succeq a$ e~$\succeq$ é reflexiva. Se~$a = (a_1,a_2)$ e~$b = (b_1,b_2) \in \B{Q}^2$ são tais que~$a \succeq b$ e~$b \succeq a$, então~$a_1 = b_1$ e~$a_2 = b_2$, portanto~$a = b$ e~$\succeq$ é antissimétrica. Se~$a = (a_1,a_2), b = (b_1,b_2)$ e~$c = (c_1,c_2) \in \B{Q}^2$ são tais que~$a \succeq b$ e~$b \succeq c$, então~$a_1 \geq c_1$ e~$a_2 \geq c_2$, logo~$a \succeq c$ e~$\succeq$ é transitiva.
\end{proof}

Como dito, desenvolveremos uma estrutura de dados que irá agilizar o cálculo de uma recorrência. Tal estrutura será capaz de calcular, para todo~${ c \in \B{Q} }$, o máximo atingido pela segunda coordenada de algum elemento que tem valor pelo menos~$c$ na primeira coordenada e já foi adicionado à estrutura. A função~$M(S,c)$ da Definição~\ref{EDPD:M} abaixo ajuda a descrever este valor, já que, se~$S$ é o conjunto dos elementos adicionados na estrutura,~$M(S,c)$ define exatamente o valor calculado pela estrutura.
\begin{defi} \label{EDPD:M}
Dados~$S \subset \B{Q}^2$ finito e~$c \in \B{Q}$, definimos~${M(S,c) = \max\{a_2 \mid (a_1,a_2) \in S \text{ e } a_1 \geq c\}}$.
\end{defi}

Com estas definições estamos prontos para apresentar a estrutura de dados que vai agilizar a solução do Problema~\ref{EDPD:HIS}.
\begin{defi}[Fronteira de Pareto] \label{EDPD:Pareto}
A fronteira de Pareto~$\Cl{P}(S)$ de um conjunto finito~$S \subset \B{Q}^2$ é uma estrutura de dados que contém todos os elementos não dominados de~$S$, isto é, $x \in \Cl{P}(S)$ se e somente se $x \in S$ e $x$ não é dominado em $S$. Podemos realizar as seguintes operações sobre~$\Cl{P}(S)$:

\begin{description}
\item[$\textsc{Insere}(\Cl{P}(S),x)$] Inserir um novo elemento~$x \in \B{Q}^2$ em~$S$ em tempo~$\Cl{O}(\lg(|S|))$ amortizado.
\item[$\textsc{CalculaM}(\Cl{P}(S),c)$] Descobrir, dado um~$c \in \B{Q}$, o valor~$M(S,c)$ em tempo~$\Cl{O}(\lg(|S|))$.
\end{description}
\end{defi}

Para implementar esta estrutura de dados, vamos manter os elementos de~$\Cl{P}(S)$ em uma árvore de busca binária balanceada ordenada lexicograficamente, isto é, crescentemente pela primeira coordenada e, em caso de empate, crescentemente pela segunda. Se dois elementos~${ a,b \in \B{Q}^2 }$ são tais que~$b$ é maior lexicograficamente do que~$a$, denotamos~${ b \geq a }$. 

Definimos as operações~$\textsc{Sucessor}(\Cl{P}(S),a)$ e~$\textsc{Predecessor}(\Cl{P}(S),a)$ para cada elemento~${ a \in \B{Q}^2 }$. A primeira retorna o menor elemento lexicograficamente maior ou igual a~$a$ e a segunda, similarmente, retorna o maior elemento lexicograficamente menor ou igual a~$a$. Em ambas as operações, se o elemento desejado não existe, o valor~$\textsc{NULL}$ é retornado. Já que a árvore de busca binária é balanceada e está ordenada lexicograficamente, é possível realizar estas operações em tempo~$\Cl{O}(\lg(|\Cl{P}(S)|))$.  Permitimos a realização de chamadas onde a segunda coordenada de~$a$ é~$-\infty$ ou~$\infty$. É possível implementar este tipo de operação como um caso especial da primeira, porém, também é possível definir, para cada aplicação, um valor suficientemente pequeno para~$-\infty$ e um outro suficientemente grande para~$\infty$ que torne esta chamada conveniente.

Guardar os elementos da fronteira nesta ordem mantém os elementos ordenados crescentemente pela primeira coordenada, o que garante uma propriedade interessante, descrita pela Proposição~\ref{EDPD:doubleorder} a seguir. 

\begin{prop} \label{EDPD:doubleorder}
Se~$S \subset \B{Q}^2$ é finito e os elementos de~$\Cl{P}(S)$ estão em ordem crescente na primeira coordenada, então eles estão em ordem decrescente na segunda coordenada.
\end{prop}
\begin{proof}
Suponha que existam~$x = (x_1,x_2)$ e~$y = (y_1,y_2) \in \Cl{P}(S)$ tais que~$x_1 \leq x_2$ e~$y_1 \leq y_2$. Assim,~$y \succeq x$ e~$x \notin \Cl{P}(S)$.
\end{proof}

A Proposição~\ref{EDPD:doubleorder} mostra que~${ M(\Cl{P}(S),c) = \textsc{Sucessor}(\Cl{P}(S),(c,-\infty) }$ para todo~$c \in \B{Q}$.

\begin{prop} \label{EDPD:DomP}
Se~$S$ é um conjunto finito e um ponto~$x \in \B{Q}^2$ é dominado em~$S$, então,~$x$ é dominado em~$\Cl{P}(S)$.
\end{prop}
\begin{proof}
Sejam~$S$ e~$x$ dados como no enunciado. Se~$x$ é dominado em~$S$ existe pelo menos um elemento em~$S$ que domina~$x$, dentre estes, tome por~$y$ algum que é maximal em relação a~$\succeq$, isto é, que não seja dominado em~$S$. Isto é possível pois~$\succeq$ é uma ordem parcial e~$S$ é finito. Vale que~${ y \in \Cl{P}(S) }$, portanto,~$x$ é dominado em~$\Cl{P}(S)$.
\end{proof}

A Proposição~\ref{EDPD:keepM} abaixo garante que~${ M(\Cl{P}(S),c) = M(S,c) }$ para todo~$c \in \B{Q}$. Já que~${\lg(|\Cl{P}(S)|) = \Cl{O}(\lg(|S|))}$ e calcular~$M(\Cl{P}(S),c)$ a partir de uma chamada a~$\textsc{Sucessor}$  custa tempo~$\Cl{O}(\lg(|\Cl{P}(S)|))$, o cálculo da operação~$\textsc{CalculaM}$ é realizado em tempo~$\Cl{ O(\lg(|S|)) }$.

\begin{prop} \label{EDPD:keepM}
Para todo~$S \subset \B{Q}^2$ finito e~$c \in \B{Q}$ vale~$M(S,c) = M(\Cl{P}(S),c)$.
\end{prop}
\begin{proof}
Já que~${ \Cl{P}(S) \subseteq S }$ segue que~${ M(S,c) \geq M(\Cl{P}(S) }$. Vamos provar o outro sentido da igualdade. Suponha que~${ M(S,c) > M(\Cl{P}(S),c) }$. Existe, portanto, algum elemento~$a = (a_1,a_2) \in S$ tal que~${ a \notin \Cl{P}(S) }$,~${ a_2 \geq c }$ e~${ a_1 > M(\Cl{P}(S),c) }$. Porém, já que~${ a \notin \Cl{P}(S) }$, vale que~$a$ é dominado em~$S$ e, pela Proposição~\ref{EDPD:DomP}, dominado em~$\Cl{P}(S)$, logo, existe~${ b \in \Cl{P}(S) }$ tal que~${ b \succeq a }$. Concluímos que~${ b_2 \geq a_2 \geq c }$ e~${ M(\Cl{P}(S),c) \geq b_1 \geq a_1 = M(S,c) }$, um absurdo.
\end{proof}

Para um~${ a \in \B{Q}^2 }$ qualquer, podemos realizar a operação~$\textsc{Insere}(\Cl{P}(S),a)$ eficientemente. Primeiramente, devemos verificar se~$a$ é dominado em~$\Cl{P}(S)$. Para tanto, basta verificar se~$\textsc{Sucessor}(\Cl{P}(S),a)$ domina~$a$. Se esta chamada retornar~$\textsc{NULL}$, todo elemento com primeira coordenada maior ou igual à primeira coordenada de~$a$, tem a segunda coordenada menor do que a de~$a$, portanto, o elemento~$a$ não é dominado. Se a chamada retornar um elemento~$b$ que não domina~$a$, este deve ter a segunda coordenada menor do que a segunda coordenada de~$a$. Com isso, sabemos que todo elemento lexicograficamente maior do que~$b$ não domina~$a$, graças à Proposição~\ref{EDPD:doubleorder} e, portanto, não existe outro elemento em~$\Cl{P}(S)$ que domina~$a$.

Além de descobrir se~$a$ é dominado, precisamos remover os elementos dominados por~$a$ de~$\Cl{P}(S)$. Para tanto, perceba que se algum elemento de~$\Cl{P}(S)$ for dominado por~$a$, o elemento~$\textsc{Predecessor}(\Cl{P}(S),a)$ será, também, dominado por~$a$. Graças a este fato, podemos buscar este elemento e removê-lo, caso seja dominado, sucessivamente, até que~$\textsc{Predecessor}(\Cl{P}(S),a)$ seja~$\textsc{NULL}$ ou não seja dominado por~$a$.

Note que todo elemento pode ser removido da estrutura apenas uma vez. Concluímos, levando em conta os custos das chamadas a~$\textsc{Predecessor}$ e~$\textsc{Sucessor}$, que a operação~$\textsc{Insere}$ é realizada em tempo~$\Cl{O}(\lg(|S|))$ amortizado. O Algoritmo~\ref{EDPD:FP:code} implementa as operações descritas aqui. 

\begin{algorithm}[h]
\caption{Operações sobre a fronteira de Pareto de um conjunto}
\label{EDPD:FP:code}
\begin{algorithmic}[1]
\Function{\textsc{Insere}}{\Cl{P}(S),x = (x_1,x_2)}
    \State $y \rec \textsc{Sucessor}(\Cl{P}(S),x)$
    \If{$y \neq \textsc{NULL}$ e $y \succeq x$}
        \State Interrompe a execução da função
    \EndIf

    \State $y \rec \textsc{Predecessor}(\Cl{P}(S),x)$
    \While{ $y \neq \textsc{NULL}$ e $x \succeq y$}
        \State Remove~$y$ de~$\Cl{P}(S)$
        \State $y \rec \textsc{Predecessor}(\Cl{P}(S),x)$
    \EndWhile
    \State Insere~$x$ na árvore de~$\Cl{P}(S)$
\EndFunction
\Function{\textsc{CalculaM}}{\Cl{P}(S),c}
    \State $y = (y_1,y_2) \rec \textsc{Sucessor}(\Cl{P}(S),(c,-\infty))$
    \If{$y \neq \textsc{NULL}$}
        \State \Return $y_2$
    \Else
        \State \Return $-\infty$
    \EndIf
\EndFunction
\end{algorithmic}
\end{algorithm}

Uma implementação de fronteira de Pareto como descrita pelo algoritmo acima em~\texttt{C++} pode ser encontrada na pasta de implementações com o nome~\texttt{Pareto.cpp}.


%----------------------------------------------------------------------------------------

\section{Otimização para a subsequência crescente de peso máximo}

Com a possibilidade de manter a fronteira de Pareto de um subconjunto finito de~$\B{Q}^2$, vamos calcular cada~$f(i)$ definido na solução original do Problema~\ref{EDPD:HIS} em tempo~$\Cl{O}(\lg(n))$ onde~$n$ é o tamanho dos vetores~$v$ e~$w$ dados pelo problema. Para isso basta notar que podemos reescrever a recorrência~\eqref{EDPD:HIS:rec} como~$f(i) = M(S_i,-v_i)$ onde~$S_i = \{(-v_j,f(j)) \mid j < i\} \cup \{(0,\infty)\}$. Aproveitando a estrutura da fronteira de Pareto, escrevemos o Algoritmo~\ref{EDPD:HIS:code} que calcula todos os valores de~$f$ em tempo~$\Cl{O}(n\lg(n))$ e retorna o máximo entre 0 e todos estes valores calculados, já que esta é a resposta para o problema proposto.

\begin{algorithm}[h]
\caption{Solução do Problema~\ref{EDPD:HIS}}
\label{EDPD:HIS:code}
\begin{algorithmic}[1]
\Function{\textsc{SubsequênciaCrescentePesoMáximo}}{n,v,w}
    \State Inicializa a fronteira de Pareto~$P$ com o elemento~$(0,\infty)$
    \State $r \rec 0$
    \For{$i$ de $1$ até $n$}
        \State $f(i) = \textsc{CalculaM}(P,-v_i)$
        \State $\textsc{Insere}(P,(-v_i,f(i)))$
        \State $r \rec \max(r,f(i))$
    \EndFor
    \State \Return $r$
\EndFunction
\end{algorithmic}
\end{algorithm}
