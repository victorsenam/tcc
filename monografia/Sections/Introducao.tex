\section{Introdução}
\label{Introducao}

\subsection{Sobre o trabalho}

\subsection{Notação}
Se~$i$ e~$j$ são dois inteiros, a expressão~$[i \tdots j]$ denota o conjunto~${ \{ k \in \B{Z} \mid i \leq k \leq j \} }$, além disso, se~$n$ é inteiro~$[n]$ denota~$[1 \tdots n]$. Note que se~$i > j$, o conjunto~$[i \tdots j]$ é o conjunto vazio. Se~$v$ é um vetor, podemos escrever~$v[i \tdots j]$ para denotar o vetor~${ (v_i, v_{i+1}, \dots, v_j) }$. O mesmo vale em termos de submatrizes. Se~$r$ e~$\ell$ também são dois inteiros e~$A$ é uma matriz, podemos escrever~$A[i \tdots j][\ell \tdots r]$ para denotar a submatriz de~$A$ que contém apenas as linhas de~$A$ em~$[i \tdots j]$ e as colunas em~$[\ell \tdots r]$. Por padrão, nestes dois últimos casos, os subvetores e as submatrizes geradas são reindexados para os conjuntos~$[j - i + 1]$ e~${ [j - i + 1] \times [r - \ell + 1] }$, respectivamente.

Se~$A$ e~$B$ são conjuntos, podemos escrever~$A^B$ para denotar o conjunto de vetores com entradas em~$A$ indexados pelos elementos de~$B$. Ainda se~$C$ também é um conjunto,~$A^{B \times C}$ denota o conjunto das matrizes com linhas indexadas por elementos de~$B$ e colunas indexadas por elementos de~$C$. Por exemplo, se~$v \in \B{R}^{ [3 \tdots 5] }$ ele é um vetor com valores~${ v_3, v_4 }$ e~$v_5$. Definimos, ainda, para cada~$n$ e~$m$ inteiros, os conjuntos~${ \B{R}^n = \B{R}^{ [n] } }$ e~${ \B{R}^{n \times m} = \B{R}^{[n] \times [m]} }$.

\subsection{Matrizes}
Explicar o que são matrizes online e offline.

\subsection{Implementações} \label{Intro:impl}
Explicar os padrôes que estou usando pra implementar os programas. Por exemplo: Funções como argumentos, 0-index (em contraste com o 1-index do pseudo-código) e wrappers.
