\section{Busca em matrizes online}
\label{Online}

Nesta seção começamos a estudar formas de encontrar índices ótimos em matrizes onde certas entradas são desconhecidas à priori. Inicialmente o formato das matrizes abordadas será explicitado juntamente com um exemplo de um caso específico interessante desta técnica apresentado por Galil e Park~\cite{Galil:1992}. Na Seção~\ref{FullyDynamic} buscaremos generalizar a técnica apresentada aqui para outras matrizes online.

Para conseguir lidar com matrizes que não são inteiramente conhecidas à priori, devemos explicar quais dependências podem aparecer e quais não podem, para isso, é útil descrever matrizes online segundo a Definição~\ref{Online:matrix}.

\begin{defi}[Matriz online] \label{Online:matrix}
Dizemos que uma matriz~$A \in \B{Q}^{n \times n}$ com vetor de índices de mínimos de linhas~$R$ é online nas linhas se para todo~$i,j \in [n]$ com~$i \neq j$ o valor de~$A[i][i]$ é conhecido à priori e os valores de~$A[i][j]$ podem depender de~$R[j]$. Isto é, em uma matriz online, a diagonal é conhecida e as outras entradas de uma \textbf{coluna}~$j$ qualquer dependem do índice do mínimo da \textbf{linha}~$j$ correspondente.

Similarmente, dizemos que uma matriz é online nas colunas quando as entradas de suas diagonais são conhecidas à priori porém uma entrada~$A[i][j]$ fora da diagonal pode depender do índice de mínimo da coluna~$i$.
\end{defi}

Note que a definição permite que outras entradas além das diagonais sejam conhecidas à priori. Dizemos ainda que uma entrada da matriz está disponível se é possível descobrir seu valor, isto é, ela era conhecida à priori ou ela só depende de informações já calculadas.

Nesta seção exploraremos especificamente o caso onde a matriz é triangular inferior, isto é, cada entrada~$A[i][j]$ da matriz onde~$i \geq j$ é indefinida. Vamos descobrir métodos para encontrar índices ótimos nestas matrizes quando elas forem totalmente monótonas convexa ou côncavas. O Problema~\ref{Online:prob} ilustra a utilidade de trabalhar com matrizes deste tipo.

\begin{prob} \label{Online:prob}
Dada uma matriz de custos~$C \in \B{Q}^{n \times n}$, computar o vetor~$E$ tal que~$E[1] = 0$ e para todo~$i \in [1 \tdots n]$ vale~$E[i] = \min\limits_{j < i} C[i][j] + E[j]$.
\end{prob}

Com uma matriz~$C$ e um vetor~$E$ definidos como no Problema~\ref{Online:prob}, criamos a matriz~$B \in \B{Q}^{n \times n}$ de forma que para todo~$i,j \in [n]$ vale
\begin{equation} \label{Online:Bmat}
    B[i][j] = \begin{cases}
        \text{indefinida} & \text{, se } i \leq j \text{ e } \\
        C[i][j] + E[j]    & \text{, caso contrário.}
    \end{cases}
\end{equation}

Primeiramente, é fácil observar que a matriz~$B$ é online nas linhas e triangular inferior, além disso, se~$R$ é o vetor de índices de mínimos das linhas de~$B$, vale~$E[i] = B[i][R[i]]$ para toda linha~$i \in [2 \tdots n]$ de~$B$, portanto, resolver o Problema~\ref{Online:prob} é encontrar os índices de mínimos das linhas de~$B$.É interessante notar que a linha 1 de uma matriz triangular inferior não possui mínimo, já que todos os seus valores são indefinidos.

Como já dito, vamos discutir maneiras eficientes encontrar os índices ótimos de uma matriz online nas linhas e triangular inferior no caso onde essa matriz é totalmente monótona nas linhas. No Problema~\ref{Online:prob} é fácil aplicar o Lema~\ref{Monge:theo+1} para perceber que se a matriz~$C$ é Monge, a matriz~$B$ é Monge no mesmo sentido e, portanto, totalmente monótona. Assim, as técnicas discutidas nesta seção são suficientes para resolver o Problema~\ref{Online:prob}.

Passamos discutir agora o problema de encontrar os mínimos das linhas de uma matriz~$A \in \B{Q}^{n \times n}$ totalmente monótona nas linhas, online nas linhas e triangular inferior. É fácil adaptar as técnicas e formulações apresentadas para problemas de maximização e para o caso onde estamos interessados nos pontos ótimos das colunas de uma matriz totalmente monótona nas colunas e online nas colunas.

Para resolver este problema, usaremos uma estrutura de dados que chamamos de envelope.

\begin{defi}
Um envelope~$\Cl{E}$ sobre uma matriz~$A$ é uma estrutura de dados que mantém um conjunto de colunas~$J$ da matriz e é capaz de responder às seguintes operações:

\begin{description}
    \item[\textsc{Insere}($\Cl{E}$,$j$)] Insere uma coluna~$j$ disponível da matriz na estrutura de dados e
    \item[\textsc{Calcula}($\Cl{E}$,$i$)] Retorna~$\min\{j \in J \mid A[i][j] \leq A[i][j'] \text{ para todo } j' \in J\}$ onde~$i \in [n]$ e~$J$ é não vazio.
\end{description}
\end{defi}

Com esta definição, vamos descrever genericamente o Algoritmo~\ref{Online:algo} que retorna o vetor~$E$ de mínimos das linhas da matriz~$A$.

\begin{algorithm}[h]
\caption{Mínimos de linhas online}
\label{Online:algo}
\begin{algorithmic}[1]
\Function{Online}{A, n}
    \State $\Cl{E}$ é um envelope sobre a matriz~$B$
    \State $\textsc{Insere}(\Cl{E},1)$

    \For{$i$ de $2$ até $n$} \label{Online:algo:loop}
        \State $R[i] = \textsc{Calcula}(\Cl{E},i)$
        \State $\textsc{Insere}(\Cl{E},i)$
    \EndFor

    \State \Return $R$
\EndFunction
\end{algorithmic}
\end{algorithm}

O algoritmo acima funciona pois em toda iteração do laço da linha~\ref{Online:algo:loop}, a variável~$R[i]$ recebe o valor~$\min\{ j < i \mid B[i][j] \leq B[i][j'] \text{ para todo } j' < i\}$, que é o índice de mínimo da linha~$i$, portanto, o algoritmo retorna o vetor de índices de mínimos de~$A$.

Falta agora descrever a estrutura de envelope tanto para o caso côncavo quanto para o caso convexo. Nos dois casos temos a garantia de que o parâmetros passados tanto para \textsc{Insere} quanto para \textsc{Calcula} são dados em ordem crescente, isto é, as colunas são inseridas em ordem crescente de índice na estrutura e os mínimos das linhas também são calculados em ordem crescente. Inicialmente, nas Subseções~\ref{Online:convex} e~\ref{Online:concave} aproveitaremos apenas a propriedade descrita para a operação \textsc{Insere} e depois, na Subseção~\ref{Online:linear} utilizaremos também a propriedade descrita para \textsc{Calcula}.

%------------------------------------------------------------------

\subsection{Caso convexo} \label{Online:convex}
Discutiremos agora a implementação de um envelope~$\Cl{E}$ sobre uma matriz~$A$ totalmente monótona convexa nas linhas e online nas linhas. 

\begin{defi}[Coluna válida]
Uma coluna~$j \in J$ é válida somente se existe uma linha~$i \in [n]$ para a qual~${j = \textsc{Calcula}(\Cl{E},i)}$, ou seja, ela é o mínimo de alguma linha. Uma coluna que não respeita esta propriedade é chamada inválida.
\end{defi}

Se~$a$ e~$b$ são duas colunas válidas de~$A$ onde~$a < b$, pela convexidade de~$A$ e pelo Lema~\ref{lema:MonotoneTotallyMonotone}, sabemos que a coluna~$a$ tem valor menor ou igual do que o da~$b$ para um prefixo das linhas e maior para o restante (um sufixo), isto é, existe um índice~$i$ tal que a partir da linha~$i$ a coluna~$b$ é a melhor entre as duas e antes da linha~$i$ a coluna~$a$ é a melhor. Definimos a operação~$\textsc{Intersecta}(a,b)$ que devolve este índice de linha. Formalmente, definimos~${\textsc{Intersecta}(a,b) = \min\{ i \in [n] \mid A[i][a] > A[i][b] \}}$. O cálculo desta operação pode ser realizado com busca binária em tempo~$\Cl{O}(\lg(n))$ da maneira descrita no Algoritmo~\ref{Online:convex:BB}.

\begin{algorithm}[h]
\caption{Intersecção de duas colunas no caso convexo}
\label{Online:convex:BB}
\begin{algorithmic}[1]
\Function{Intersecta}{a,b}
    \State $\ell \rec 1$
    \State $r \rec n$

    \While{ $\ell < r$}
        \State $p \rec (\ell + r)/2$
        \If{ $A[p][a] > A[p][b]$} \label{Online:convex:BB:ineq}
            \State $r \rec p$
        \Else
            \State $\ell \rec p+1$
        \EndIf
    \EndWhile
    \State \Return $\ell$
\EndFunction
\end{algorithmic}
\end{algorithm}

Para cada coluna~$j \in J$ é interessante descobrir a primeira linha onde esta é a de menor valor dentre todas as colunas de índice menor do que ela em~$J$. Vamos definir este valor como~$s(j)$ e, assim como em~\textsc{Intersecta}, se não houver uma linha que respeite isso definimos~$s(j) = n+1$. Formalmente, definimos~${s(j) = \max(\{ \textsc{Intersecta}(j',j) \mid j' \in J \text{ e } j' < j\} \cup \{1\})}$.

\begin{prop} \label{Online:convex:easys}
Para qualquer conjunto~$J$ de colunas de~$A$ e qualquer~$j \in J$ ou conjunto~${\{j' \in J \mid j' \text{ é valido e } j' < j\}}$ é vazio e~$s(j) = 1$ ou~$\{s(j) = \textsc{Intersecta}(j,j')\}$ onde~$j'$ é o máximo deste conjunto.

Isto é, para calcular o valor de~$s(j)$ basta escolher o maior~$j' \in J$ válido de valor menor do que~$j$. Se tal valor existe basta calcular sua intersecção com~$j$ e, caso contrário, definir~$s(j) = 1$.
\end{prop}

\begin{proof}
Denotamos por~$J_<$ o subconjunto de~$J$ que contém apenas os elementos menores do que~$j$. Se~$| J_<| \leq 1$ é fácil ver que as definições propostas para~$s(j)$ coincidem analisando 3 casos separadamente, onde~$J_<$ é vazio, onde possui um elemento válido e onde possui um elemento inválido e assumimos, sem perda de generalidade, que~$|J_<| > 1$.

Sejam~$j_0$ e~$j_1$ dois elementos de~$J_<$ tais que~$j_0 < j_1$. Definimos~${i_0 = \textsc{Intersecta}(j_0,j)}$ e~${i_1 = \textsc{Intersecta}(j_1,j)}$. Suponha que~$i_1 < i_0$. Vamos descobrir para cada linha da matriz qual dentre as três colunas,~$j_0$,~$j_1$ e~$j$, é a de valor mínimo. Para todo~$i \in [i_0 \tdots n]$, pela total monotonicidade de~$A$ e já que~$i_0 < i$ e~$i_1 < i$, valem as desigualdades~$A[i][j] < A[i][j_0]$ e~$A[i][j] < A[i][j_1]$. 

Além disso, sabemos~$A[i_0-1][j_0] \leq A[i_0-1][j]$ e~$A[i_0-1][j] < A[i_0-1][j_1]$ o que nos leva a~$A[i_0-1][j_0] < A[i_0-1][j_1]$ e, portanto, para todo~$i' \in [1 \tdots i_0-1]$ vale~$A[i][j_0] < A[i][j_1]$. Note que usamos a total monotonicidade de~$A$ em todo passo desta demonstração. Isso nos mostra que no intervalo~$[1 \tdots i_0-1]$~$j_0$ é ótimo e no intervalo~$[i_0 \tdots i_1]$~$j$ é ótimo, portanto,~$j_1$ é inválido.

Com isso, sabemos que o maior~$j'$ válido em~$J_<$ é o de maior intersecção com~$j$ dentre todos os elementos de~$J_<$, provando a proposição.
\end{proof}

A Proposição~\ref{Online:convex:easys} dá motivo a uma representação do envelope~$\Cl{E}$ que guarda todos os elementos válidos de~$J$ numa lista ordenada crescentemente. Nesta representação, calcular o~$s(j)$ de um dado~$j$ é fácil. Se~$j$ for o primeiro elemento,~$s(j) = 1$, caso contrário, basta visitar o elemento que precede~$j$ e calcular sua intersecção com~$j$. Vamos utilizar esta representação de~$\Cl{E}$. Além disso, guardaremos para cada elemento~$j$ da lista qual é o seu valor~$s(j)$. A nossa estrutura deve, então, manter 3 invariantes. Um valor~$j \in J$ pertence à estrutura se e somente se é válido, os valores aparecem em ordem crescente e todos os~$s(j)$ estão calculados corretamente.

\begin{algorithm}[h]
\caption{Envelope convexo}
\label{Online:envelope:convex}
\begin{algorithmic}[1]
\Function{Calcula}{\Cl{E},i}
    \State $\ell \rec 1$
    \State $r \rec |\Cl{E}|$
    
    \While{ $\ell < r$ }
        \State $p = (\ell+r)/2$
        \If{ $s(\Cl{E}_p) \leq i$ }
            \State $r = p$
        \Else
            \State $\ell = p+1$
        \EndIf
    \EndWhile

    \State \Return $\ell$
\EndFunction

\Function{Insere}{\Cl{E},j}
    \If{ $A[n][\Cl{E}_{-1}] \leq A[n][j]$ }
        \State \Return
    \EndIf
    
    \While{ $\Cl{E}$ é não vazio e~$A[s(\Cl{E}_{-1})][\Cl{E}_{-1}] > A[\Cl{E}_{-1}][j]$ }
        \State Remove o final de~$\Cl{E}$
    \EndWhile

    \State Insere~$j$ ao final de~$\Cl{E}$
\EndFunction
\end{algorithmic}
\end{algorithm}

O Algoritmo~\ref{Online:envelope:convex} implementa as operações~\textsc{Calcula} e~\textsc{Insere} como descrito a seguir. Para responder a uma pergunta da forma~$\textsc{Insere}(\Cl{E},j)$ vamos aproveitar a garantia já citada de que~$j > j'$ para todo~$j' \in J$ já que as operações de inserção estão ordenadas crescentemente. É possível que~$j$ seja uma coluna inválida em~$J$, esta condição pode ser testada comparando os valores das coluna~$j$ com o valor da última coluna~$j_{-1}$ da estrutura, se~${A[n][j_{-1}] \leq A[n][j]}$, pela total monotonicidade,~$j$ será inválida, caso contrário,~$j$ será a coluna ótima da estrutura na linha~$n$. No caso onde~$j$ é inválida, devemos ignorar esta operação de inserção e podemos, agora, assumir sem perda de generalidade que~$j$ é válida.

Devemos verificar se a nossa coluna~$j$ torna alguma outra coluna de~$J$ inválida e remover esta coluna. Escolhemos a última coluna~$j_{-1}$ da estrutura e verificamos~${A[s(j_{-1})][j_{-1}] \leq A[s(j_{-1})][j]}$, se esta condição for verdade~$j_{-1}$ é válida pois é ótima no ponto~$s(j_{-1})$ e todas as colunas anteriores também são, já que, pela total monotonicidade, o valor da coluna~$j_{-1}$ supera o valor de~$j$ em todas as linhas anteriores a~$s(j_{-1})$, mantendo a validade de todos os elementos que possuem algum ponto ótimo em alguma dessas linhas. Se a condição for mentira, pela total monotonicidade,~$j_{-1}$ é superada por~$j$ de~$s(j_{-1})$ em diante e por alguma outra coluna antes desta linha, portanto, é inválida e deve ser removida, portanto, removemos~$j_{-1}$ e repetimos o processo até não haver mais elementos na lista ou até o último elemento ser válido. Com este trabalho, mantivemos a invariante de que todas as colunas de~$\Cl{E}$ são válidas.

Falta atualizar os valores~$s(j')$ para todo~$j' \in \Cl{E}$. Temos a garantia de que inserimos e removemos elementos apenas no final da lista, já que os valores~$s(j')$ dependem do elemento anterior, basta calcular o valor~$s(j)$ do novo elemento adicionado, o que sabemos fazer em tempo~$\Cl{O}(\lg(n))$. Isso garante a invariante de que os valores da função~$s$ estão guardados corretamente. A invariante de que o vetor é ordenado é trivialmente mantida já que só adicionamos elementos no final da lista.

Agora precisamos usar esta estrutura para responder rapidamente a uma pergunta da forma~${ \textsc{Calcula}(\Cl{E},i) }$ com~$i \in [n]$. Para isso, vamos descobrir o maior elemento~$j \in \Cl{E}$ tal que~$s(j) \leq i$. Este valor é a resposta correta pois a resposta deve ser válida, portanto, pertencer a~$\Cl{E}$ e~$j$ supera todos os elementos anteriores a ele a partir de~$s(j)$ e todo elemento~$j'$ seguinte antes de~$s(j')$ que, por construção, é maior do que~$i$. A tarefa de encontrar este~$j$ pode ser realizada com uma busca binária na lista~$\Cl{E}$.

Assumimos que a nossa implementação da lista~$\Cl{E}$ é capaz de visitar um elemento em uma dada posição em tempo~$\Cl{O}(1)$ além de adicionar e remover elementos do final da lista, isso, em \texttt{C++}, pode ser implementado utilizando a estrutura~\texttt{vector} da biblioteca padrão. Podemos criar uma estrutura que guarda tanto o valor~$j$ quanto o~$s(j)$ correspondente e montar nossa lista fazendo com que cada elemento seja um objeto deste tipo, com isso, acessar ou atualizar o~$s(j)$ de um~$j$ é equivalente a acessar o valor de~$j$. Uma implementação em~\texttt{C++} que respeita estas hipóteses pode ser encontrada em~\texttt{envelope\_convex}.Vamos calcular a complexidade dos algoritmos sugeridos.

Cada operação~\textsc{Calcula} envolve apenas uma busca binária na lista~$\Cl{E}$ onde cada iteração custa tempo constante, já que esta lista tem tamanho no máximo~$n$ (pois~$A$ tem~$n$ colunas), estas chamadas custam~$\Cl{O}(\lg(n))$ cada. Uma operação~\textsc{Insere} pode remover vários elementos de~$\Cl{E}$, custando tempo potencialmente linear, porém, cada elemento de~$\Cl{E}$ só pode ser removido uma única vez, fazendo com que a soma de todas as remoções custe tempo~$\Cl{O}(n)$. Além disso, esta operação insere um novo elemento em~$\Cl{O}(1)$ e calcula o valor~$s$ do novo elemento em tempo~$\Cl{O}(\lg(n))$, desta maneira, cada operação custa tempo~$\Cl{O}(\lg(n))$ amortizado.

%------------------------------------------------------------------

\subsection{Caso côncavo} \label{Online:concave}

Discutiremos agora a implementação do envelope~$\Cl{E}$ sobre uma matriz~$A$ totalmente monótona côncava nas linhas e online nas linhas. Este caso se desenvolve de maneira extremamente similar ao caso convexo, portanto, aproveitaremos várias das definições do caso convexo.

Se~$a$ e~$b$ são colunas válidas de~$A$ onde~$a < b$, pela concavidade de~$A$ e pelo Lema~\ref{lema:MonotoneTotallyMonotone}, sabemos que a coluna~$a$ tem valor maior do que o da~$b$ para um prefixo das linhas e menor ou igual para o restante (um sufixo). Definimos a operação~$\textsc{Intersecta}(a,b)$ que retorna o primeiro índice de linha para o qual~$a$ tem valor menor ou igual a~$b$. Formalmente, vale~${\textsc{Intersecta}(a,b) = \min\{i \in [n] \mid A[i][a] \leq A[i][b]\}}$. Podemos calcular este valor, também, com busca binária, de forma parecida com o Algoritmo~\ref{Online:convex:BB} com a alteração de que a linha~\ref{Online:convex:BB:ineq} deve checar a condição~$A[p][a] \leq A[p][b]$.

%------------------------------------------------------------------

\subsection{Envelopes em tempo linear} \label{Online:linear}

%------------------------------------------------------------------

\subsection{Relação com Pareto}

O caso convexo funciona de maneira parecida como uma fronteira de Pareto em pares~$(j,s(j))$ que restringe a ordem das buscas e inserções e o caso côncavo pode ser interpretado, também, de maneira parecida. É razoável imaginar o motivo de não podermos realizar todas as operações de uma fronteira de Pareto no envelope. A próxima seção explora esta possibilidade.
