\section{Busca em matrizes online}
\label{Online}

Nesta seção começamos a estudar formas de encontrar índices ótimos em matrizes onde certas entradas são desconhecidas à priori. Inicialmente o formato das matrizes abordadas será explicitado juntamente com um exemplo de um caso específico interessante desta técnica apresentado por Galil e Park~\cite{Galil:1992}. Nas Subseções~\ref{Online:convex},~\ref{Online:concave} e~\ref{Online:linear} resolvemos os problemas para este formato e na seção seguinte utilizamos as ideias desenvolvidas aqui para resolver problemas mais genéricos relacionados a este.

Para conseguir lidar com matrizes que não são inteiramente conhecidas à priori, devemos explicar quais dependências podem aparecer e quais não podem, para isso, é útil descrever matrizes online segundo a Definição~\ref{Online:matrix}.

\begin{defi}[Matriz online] \label{Online:matrix}
Dizemos que uma matriz~$A \in \B{Q}^{n \times n}$ com vetor de índices de mínimos de linhas~$R$ é online nas linhas se para todo~$i,j \in [n]$ com~$i \neq j$ o valor de~$A[i][i]$ é conhecido à priori e os valores de~$A[i][j]$ podem depender de~$R[j]$. Isto é, em uma matriz online, a diagonal é conhecida e as outras entradas de uma \textbf{coluna}~$j$ qualquer dependem do índice do mínimo da \textbf{linha}~$j$ correspondente.

Similarmente, dizemos que uma matriz é online nas colunas quando as entradas de suas diagonais são conhecidas à priori e uma entrada~$A[i][j]$ fora da diagonal pode depender do índice de mínimo da coluna~$i$.
\end{defi}

\begin{figure}[b]
    \centering
    % === Based On ===
% Geometric representation of the sum 1/4 + 1/16 + 1/64 + 1/256 + ...
% Author: Jimi Oke
% ================

\begin{tikzpicture}[scale=.35]\footnotesize
 \pgfmathsetmacro{\n}{5}

\begin{scope}<+->;
% grid
  \foreach \i in {1,...,\n} {
      \draw[gray,very thin] (\n-\i,\i-1) -- (\n,\i-1);
      \draw[gray,very thin] (\i-1,\n-\i) -- (\i-1,\n);
  }
  \draw[gray,very thin] (0,\n) -- (\n,\n);
  \draw[gray,very thin] (\n,0) -- (\n,\n);
\end{scope}

% function
\begin{scope}[pattern=north west lines,pattern color=red]
  \fill (\n-2,2) rectangle (\n,1);
  \fill (\n-1,1) rectangle (\n,0);
\end{scope}
\begin{scope}[pattern=vertical lines,pattern color=green]
  \fill (\n-3,2) rectangle (\n-2,\n);
\end{scope}

\end{tikzpicture}

    \caption{Dependências de uma matriz online por linhas.} \label{Online:matrix:fig}
\end{figure}

Note que a definição permite que outras entradas além das diagonais sejam conhecidas à priori. Dizemos ainda que uma entrada da matriz está disponível se é possível descobrir seu valor, isto é, ela era conhecida à priori ou ela só depende de informações já calculadas. Além de online, vamos exigir que as matrizes onde trabalhamos sejam triangulares inferiores, isto é, cada entrada~$A[i][j]$ da matriz onde~$i \leq j$ é indefinida.

Vamos apresentar métodos para encontrar índices ótimos nestas matrizes quando elas forem totalmente monótonas. Trabalharemos durante esta seção especificamente com o caso onde estamos interessados nos mínimos das linhas de uma matriz totalmente monótona nas linhas e online nas linhas, assim como nas seções anteriores, é fácil adaptar estes algoritmos para trabalhar em termos de colunas ou máximos, porém, os casos convexo e côncavo são resolvidos de maneira diferente e serão explicados individualmente. É interessante perceber também que já que a matriz é triangular inferior a linha 1 não possui nenhum valor definido e, portanto, não possui mínimo. 

A Figura~\ref{Online:matrix:fig} ilustra as relações de dependência nas matrizes tratas nesta seção. As entradas não definidas estão pintadas em cinza e as entradas de uma dada coluna, como a com linhas verticais em verde dependem dos valores das entradas da linha correspondente que neste caso estão hachuradas em vermelho. O Problema~\ref{Online:prob} exemplifica a utilidade de trabalhar com matrizes deste tipo.

\begin{prob} \label{Online:prob}
Dada uma matriz de custos~$C \in \B{Q}^{n \times n}$, computar o vetor~$E$ tal que~$E[1] = 0$ e para todo~$i \in [1 \tdots n]$ vale~$E[i] = \min\limits_{j < i} C[i][j] + E[j]$.
\end{prob}

Com uma matriz~$C$ e um vetor~$E$ definidos como no Problema~\ref{Online:prob}, criamos a matriz~$B \in \B{Q}^{n \times n}$ de forma que para todo~$i,j \in [n]$ vale
\begin{equation} \label{Online:Bmat}
    B[i][j] = \begin{cases}
        \text{indefinida} & \text{, se } i \leq j \text{ e } \\
        C[i][j] + E[j]    & \text{, caso contrário.}
    \end{cases}
\end{equation}

Primeiramente, é fácil observar que a matriz~$B$ é online nas linhas e triangular inferior, além disso, se~$R$ é o vetor de índices de mínimos das linhas de~$B$, vale~$E[i] = B[i][R[i]]$ para toda linha~$i \in [2 \tdots n]$ de~$B$, portanto, resolver o Problema~\ref{Online:prob} é encontrar os índices de mínimos das linhas de~$B$.É interessante notar que a linha 1 de uma matriz triangular inferior não possui mínimo, já que todos os seus valores são indefinidos.

Como já dito, vamos discutir maneiras eficientes encontrar os índices ótimos de uma matriz online nas linhas e triangular inferior no caso onde essa matriz é totalmente monótona nas linhas. No Problema~\ref{Online:prob} é fácil aplicar o Lema~\ref{Monge:theo+1} para perceber que se a matriz~$C$ é Monge, a matriz~$B$ é Monge no mesmo sentido e, portanto, totalmente monótona nas linhas. Além disso, por construção, a matriz~$B$ é online nas linhas e triangular inferior, assim, as técnicas discutidas nesta seção são suficientes para resolver o Problema~\ref{Online:prob}.

Nos voltamos agora para a solução do problema proposto, isto é, encontrar mínimos das linhas de uma matriz online nas linhas totalmente monótona nas linhas. Lembramos que, já que estamos trabalhando com minimização de linhas, dizemos que uma coluna~$a$ de~$A$ é melhor do que uma outra coluna~$b$ de~$A$ em uma linha~$i$ de~$A$ se tem valor menor ou se tem valor igual e índice menor, isto é, se~$A[i][a] < A[i][b]$ ou se~$A[i][a] = A[i][b]$ e~$a < b$, dizemos também que~$a$ é pior do que~$b$ quando~$b$ é melhor do que~$a$. Para resolver este problema, usaremos uma estrutura de dados que chamamos de envelope.

\begin{defi}
Um envelope~$\Cl{E}$ sobre uma matriz~$A$ é uma estrutura de dados que mantém um conjunto de colunas~$J$ da matriz e é capaz de responder às seguintes operações:

\begin{description}
    \item[\textsc{Insere}($\Cl{E}$,$j$)] Insere uma coluna~$j$ disponível da matriz na estrutura de dados e
    \item[\textsc{Calcula}($\Cl{E}$,$i$)] Retorna~$\min\{j \in J \mid A[i][j] \leq A[i][j'] \text{ para todo } j' \in J\}$ onde~$i \in [n]$ e~$J$ é não vazio.
\end{description}
\end{defi}

Com esta definição, vamos descrever genericamente o Algoritmo~\ref{Online:algo} que retorna o vetor~$E$ de mínimos das linhas da matriz~$A$.

\begin{algorithm}[h]
\caption{Mínimos de linhas online}
\label{Online:algo}
\begin{algorithmic}[1]
\Function{Online}{A, n}
    \State $\Cl{E}$ é um envelope sobre a matriz~$B$
    \State $\textsc{Insere}(\Cl{E},1)$

    \For{$i$ de $2$ até $n$} \label{Online:algo:loop}
        \State $R[i] = \textsc{Calcula}(\Cl{E},i)$
        \State $\textsc{Insere}(\Cl{E},i)$
    \EndFor

    \State \Return $R$
\EndFunction
\end{algorithmic}
\end{algorithm}

O algoritmo acima funciona pois em toda iteração do laço da linha~\ref{Online:algo:loop} uma coluna disponível é inserida no envelope e a variável~$R[i]$ recebe o valor~$\min\{ j < i \mid B[i][j] \leq B[i][j'] \text{ para todo } j' < i\}$, que é o índice de mínimo da linha~$i$, isto é, o algoritmo calcula as entradas de~$R$ corretamente.

Falta agora descrever a estrutura de envelope tanto para o caso côncavo quanto para o caso convexo. Nos dois casos temos a garantia de que o parâmetros passados tanto para \textsc{Insere} quanto para \textsc{Calcula} são dados em ordem crescente, isto é, as colunas são inseridas em ordem crescente de índice na estrutura e os mínimos das linhas também são calculados em ordem crescente. Inicialmente, nas Subseções~\ref{Online:convex} e~\ref{Online:concave} aproveitaremos apenas a propriedade descrita para a operação \textsc{Insere} e depois, na Subseção~\ref{Online:linear} utilizaremos também a propriedade descrita para \textsc{Calcula}.

%------------------------------------------------------------------

\subsection{Caso convexo} \label{Online:convex}
Discutiremos agora a implementação de um envelope~$\Cl{E}$ sobre uma matriz~$A$ totalmente monótona convexa nas linhas e online nas linhas. Chamamos de~$J$ o conjunto de colunas já inseridas em~$\Cl{E}$.

\begin{defi}[Coluna válida]
Uma coluna~$j \in J$ é válida somente se existe uma linha~$i \in [n]$ para a qual~${j = \textsc{Calcula}(\Cl{E},i)}$, ou seja, ela é melhor do que todas as outras colunas de~$J$ nesta linha. Uma coluna que não respeita esta propriedade é chamada inválida.
\end{defi}

Sejam~$a$ e~$b$ duas colunas de~$A$ tais que~$a < b$. Pela convexidade de~$A$ e pelo Lema~\ref{lema:MonotoneTotallyMonotone} sabemos que a coluna~$a$ é melhor do que a~$b$ para um prefixo potencialmente vazio das linhas e pior para o restante, um sufixo também potencialmente vazio. Definimos a operação~$\textsc{Intersecta}(a,b)$ que devolve o primeiro índice de linha deste sufixo. Formalmente, definimos~${\textsc{Intersecta}(a,b) = \min(\{ i \in [n] \mid A[i][a] > A[i][b] \} \cup \{n+1\})}$. O cálculo desta operação pode ser realizado com busca binária em tempo~$\Cl{O}(\lg(n))$ da maneira descrita no Algoritmo~\ref{Online:convex:BB}. Lembre que neste caso, já que~$a < b$,~$a$ é melhor do que~$b$ numa determinada linha~$p$ se e somente se~${A[p][a] \leq A[p][b]}$.

\begin{algorithm}[h]
\caption{Intersecção de duas colunas no caso convexo}
\label{Online:convex:BB}
\begin{algorithmic}[1]
\Function{Intersecta}{a,b}
    \State $\ell \rec 1$
    \State $r \rec n+1$

    \While{ $\ell < r$}
        \State $p \rec \floor{(\ell + r)/2}$
        \If{ $a$ é melhor do que~$b$ na linha~$p$} \label{Online:convex:BB:ineq}
            \State $\ell \rec p + 1$
        \Else
            \State $r \rec p$
        \EndIf
    \EndWhile
    \State \Return $\ell$
\EndFunction
\end{algorithmic}
\end{algorithm}

Se~$j \in J$ for melhor do que todas as colunas de índice menor em~$J$ em alguma linha, pela total monotonicidade, ela passa a ser melhor em todas as linhas a partir desta, por isso, é interessante definir um valor~$s(j)$ para toda coluna~$j \in J$ que diz a menor linha para a qual~$j$ é melhor do que todas as de índice menor em~$J$ ou~$n+1$ caso isso nunca ocorra.  Formalmente~${ s(j) = \min(\{i \in [n] \mid A[i][j] < A[i][j'] \text{ para todo } j' \in J \text{ tal que } j' < j\} \cup \{n+1\}) }$.

\begin{prop} \label{Online:convex:easys}
Para qualquer~$j \in J$ válido, considerando o conjunto~$\{j' \in J \mid j' \text{ é válido e } j' < j\}$ de colunas válidas em~$J$ de índice menor do que~$j$, vale uma das duas alternativas: O conjunto é vazio e~$s(j) = 1$ ou o conjunto não é vazio e~$s(j) = \textsc{Intersecta}(j^*,j)$ onde~$j^*$ é o máximo do conjunto.

Isto é, para calcular o valor de~$s(j)$ basta escolher o maior~$j^* \in J$ válido de índice menor do que~$j$. Se tal~$j^*$ existe basta calcular sua intersecção com~$j$ e, caso contrário, basta definir~$s(j) = 1$.
\end{prop}

\begin{proof}
Se o conjunto é vazio vale~$s(j) = 1$, já que, por vacuidade,~$A[1][j] < A[1][j']$ para todo~$j' \in J$ onde~$j' < j$. Se existe pelo menos um elemento no conjunto tomamos o~$j^*$ de maior índice do conjunto e sabemos que~$s(j) > 1$, caso contrário, pela total monotonicidade,~$j^*$ seria inválido. Vamos provar que~$j^*$ é ótima em~$s(j) - 1$. Suponha que existe uma~$j' \in J$ diferente de~$j^*$ ótima em~$i$. Separamos em três casos, se~$j'>j$ sabemos que~$j^*$ é melhor do que~$j$ em~$s(j)-1$, portanto~$j'$ também é e, pela total monotonicidade, é melhor do que~$j$ deste ponto em diante, além disso, também pela total monotonicidade,~$j^*$ é melhor do que~$j$ antes desta linha e concluímos que~$j$ é inválida, um absurdo. Se~$j' = j$ vale que~$j$ é ótima em~$s(j) - 1$ o que contraria a definição de~$s(j)$, um absurdo. Resta o caso onde~$j' < j$, porém, já que~$j' \neq j^*$ e~$j^*$ é o máximo do conjunto,~$j' < j^*$ e, pela total monotonicidade,~$j^*$ é pior do que~$j'$ até~$s(j) - 1$ e pior do que~$j$ de~$s(j)$ em diante, o que a torna inválida, um absurdo.

Com isto, sabemos que~$j^*$ é ótima no ponto~$s(j) - 1$, assim, decidir~$s(j)$ é decidir o primeiro momento onde~$j$ é melhor do que~$j^*$, isto é, calcular~$\textsc{Intersecta}(j^*,j)$.
\end{proof}

A Proposição~\ref{Online:convex:easys} dá motivo a uma representação do envelope~$\Cl{E}$ que guarda todos os elementos válidos de~$J$ numa lista ordenada crescentemente. Nesta representação, calcular o~$s(j)$ de um dado~$j$ é fácil. Se~$j$ for o primeiro elemento,~$s(j) = 1$, caso contrário, basta visitar o elemento que precede~$j$ e calcular sua intersecção com~$j$. Vamos utilizar esta representação de~$\Cl{E}$. Além disso, guardaremos para cada elemento~$j$ da lista qual é o seu valor~$s(j)$, o que nos impede de recalcular este valor desnecessariamente. A nossa estrutura deve, então, manter 3 invariantes. Um valor~$j \in J$ pertence à estrutura se e somente se é válido, os valores aparecem em ordem crescente e todos os~$s(j)$ estão calculados corretamente. Note que quando~$J$ é vazio as três invariantes valem trivialmente.

\begin{algorithm}[h]
\caption{Envelope convexo}
\label{Online:envelope:convex}
\begin{algorithmic}[1]
\Function{Calcula}{\Cl{E},i}
    \State $\ell \rec 1$
    \State $r \rec |\Cl{E}|$
    
    \While{ $\ell < r$ }
        \State $p = \floor{ (\ell+r)/2 }$
        \If{ $s(\Cl{E}_p) \leq i$ }
            \State $r = p$
        \Else
            \State $\ell = p+1$
        \EndIf
    \EndWhile

    \State \Return $\ell$
\EndFunction

\Function{Insere}{\Cl{E},j}
    \If{ $\Cl{E}$ é não vazio e~$A[n][\Cl{E}_{-1}] \leq A[n][j]$ }
        \State \Return
    \EndIf
    
    \While{ $\Cl{E}$ é não vazio e~$A[s(\Cl{E}_{-1})][\Cl{E}_{-1}] > A[s(\Cl{E}_{-1})][j]$ }
        \State Remove o final de~$\Cl{E}$
    \EndWhile

    \State Insere~$j$ ao final de~$\Cl{E}$
    \State Calcula e guarda~$s(j)$
\EndFunction
\end{algorithmic}
\end{algorithm}

O Algoritmo~\ref{Online:envelope:convex} implementa as operações~\textsc{Calcula} e~\textsc{Insere} como descrito a seguir. Para responder a uma pergunta da forma~$\textsc{Insere}(\Cl{E},j)$ aproveitamos a garantia já citada de que estas operações estão ordenadas crescentemente, ou seja,~$j > j'$ para todo~$j' \in J$. É possível que~$j$ se torne uma coluna inválida quando adicionada a~$J$ o que, pela total monotonicidade, só ocorre se~$\Cl{E}_{-1}$ (a coluna válida de maior índice) for melhor do que~$j$ na linha~$n$, portanto, verificamos inicialmente se esta condição é verdade e, caso seja, ignoramos a operação. Podemos assumir sem perda de generalidade que~$j$ é válida daqui em diante.

Devemos verificar se a nova coluna~$j$ torna alguma outra coluna de~$J$ inválida e remover esta coluna. Escolhemos a última coluna~$\Cl{E}_{-1}$ da estrutura. Se~$j$ for melhor do que~$\Cl{E}_{-1}$ na linha~$s(\Cl{E}_{-1})$, pela total monotonicidade, a coluna~$\Cl{E}_{-1}$ passa a ser inválida e, por este motivo, removemos ela da estrutura. Se~$j$ for pior, também pela total monotonicidade, é pior em todas as linhas seguintes e mantém a validade de todas as outras colunas da estrutura. Este argumento nos mostra que para remover todos os elementos inválidos de~$\Cl{E}$ basta remover o último elemento enquanto ele for inválido. Com este trabalho mantivemos a invariante de que uma coluna de~$J$ está em~$\Cl{E}$ se e somente se é válida.

Falta atualizar os valores~$s(j')$ para todo~$j' \in \Cl{E}$. Temos a garantia de que inserimos e removemos elementos apenas no final da lista, já que os valores~$s(j')$ dependem do elemento anterior, basta calcular o valor~$s(j)$ do novo elemento adicionado, o que sabemos fazer em tempo~$\Cl{O}(\lg(n))$. Isso garante a invariante de que os valores da função~$s$ estão guardados corretamente. A invariante de que o vetor é ordenado é trivialmente mantida já que só adicionamos elementos no final da lista.

Agora precisamos usar esta estrutura para responder rapidamente a uma pergunta da forma~${ \textsc{Calcula}(\Cl{E},i) }$ com~$i \in [n]$. Queremos encontrar o elemento~$j$ ótimo na coluna~$i$. Este elemento é válido e, portanto, pertence a~$\Cl{E}$. Vamos definir~$j$ como o maior elemento de~$\Cl{E}$ tal que~$s(j) \leq i$. Pela total monotonicidade todo elemento de~$J$ menor do que~$j$ é pior do que~$j$ a partir de~$s(j)$, em particular em~$i$. Além disso já que para todo~$j' \in \Cl{E}$ maior do que~$j$ vale~$s(j') > i$ vale também que~$j$ é melhor do que~$j'$ em~$i$ pela total monotonicidade. O valor~$j$ pode ser encontrado com uma busca binária na lista~$\Cl{E}$.

Assumimos que a nossa implementação da lista~$\Cl{E}$ é capaz de visitar um elemento em uma dada posição em tempo~$\Cl{O}(1)$ além de adicionar e remover elementos do final da lista, isso, em \texttt{C++}, pode ser implementado utilizando a estrutura~\texttt{vector} da biblioteca padrão. Podemos criar uma estrutura que guarda tanto o valor~$j$ quanto o~$s(j)$ correspondente e montar nossa lista fazendo com que cada elemento seja um objeto deste tipo, com isso, acessar ou atualizar o~$s(j)$ de um~$j$ é equivalente a acessar o valor de~$j$. Uma implementação em~\texttt{C++} que respeita estas hipóteses pode ser encontrada em~\texttt{envelope\_convex}. Vamos calcular a complexidade dos algoritmos sugeridos.

Cada operação~\textsc{Calcula} envolve apenas uma busca binária na lista~$\Cl{E}$ onde cada iteração custa tempo constante, já que esta lista tem tamanho no máximo~$n$ (pois~$A$ tem~$n$ colunas), estas chamadas custam~$\Cl{O}(\lg(n))$ cada. Uma operação~\textsc{Insere} pode remover vários elementos de~$\Cl{E}$, custando tempo potencialmente linear, porém, cada elemento de~$\Cl{E}$ só pode ser removido uma única vez, fazendo com que a soma de todas as remoções custe tempo~$\Cl{O}(n)$. Além disso, esta operação insere um novo elemento em~$\Cl{O}(1)$ e calcula o valor~$s$ do novo elemento em tempo~$\Cl{O}(\lg(n))$, desta maneira, cada operação custa tempo~$\Cl{O}(\lg(n))$ amortizado.

%------------------------------------------------------------------

\subsection{Caso côncavo} \label{Online:concave}

Vamos aproveitar várias das ideias e definições do caso convexo para desenvolver um envelope~$\Cl{E}$ sobre uma matriz~$A$ totalmente monótona côncava nas linhas e online nas linhas. Novamente, o conjunto~$J$ representa as colunas já inseridas em~$\Cl{E}$.

Sejam~$a$ e~$b$ colunas de~$A$ onde~$b < a$ (note a inversão). Pela concavidade de~$A$ e pelo Lema~\ref{lema:MonotoneTotallyMonotone}, sabemos que a coluna~$a$ é melhor do que a coluna~$b$ num prefixo potencialmente vazio das linhas e pior no restante das linhas, um sufixo também potencialmente vazio. Perceba que esta operação é escrita exatamente como no caso convexo justamente por causa da inversão da ordem de~$a$ e~$b$. Definimos a operação~$\textsc{Intersecta}(a,b)$ que devolve o primeiro índice de linha deste sufixo. Formalmente definimos~$\textsc{Intersecta}(a,b) = \min\{i \in [n] \mid A[i][a] \geq A[i][b]\}$. Tanto no caso convexo quanto no caso côncavo a operação busca o primeiro índice de linha para o qual o segundo parâmetro é melhor do que o primeiro e isso faz com que o Algoritmo~\ref{Online:convex:BB} também possa ser aplicado aqui, porém, para verificar se~$a$ é melhor do que~$b$ numa dada linha~$p$, devemos verificar se~${ A[p][a] < A[p][b] }$, diferente do que foi feito no caso convexo.

Se algum~$j \in J$ é melhor do que todas as colunas de índice~\textbf{maior} em~$J$ em alguma linha, pela total monotonicidade, ela passa a ser melhor também em todas as linhas a partir desta, por isso, é interessante definir o valor~$s(j)$ para toda coluna~$j \in J$ que diz a menor linha para a qual~$j$ é melhor do que todas as de índice maior em~$J$ ou~$n+1$ caso isso nunca ocorra. Formalmente~${ s(j) = \min(\{i \in [n] \mid A[i][j] \leq A[i][j'] \text{ para todo } j' \in J \text{ tal que } j' > j\} \cup \{n+1\}) }$.

\begin{prop} \label{Online:concave:easys}
Para qualquer~$j \in J$ válido, considerando o conjunto~${ \{j' \in J \mid j' \text{ é válido e } j' > j\} }$ de colunas válidas em~$J$ de índice maior do que~$j$, vale uma das duas alternativas: O conjunto é vazio e~${ s(j) = 1 }$ ou o conjunto não é vazio e~${ s(j) = \textsc{Intersecta}(j^*,j) }$ onde~$j^*$ é o mínimo do conjunto.
\end{prop}

\begin{proof}
A prova vai pelas mesmas linhas da prova da Proposição~\ref{Online:convex:easys}. Se o conjunto é vazio vale~$s(j) = 1$ já que, por vacuidade,~${ A[1][j] \leq A[1][j'] }$  para todo~$j'$ do conjunto. Se existe pelo menos um elemento no conjunto tomamos o~$j^*$ de menor índice e sabemos que~$s(j) > 1$ já que se~$s(j) = 1$, pela total monotonicidade,~$j^*$ seria inválida. Vamos provar que~$j^*$ é ótima em~$s(j) - 1$. Suponha que existe uma~$j' \in J$ diferente de~$j^*$ ótima em~$i$. Separamos em três casos, se~$j' < j$ sabemos que~$j^*$ é melhor do que~$j$ em~$s(j) - 1$ portanto~$j'$ também é e, pela total monotonicidade, é melhor do que~$j$ deste ponto em diante, além disso, também pela total monotonicidade,~$j^*$ é melhor do que~$j$ antes desta linha e concluímos que~$j$ é inválida, um absurdo. Se~$j' = j$ vale que~$j$ é ótima em~$s(j) - 1$ o que contraria a definição de~$s(j)$. Resta o caso onde~$j' > j$, porém, já que~$j' \neq j^*$,~$j^*$ é o mínimo do conjunto e~$j'$ pertence ao conjunto,~$j' > j^*$ e, pela total monotonicidade,~$j^*$ é pior do que~$j'$ até~$s(j)-1$ e pior do que~$j$ de~$s(j)$ em diante, o que a torna inválida, um absurdo.

Com isso, sabemos que~$j^*$ é ótima no ponto~$s(j) - 1$, assim, decidir~$s(j)$ é decidir o primeiro momento onde~$j$ é melhor do que~$j^*$, isto é, calcular~$\textsc{Intersecta}(j^*,j)$.
\end{proof}

Novamente, a Proposição~\ref{Online:concave:easys} indica uma representação do envelope~$\Cl{E}$ que guarda todos os elementos válidos de~$J$ numa lista ordenada, porém, desta vez, vamos ordenar os itens decrescentemente. Calcular o~$s(j)$ de um dado~$j$ ainda é fácil e pode ser realizado exatamente da mesma maneira do caso convexo. Além disso, para cada elemento de~$\Cl{E}$ vamos guardar o valor~$s(j)$. A nossa estrutura deve, novamente, manter 3 invariantes: Um valor~$j \in J$ pertence à estrutura se e somente se é válido, os valores aparecem em ordem decrescente e todos os~$s(j)$ estão calculados corretamente. Note que quando~$J$ é vazio as três invariantes valem trivialmente.

\begin{algorithm}[h]
\caption{Envelope côncavo}
\label{Online:envelope:concave}
\begin{algorithmic}[1]
\Function{Calcula}{\Cl{E},i}
    \State $\ell \rec 1$
    \State $r \rec |\Cl{E}|$
    
    \While{ $\ell < r$ }
        \State $p = \floor{(\ell+r)/2}$
        \If{ $s(\Cl{E}_p) \leq i$ }
            \State $r = p$
        \Else
            \State $\ell = p + 1$
        \EndIf
    \EndWhile

    \State \Return $\ell$
\EndFunction

\Function{Insere}{\Cl{E},j}
    \If{ $\Cl{E}$ é não vazio e~$A[1][\Cl{E}_{1}] < A[1][j]$ }
        \State \Return
    \EndIf
    
    \While{ $\Cl{E}$ tem pelo menos dois elementos e~$A[s(\Cl{E}_{2})-1][\Cl{E}_{1}] > A[s(\Cl{E}_{2})-1][j]$ }
        \State Remove o início de~$\Cl{E}$
    \EndWhile

    \If{ $\Cl{E}$ tem exatamente um elemento e~$A[n][\Cl{E}_1] > A[n][j]$ }
        \State Remove o início de~$\Cl{E}$
    \EndIf

    \State Insere~$j$ no início de~$\Cl{E}$
    \State Calcula e guarda~$s(j)$
\EndFunction
\end{algorithmic}
\end{algorithm}

O Algoritmo~\ref{Online:envelope:concave} implementa as operações~\textsc{Calcula} e~\textsc{Insere} como descrito a seguir. Para responder a uma pergunta da forma~$\textsc{Insere}(\Cl{E},j)$ aproveitamos a garantia já citada de que estas operações estão ordenadas crescentemente, ou seja,~$j > j'$ para todo~$j' \in J$. É possível que~$j$ se torne uma coluna inválida quando adicionada a~$J$ o que, pela total monotonicidade, só ocorre se~$\Cl{E}_1$ (a coluna válida de maior índice) for melhor do que~$j$ na linha~$1$, portanto, verificamos inicialmente se esta condição é verdade e, caso seja, ignoramos a operação. Podemos assumir sem perda de generalidade que~$j$ é válida daqui em diante.

Devemos verificar se a nova coluna~$j$ torna alguma outra coluna de~$J$ inválida e remover esta coluna. Escolhemos a primeira coluna~$\Cl{E}_1$ e vamos descobrir se ela é inválida, para tanto, basta verificar se~$j$ é melhor do que~$\Cl{E}_1$ na última linha onde~$\Cl{E}_1$ é ótima, esta linha é~${ s(\Cl{E}_2) - 1 }$ caso exista~$\Cl{E}_2$ e~$n$ caso contrário. Se~$j$ for melhor, a coluna~$\Cl{E}_1$ se tornou inválida e é removida, caso contrário não há outra coluna inválida na estrutura graças à total monotonicidade. Este argumento nos mostra que para remover todos os elementos inválidos de~$\Cl{E}$ basta remover o primeiro elemento enquanto ele for inválido. Com este trabalho mantivemos a invariante de uma coluna de~$J$ pertence a~$\Cl{E}$ se e somente se é válida.

Para atualizar os valores~$s(j')$ de todo valor~$j' \in \Cl{E}$ basta perceber, mais uma vez que estes não mudam, já que só inserimos e removemos elementos do início de~$\Cl{E}$ e cada um dos~$s(j')$ depende apenas do elemento que antecede~$j'$ na estrutura. Portanto, basta calcular o valor de~$s(j)$ para o elemento~$j$ recém adicionado em tempo~$\Cl{O}(\lg(n))$ garantindo a invariante de que todos os valores de~$s$ estão corretamente calculados. Além destas duas, a invariante de que o vetor é ordenado é trivialmente mantida.

Para responder a pergunta da forma~${ \textsc{Calcula}(\Cl{E},i) }$ com~$i \in [n]$ precisamos descobrir o maior valor~$j \in \Cl{E}$ tal que~$s(j) \leq i$ e, similarmente ao caso convexo, este valor pode ser encontrado com busca binária em tempo~$\Cl{O}(\lg(n))$. As mesmas hipóteses sobre acessos a elementos da estrutura realizadas no caso convexo são feitas aqui e é possível encontrar uma implementação que respeita estas hipóteses em~\texttt{envelope\_concave}. 

Vamos calcular a complexidade dos algoritmos sugeridos. Cada operação~\textsc{Calcula} custa tempo~$\Cl{O}(\lg(n))$, como já citado, e cada elemento de~$J$ só é inserido e removido pela operação~\textsc{Insere} no máximo uma vez, portanto, todas as inserções e remoções custam tempo~$\Cl{O}(1)$ amortizado. Além disso a operação~\textsc{Insere} deve calcular~$s(j)$ ao inserir um elemento~$j$, o que faz com que seu tempo final seja~$\Cl{O}(\lg(n))$ amortizado.

%------------------------------------------------------------------

\subsection{Envelopes em tempo linear} \label{Online:linear}

Um caso especial interessante deste problema é aquele onde matriz~$A$ é definida em termo de retas com coeficientes angulares ordenados, isto é, existem vetores~$a,b \in \B{Q}^n$ tais que~$a$ é monótona e~$A[i][j] = a_j i + b_j$. É fácil provar que se~$a$ é decrescente,~$A$ é totalmente monótona convexa nas linhas e se~$a$ é crescente,~$A$ é totalmente monótona côncava nas linhas.

Este é um bom exemplo de caso para a qual a função~\textsc{Intersecta} pode ser calculada em tempo~$\Cl{O}(1)$. Galil e Park, ao descrever os algoritmos discutidos aqui, fazem com que a utilização deste fato se torne clara. Da forma como foram descritos aqui, é necessário fazer uma observação para atingir esta complexidade.

Durante o desenvolvimento do nosso método, não aproveitamos o fato de sabermos que as operações~\textsc{Calcula} serão chamadas com parâmetro crescente, isto é, as linhas para as quais deveremos descobrir o valor ótimo estarão ordenadas. Para utilizar este fato, vamos manter um ponteiro~$\ell$ para o último item da estrutura que foi resposta de alguma pergunta deste tipo ou para o primeiro item da estrutura, caso nenhuma pergunta tenha sido feita ainda. O Algoritmo~\ref{Online:envelope:linear} reimplementa a operação em questão no caso convexo mostrando como utilizar este ponteiro para responder às perguntas mais eficientemente. Utilizamos a notação~$\ell + 1$ para denotar o elemento seguinte a~$\ell$ na estrutura

\begin{algorithm}[h]
\caption{Calcula no envelope convexo em tempo linear}
\label{Online:envelope:linear}
\begin{algorithmic}[1]
\Function{Calcula}{\Cl{E},i}
    \If{ $\ell$ aponta para um elemento fora de~$\Cl{E}$ }
        \State $\ell$ passa a apontar para~$\Cl{E}_1$
    \EndIf
    
    \While{ $\ell+1$ existe é melhor do que~$\ell$ em~$i$}
        \State $\ell \rec \ell+1$
    \EndWhile

    \State \Return valor de $\ell$
\EndFunction
\end{algorithmic}
\end{algorithm}

Se o ponteiro aponta para alguém fora da estrutura, ele é mandado novamente para o início, se o elemento atual é menor do que o seguinte na linha atual, ele é o ótimo, caso contrário, o ponteiro é movido para o próximo elemento e o algoritmo recomeça. Este algoritmo leva tempo constante amortizado já que o ponteiro só passa a apontar para cada elemento uma única vez, portanto, só pode ser movido no máximo~$n$ vezes. É fácil adaptar este algoritmo para o caso côncavo.

%------------------------------------------------------------------

\subsection{Relação com Pareto}

 caso convexo funciona de maneira parecida como uma fronteira de Pareto em pares~$(j,s(j))$ que restringe a ordem das buscas e inserções e o caso côncavo pode ser interpretado, também, de maneira parecida. É razoável imaginar o motivo de não podermos realizar todas as operações de uma fronteira de Pareto no envelope. A próxima seção explora esta possibilidade generalizando a estrutura de envelope.
