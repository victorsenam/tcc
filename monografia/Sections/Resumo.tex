\chapter*{Resumo}
\label{Resumo}

As matrizes Monge têm propriedades como a monotonicidade total e a monotonicidade, que são exploradas por uma série de algoritmos para a busca de máximos e mínimos em linhas e colunas destas matrizes. Algumas destas técnicas são populares pois são utilizadas para a agilização de problemas clássicos e aparecem com crescente frequência em competições de programação por sua utilidade na solução eficiente de problemas de programação dinâmica.

Busca-se neste trabalho formalizar, compilar e aprofundar os conhecimentos adquiridos sobre estes algoritmos por meio do estudo da literatura já existente a fim de facilitar o aprendizado destas técnicas. Além disso, procura-se generalizar os algoritmos inspirando-se em desafios propostos em competições, explorando suas limitações e modificando as formalizações encontradas na literatura para ampliar os escopos tradicionais. 

\textbf{Palavras-chave:} Monge, programação competitiva, otimização, programação dinâmica, desigualdade quadrangular.

%\chapter*{Abstract}
%\label{Abstract}
%
%The Monge matrices have properties such as the total monotonicity and the monotonicity, which are explored by a series of row and column minima and maxima searching algorithms on these matrices. Some of these techniques are popular because they speed up solutions for classic problems and keep becoming more frequent in competitive programming contests because of their use on efficient solutions for specific dynamic programming problems.
%
%This work aims to formalize, compile and deepen the aquired knowledge regarding these algorithms by studying the existing literature so that it becomes easier to learn about these techniques. Beyond that, it attempts to generalize the algorithms based on challenges proposed in competitions, exploring its limitations and modifying the formalizations found in literature so that their standart applications can be extended.
%
%\textbf{Keywords:} Monge, competitive programming, optimization, dynamic programming, quadrangle inequality.
