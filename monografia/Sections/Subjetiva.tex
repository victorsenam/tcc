\chapter{Parte subjetiva}
\label{Subjetiva}

Estou feliz com o resultado deste trabalho. Trabalhei bastante durante todo o ano e acredito que, principalmente por isso, aprendi muito com ele. Aprendi muito sobre os algoritmos e objetos estudados. Aprendi também sobre como implementar, testar e explicar os tópicos com o máximo de cuidado que consegui.

Eu conhecia, antes de começar o trabalho, alguns dos algoritmos sobre os quais eu escrevi. Deixar eles no papel de uma forma que conversasse com o restante do trabalho, colocando a minha visão sobre eles no texto, foi extremamente trabalhoso. A toda leitura de cada parte do texto, surgiam novas formas de explicar ou novas observações sobre as ideias apresentadas. Aprendi que, em algum momento, é necessário aceitar que não tem como ficar perfeito. Isso é um pouco frustrante, mas me mostra o quanto eu realmente aprendi até mesmo sobre os algoritmos que eu já conhecia e já era capaz de implementar.

Cada texto da literatura estudada apresenta um algoritmo de uma certa maneira, eu decidi apresentar todos os algoritmos em termos de encontrar mínimos das linhas das matrizes. Isso me forçou a repensar inteiramente o algoritmo e o texto, verificando todos os passos e criando testes para ter certeza de que minhas adaptações estavam corretas. Este esforço me fez entender muito bem os tópicos, de forma que esse entendimento me ajuda a visualizar e aplicar os conhecimentos obtidos aqui em problemas que encontro durante competições de programação.

Eu tenho que agradecer à Maratona de Programação. Com isso eu me refiro à comunidade de programação competitiva que é forte tanto no Brasil quanto fora. Esta comunidade faz com que a importância das competições se reafirme, me forçando a continuar crescendo e estudando a fim de conseguir resultados cada vez melhores. Agradeço em especial ao MaratonIME, grupo do qual eu tenho muito orgulho de fazer parte e que espero que continue forte como hoje.

Agradeço também aos meus professores do Instituto de Matemática e Estatística, em especial aos ótimos professores de teoria do departamento de computação. A grande preocupação que estes tem com seus alunos me influenciaram a me interessar tanto por teoria da computação, o que é essencial para que eu me mantenha motivado a aprender e estudar. Além disso, é claro, agradeço ao apoio incessante dos meus amigos, em especial da minha família e da Olivia, que sempre se mostraram orgulhosos e me inspiraram a fazer o esforço que venho fazendo.
