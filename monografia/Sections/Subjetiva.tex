\chapter{Parte subjetiva}
\label{Subjetiva}

Nesta seção começamos a estudar formas de encontrar índices ótimos em matrizes onde certas entradas são desconhecidas a priori. Inicialmente o formato das matrizes abordadas será explicitado juntamente com um exemplo de um caso específico interessante desta técnica apresentado por Galil e Park~\cite{Galil:1992}. Galil e Giancarlo~\cite{Galil:1989} mostraram que este problema pode ser resolvido em tempo~$\Cl{O}(n\lg(n))$ para matrizes~$n \times n$ convexas e côncavas, apresentaremos estas soluções aqui. O caso convexo é resolvido na Subseção~\ref{Online:convex}. Na Subseção~\ref{Online:concave} mostramos como adaptar o que foi discutido para o caso côncavo. Em certas matrizes, é possível melhorar o tempo destes algoritmos de~$\Cl{O}(n\lg(n))$ para~$\Cl{O}(n)$, esta possibilidade é explicada na Subseção~\ref{Online:linear}.

