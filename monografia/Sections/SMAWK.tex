\section{SMAWK}
\label{SMAWK}

%----------------------------------------------------------------------------------------

Nesta seção discutiremos o algoritmo SMAWK. Ele é conhecido pela sua aplicação no problema de encontrar o vértice mais distante de cada vértice num poígono convexo em tempo linear~\cite{Aggarwal:1987}. Ao final desta seção serão explicadas esta e outras aplicações deste algoritmo.  

Dada uma matriz $A \in \B{Q}^{n \times m}$, listamos os casos de uso deste algoritmo:
\begin{itemize}
    \item Se $A$ é totalmente monótona convexa ou côncava nas linhas podemos encontrar os índices de mínimos e máximos das linhas em tempo $\Cl{O}(n + m)$ e
    \item se $A$ é totalmente monótona convexa ou côncava nas colunas podemos encontrar os índices de mínimos e máximos das colunas em tempo $\Cl{O}(n + m)$.
\end{itemize}

Apresentaremos o caso onde $A$ é totalmente monótona convexa nas linhas e estamos interessados nos índices de mínimos. Na Subseção~\ref{SMAWK_Generalizacoes} explicamos como reduzir os problemas elencados para este caso.
%----------------------------------------------------------------------------------------

\subsection{Técnica Primordial}
Para facilitar a compreensão do algoritmo SMAWK, iremos apresentar uma técnica parecida com a Divisão e Conquista apresentada na Seção~\ref{DivisaoEConquista} e mostrar uma otimização desta técnica que leva ao algoritmo SMAWK.

Dada uma matriz~$A \in \B{Q}^{n \times m}$ totalmente monótona convexa por linhas, queremos encontrar o menor índice de mínimo de cada uma das linhas de~$A$. Se para uma dada linha~$i$ onde~$i > 0$ e~$i < n$ conhecermos os índices~$l$ e~$r$ de mínimos das linhas~$i-1$ e~$i+1$, respectivamente, já que~$A$ tem os índices de mínimos das linhas crescente (por ser totalmente monótona) basta buscar o índice de mínimo da linha~$i$ no intervalo entre~$l$ e~$r$ (inclusive). Além disso, se~$i$ é a primeira linha da matriz podemos considerar~$l = 1$ ou se~$i$ é a última linha da matriz podemos considerar~$r = n$ sem perder a validade do fato de que basta buscar entre~$l$ e~$r$.  

Após realizar as observações acima note que, já que~$A$ é totalmente monótona, remover qualquer linha de~$A$ mantém a total monotonicidade e não altera o índice de mínimo de outra linha. Com esta observação, concluímos que podemos remover todas as linhas pares da matriz, resolver o problema recursivamente para a matriz resultante e utilizar este resultado para calcular os índices de interessa para as linhas pares da matriz.  

Com uma análise similar à realizada para a técnica da Divisão e Conquista é fácil concluir que uma implementação desta técnica que consiga remover as linhas pares da matriz (e adicionar elas de volta) em tempo~$\Cl{O}(1)$ resolve o problema em tempo~$\Cl{O}((n+m)\lg(n))$, assim como a técnica da divisão e conquista.  
%----------------------------------------------------------------------------------------

\subsection{Reduce}
Queremos agilizar a técnica apresentada acima. Para isso, vamos adicionar uma nova hipótese. Vamos supor que a matriz~$A$ é quadrada, ou seja,~$n = m$. Lembre que a cada passo, removemos as~$\floor{n/2}$ linhas pares da matriz gerando uma nova matriz $A'$, resolvemos o problema recursivamente para~$A'$ e usamos a solução de~$A'$ para resolver para as linhas restantes de~$A$. O problema é que quando removemos linhas da nossa~$A$, ela deixa de ser quadrada e passa a ser uma matriz com mais colunas do que linhas, isto é,~$m \geq n$. Chamamos de interessantes as colunas que contém o índice de mínimo de pelo menos uma linha, queremos remover colunas não interessantes de uma matriz~$A$ com mais colunas do que linhas fazendo com que~$A$ se torne quadrada.  

Primeiramente, note que sempre é possível remover pelo menos $m-n$ colunas da matriz, já que no máximo $n$ colunas podem ser interessantes (o mínimo de uma coluna não pode estar em duas colunas), assim, sempre podemos remover colunas de forma a deixar $A$ quadrada. 
%----------------------------------------------------------------------------------------

\subsection{SMAWK}
%----------------------------------------------------------------------------------------

\subsection{Análise}
%----------------------------------------------------------------------------------------

\subsection{Implementação}
%----------------------------------------------------------------------------------------

\subsection{Generalizações} \label{SMAWK_Generalizacoes}
