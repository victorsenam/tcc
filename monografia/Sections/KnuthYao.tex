\section{Otimização de Knuth-Yao}
\label{KY}

%----------------------------------------------------------------------------------------

O problema da árvore de busca binária ótima~\cite{CLRS} é um exemplo clássico de aplicação de programação dinâmica que é facilmente resolvido em tempo~$\Cl{O}(n^3)$, onde~$n$ é o número de chaves da árvore, com uma modelagem que associa o custo de cada subárvore a uma entrada de uma matriz~${ n \times n }$, reduzindo o problema a calcular estas entradas por uma recorrência. Aproveitando algumas propriedades dessa matriz, Knuth~\cite{Knuth:1971} apresentou uma solução que calcula suas entradas em tempo~$\Cl{O}(n^2)$, resolvendo o problema original nesta complexidade.

Mais tarde, a solução de Knuth foi estudada por Yao~\cite{Yao:1980,Yao:1982}, que mostrou que as propriedades observadas por Knuth eram consequência do fato de que a matriz em questão era Monge convexa. Desta maneira, foi possível perceber que a otimização de Knuth poderia ser útil em vários outros problemas de programação dinâmica. 

Bein, Golin, Larmore e Zhang~\cite{Bein:2009} buscaram enfraquecer a condição encontrada por Yao e mostraram que as matrizes derivadas dos problemas agilizados com a otimização de Knuth-Yao podem ser decompostas de três maneiras diferentes em matrizes totalmente monótonas. Com estas decomposições, fica possível resolver ainda mais problemas, inclusive em casos em que a matriz não é totalmente conhecida à priori. As descobertas do artigo citado neste parágrafo serão discutidas aqui. Esta introdução foi baseada em tal artigo.

Vamos discutir os resultados observados por Yao, descrever a técnica desenvolvida por Knuth e aplicar este conhecimento para resolver um problema de programação dinâmica.

%----------------------------------------------------------------------------------------

\subsection{Definições básicas} \label{KY:defs}

Vamos apresentar a otimização de Knuth-Yao para problemas de minimização, o que nos leva a trabalhar com convexidade. É fácil adaptar o conhecimento discutido nesta seção para problemas de maximização, porém, em vez da convexidade, a concavidade deve ser usada para provar os resultados e modelar os problemas de interesse. Note que, para realizar esta adaptação, alguns conceitos devem ser definidos de maneira levemente diferente. As entradas com~$+\infty$ da Definição~\ref{ KY:recint } abaixo, por exemplo, se tornam~$-\infty$.

\begin{defi}[Recorrência de intervalos] \label{ KY:recint }
Uma matriz~$A \in \B{Q}^{n \times n}$ é considerada uma recorrência de intervalos se existe uma matriz~$C \in \B{Q}^{n \times n}$ tal que, para todo~$i,j \in [n]$,
\begin{equation*}
A[i][j] = \begin{cases}
C[i][j]                                                           & \text{, se } i = j \text{, }  \\
C[i][j] + \min\{A[i][k-1] + A[k][j] \mid i < k \leq j \}          & \text{, se } i < j \text{ e } \\
+\infty                                                           & \text{ caso contrário. }
\end{cases}
\end{equation*}
A matriz~$C$ é chamada de matriz de custos de~$A$.
\end{defi}

É fácil resolver uma recorrência desta forma em tempo~$\Cl{O}(n^3)$. 

\begin{defi}[Matriz de cortes ótimos]
Se~$A \in \B{Q}^{n \times n}$ é uma recorrência de intervalos com matriz de custos~$C$, definimos a matriz de cortes ótimos~$P$ de~$A$. Para todo~$i \in [n]$,~$P[i][i] = i$ e, para todo~$j \in [n]$ com~$i < j$, 
$$P[i][j] = \min\{k \mid i < k \leq j \text{ e } A[i][j] = C[i][j] + A[i][k-1] + A[k][j]\} \text{.}$$
\end{defi}

Assim, a matriz~$P$ guarda, para cada~$i < j$, o menor argumento para o qual a função de mínimo na definição de~$A[i][j]$ atinge seu valor ótimo. Note que, enquanto descobrimos os valores da matriz~$A$ em tempo~$\Cl{O}(n^3)$, descobrimos também os valores de~$P$.

\begin{defi}[Knuth-Yao otimizável]
Se~$A \in \B{Q}^{n \times n}$ é uma recorrência de intervalos e~$P$ é sua matriz de cortes ótimos. Dizemos que~$A$ é Knuth-Yao otimizável se, para todo~$i,j \in [n]$ com~$i < j$, vale que~${P[i][j-1] \leq P[i][j] \leq P[i+1][j]}$.
\end{defi}

Vamos mostrar que se~$A \in \B{Q}^{n \times n}$ é Knuth-Yao otimizável, tanto~$A$ quanto sua matriz de cortes ótimos~$P$ podem ser calculadas em~$\Cl{O}(n^2)$.

%----------------------------------------------------------------------------------------

\subsection{Técnica}

\newcommand{\KY}{\textsc{KnuthYao}}
\begin{algorithm}[h]
\caption{Otimização de Knuth-Yao}
\label{KY:algo}
\begin{algorithmic}[1]
\Function{\KY}{C, n} 
    \For{$i$ de $1$ até $n$}
        \State $A[i][i] \rec C[i][i]$
        \State $P[i][i] \rec i$
    \EndFor
    \For{$d$ de $1$ até $n-1$}
        \For{$i$ de $1$ até $n-d$}
            \State $j \rec i+d$
            \State $A[i][j] \rec +\infty$
            \For{$k$ de $P[i][j-1]$ até $P[i+1][j]$} \label{KY:algo:loop}
                \State $v \rec C[i][j] + A[i][k-1] + A[k][j]$
                \If{$v < A[i][j]$} 
                    \State $A[i][j] \rec v$
                    \State $P[i][j] \rec k$
                \EndIf
            \EndFor
        \EndFor
    \EndFor
    \State \Return~$(A,P)$
\EndFunction
\end{algorithmic}
\end{algorithm}

Seja~$A \in \B{Q}^{n \times n}$ uma matriz Knuth-Yao otimizável. Vamos calcular as entradas~$A[i][j]$ onde~$i \leq j$ em ordem crescente de~$j - i$, ou seja, as entradas~$A[i][i]$ serão calculadas para todo~$i$, seguidas das~$A[i][i+1]$,~$A[i][i+2]$ e assim por diante. É possível calcular as entradas nesta ordem pois ela respeita as relações de dependência da matriz~$A$, isto é, ao calcular uma entrada~$A[i][j]$ qualquer, todas as entradas~$A[i][k-1]$ e~$A[k][j]$ com~$i < k \leq j$ já estarão disponíveis e, portanto, será possível descobrir o valor de~$A[i][j]$ das entradas previamente calculadas.

Se calcularmos também as entradas da matriz~$P$ enquanto calculamos as da~$A$, poderemos aproveitar o fato de que~$A$ é Knuth-Yao otimizável para buscar o valor de uma entrada de~$A$ em um intervalo menor do que o trivial, isto é, podemos escrever, para todo~$i,j \in [n]$ com~$i < j$, a igualdade
\begin{equation} \label{KY:eq}
A[i][j] = C[i][j] + \min\{A[i][k-1] + A[k][j] \mid i < k \leq j \text{ e } P[i][j-1] \leq k \leq P[i+1][j]\} \text{.}
\end{equation}

Esta observação induz o Algoritmo~\ref{KY:algo} para calcular as entradas das matrizes~$A$ e~$P$. Na linha~\ref{KY:algo:loop} a variável~$k$ varia de~$P[i][j-1]$ até~$P[i+1][j]$ e não de~${ \max(i+1,P[i][j-1]) }$ até~$\min(j,P[i+1][j])$, como indica a igualdade~\eqref{KY:eq}. Perceba que~${ P[i+1][j] \leq j }$, portanto~${ \min(j,P[i+1][j]) = P[i+1][j] }$. Além disso, sabemos que~${ P[i][j-1] \geq i }$, portanto separamos em dois casos. No primeiro,~${ P[i][j-1] > i }$ e vale~$\max(i+1,P[i][j-1]) = P[i][j-1]$. No outro, onde~$P[i][j-1] = i$, a iteração onde~$k=i$ ocorre indevidamente mas, note que, quando~$k = i$,~$i > k-1$ e~$A[i][k-1] = +\infty$, portanto~$C[i][j] + A[i][k-1] + A[k][j] = +\infty$ e esta iteração será irrelevante para a resposta final do algoritmo, o que nos permite realizar a iteração~$k=i$ sem problemas.

%----------------------------------------------------------------------------------------

\subsection{Análise}

Vamos analisar a complexidade do Algoritmo~\ref{KY:algo}. Podemos escrever a quantidade de iterações do laço da linha~\ref{KY:algo:loop} como
\begin{equation} \label{KY:comp}
{\sum\limits_{d = 1}^{n-1} \sum\limits_{i=1}^{n-d} \sum\limits_{k=P[i][i+d-1]}^{P[i+1][i+d]} 1 = \sum\limits_{d = 1}^{n-1} \sum\limits_{i=1}^{n-d} ( P[i+1][i+d] - P[i][i+d-1] + 1} ) \text{, }
\end{equation}
com um~$d$ fixo, a soma~$\sum\limits_{i=1}^{n-d} ( P[i+1][i+d] - P[i][i+d-1] )$ é uma soma telescópica e tem valor igual a~$P[n-d+1][n] - P[1][1+d] = \Cl{O}(n)$. Com isso, escrevemos~\eqref{KY:comp} como~$\sum\limits_{d=1}^{n-1} \Cl{O}(n) + n - 1 = \Cl{O}(n^2)$.

%----------------------------------------------------------------------------------------

\subsection{Quebrando strings}

Para exemplificar a otimização de Knuth e apresentar a relação das matrizes Monge com as definições da Subseção~\ref{KY:defs}, iremos resolver um outro problema clássico de programação dinâmica~\cite[Exercício~15-9]{CLRS} disponível no juíz online SPOJ em \url{http://www.spoj.com/problems/BRKSTRNG/}. 

Considere uma linguagem de processamento de strings que consegue quebrar uma string~$s$ de tamanho~$m > 1$ em qualquer posição~$t \in [m-1]$, ou seja, gerar duas strings~$s[1 \tdots t]$ e~$s[t+1 \tdots m]$. Um programador quer usar esta linguagem para separar uma string~$n$ vezes, nas posições~${p_1 < p_2 < \dots < p_{n}}$, porém, para quebrar uma string de tamanho~$m$ em qualquer posição, a linguagem gasta tempo~$m$. Queremos descobrir qual é a melhor ordem para realizar estes cortes. 

Suponha, por exemplo, que estamos interessados em quebrar uma string~\texttt{stringdeexemplo} de tamanho 15 nas posições 6 e 8 para gerar as strings~\texttt{string},~\texttt{de} e~\texttt{exemplo}. Isso pode ser realizado de duas maneiras. Uma maneira é quebrar primeiro na posição 8, gerando as strings~\texttt{stringde} e~\texttt{exemplo} e depois na posição 6, gerando as três strings desejadas. A outra maneira é quebrar primeiro na posição 6, gerando~\texttt{string} e~\texttt{deexemplo}, e depois na posição 8. A primeira opção tem custo~$15 + 8$, enquanto a segunda tem custo~$15 + 9$, o que faz a resposta ótima ser a primeira alternativa.

Dados os valores~$n$,~$m$ e os pontos~$p_1 < p_2 < \dots < p_{n}$ dos cortes desejados, chamamos de~$s$ a string que desejamos separar e definimos, por conveniência,~$p_0 = 0$ e~$p_{n+1} = m$. Assim, se~${A \in \B{Q}^{n \times n}}$ guarda em toda posição~$A[i][j]$ com~$i \leq j$, a melhor solução para o subproblema que recebe a string~${s[p_{i-1} + 1 \tdots p_{j+1}]}$ e as posições de corte~$p_{i},p_{i+1},\dots,p_{j}$ como entrada, podemos concluir facilmente que~$A$ é uma matriz de recorrência de intervalos com matriz de custo~$C$ onde~$C[i][j] = p_{j+1} - p_{i-1}$. O valor de~$A[1][n]$ nos dará o tempo mínimo de concluir a tarefa desejada e uma ordem ótima das quebras pode ser reconstruída através da matriz de cortes ótimos de~$A$.

Como observado anteriormente, se~$A$ é uma recorrência de intervalos, então~$A$ pode ser calculada em tempo~$\Cl{O}(n^3)$. Pretendemos aproveitar propriedades da matriz~$C$ e alguns resultados provados por Yao~\cite{Yao:1980} para concluir que~$A$ é Knuth-Yao otimizável e aplicar o Algoritmo~\ref{KY:algo} para calcular~$A$ em tempo~$\Cl{O}(n^2)$. Vamos começar provando a seguinte proposição.

\begin{prop}
$C$ é uma matriz Monge convexa.
\end{prop}

\begin{proof}
Sejam~$i,j \in [n-1]$ quaisquer. Temos 
\begin{align*}
    C[i][j] + C[i+1][j+1] &= p_{j+1} - p_{i-1} + p_{j+2} - p_{i} \\
                          &= p_{j+1} - p_{i} + p_{j+2} - p_{i-1} \\
                          &= C[i+1][j] + C[i][j+1] \text{.}
\end{align*}
Com isso, vale que~$C[i][j] + C[i+1][j+1] \leq C[i+1][j] + C[i][j+1]$ e usamos o Teorema~\ref{Monge:theo+1} para concluir que~$C$ é Monge convexa.
\end{proof}

\begin{defi}[Monótona nos intervalos] \label{KY:MonInt}
Uma matriz~$C \in \B{Q}^{n \times n}$ é monótona nos intervalos se para todo~$i,i',j,j' \in [n]$ onde~$i \leq i' \leq j \leq j'$, vale que
$$C[i'][j] \leq C[i][j'] \text{.}$$
\end{defi}

A Definição~\ref{KY:MonInt} relaciona a distância entre os índices de linha e coluna da matriz com o valor da matriz. Aplicando esse conceito ao problema discutido nesta subseção, dizer que a matriz~$C$ é monótona nos intervalos é equivalente a dizer que, quanto maior a string que está sendo cortada, mais caro é o corte. Vamos provar que esta propriedade vale. Sejam~${i,i',j,j' \in [n]}$ tais que~${ i \leq i' \leq j \leq j' }$. Vale que~${ C[i'][j] = p_{i'+1} - p_{j-1} }$, por definição. Já que~${ p_{i'+1} \geq p_{i+1} }$ e~${ -p_{j-1} \geq -p_{j'-1} }$, temos~${ p_{i' + 1} - p_{j-1} \geq p_{i+1} - p_{ j'-1} = C[i][j'] }$.

\begin{lema} \label{KY:CtoA}
Se~$A \in \B{Q}^{n \times n}$ é uma recorrência de intervalos com matriz de custos~$C$ Monge convexa e monótona nos intervalos, então~$A$ é Monge convexa.
\end{lema}

\begin{proof}
Sejam~$A$ e~$C$ matrizes que respeitam as condições do enunciado e~$P$ a matriz de cortes ótimos de~$A$. Sejam ainda~$i,i',j$ e $j' \in [n]$ onde~$i \leq i'$ e $j \leq j'$. Queremos mostrar que sempre vale a desigualdade de Monge, isto é,~${A[i][j] + A[i'][j'] \leq A[i][j'] + A[i'][j]}$. Seja~$l = j' - i$. Usaremos indução em~$l$. Se~${l < 0}$, vale que~${j > i'}$ e, portanto~${A[i'][j] = +\infty}$ o que faz valer a desigualdade.

Fixamos~$l \geq 0$, assumindo que a desigualdade vale nos casos onde~${j' - i < l}$. Note que se~${i = i'}$ ou~${j = j'}$, a desigualdade vale trivialmente, bem como no caso onde~${j > i'}$, analisado acima. Podemos assumir que~$i < i' \leq j < j'$ e separamos isso em dois casos: quando~$i' = j$ e quando~$i' < j$.

Olhamos para o caso em que~${i' = j}$. Seja~${x = P[i][j']}$ o ponto de corte ótimo do estado~$A[i][j']$, ou seja,~${A[i][j'] = C[i'][j] + A[i'][x-1] + A[x][j]}$. Assumimos que~${x \leq j}$ e temos que
\begin{align}
A[i][j] + A[i'][j'] - A[i'][j] &= A[i][j] + A[j][j'] - A[j][j] \nonumber \\
                               &\leq C[i][j] + A[i][x-1] + A[x][j] + A[j][j'] - A[j][j] \label{KY:CtoA:1} \\
                               &\leq C[i][j] + A[i][x-1] + A[x][j'] \label{KY:CtoA:2} \\
                               &\leq C[i][j'] + A[i][x-1] + A[x][j'] \label{KY:CtoA:3} \\
                               &= A[i][j'] \text{,} \nonumber
\end{align}
onde~\eqref{KY:CtoA:1} vale pois~${i < x \leq j}$ e~${A[i][j] = C[i][j] + \min\limits \{ A[i][k-1] + A[k][j]) \mid k \in [i+1 \tdots j]\} }$, a desigualdade~\eqref{KY:CtoA:2} vale pois~${x \leq j \leq j'}$ e~$j' - x < l$, portanto, pela hipótese de indução, vale que~${A[x][j] + A[j][j'] \leq A[x][j'] + A[j][j]}$ e, finalmente, também vale~\eqref{KY:CtoA:3} pois~$C$ é monótona nos intervalos. Com isso, provamos que~${A[i][j] + A[i'][j'] \leq A[i'][j] + A[i][j']}$. No caso em que~${x > j}$, basta utilizar~${A[j][j'] \leq C[j][j'] + A[j][x-1] + A[x][j']}$ em vez do que foi utilizado em~\eqref{KY:CtoA:1} e adaptar o passo~\eqref{KY:CtoA:2} para aplicar a hipótese de indução em~${i \leq j \leq x-1}$.

Agora, vamos resolver o caso onde~${i' < j}$. Definimos os pontos ótimos de corte~${x = P[i][j']}$ e~${y = P[i'][j]}$ e seguimos um raciocínio parecido com o caso anterior. Assumindo que~${x \leq y}$, temos 
\begin{align}
A[i][j] + A[i'][j'] &\leq C[i][j] + C[i'][j'] + A[i][x-1] + A[x][j] + A[i'][y-1] + A[y][j'] \label{KY:CtoA:4} \\
                    &\leq C[i][j] + C[i'][j'] + A[x][j'] + A[y][j] + A[i][x-1] + A[i'][y-1] \label{KY:CtoA:5} \\
                    &\leq C[i][j'] + A[i][x-1] + A[x][j'] + C[i'][j] + A[i'][y-1] + A[y][j] \label{KY:CtoA:6} \\
                    &= A[i][j'] + A[i'][j] \text{,} \nonumber
\end{align}
onde~\eqref{KY:CtoA:4} se justifica pois~${i < x \leq j}$ e~${i' < y \leq j'}$, a hipótese de indução é aplicada em~\eqref{KY:CtoA:5} pois~${x \leq y \leq j \leq j'}$ e~${j' - x < l}$ e, por fim,~\eqref{KY:CtoA:6} vale pelo fato de que~$C$ é Monge convexa. O caso em que~${x > y}$ é similar. Basta observar que~${i' < x \leq j'}$ e~${i < y \leq j}$, o que implica que~${A[i][j] + A[i'][j'] \leq C[i][j] + C[i'][j'] + A[i][y-1] + A[y][j] + A[i'][x-1] + A[x][j']}$ e substituir o passo~\eqref{KY:CtoA:4} por essa desigualdade, adaptando o passo~\eqref{KY:CtoA:5} de acordo, usando indução em~${i \leq i' \leq y - 1 \leq x - 1}$.
\end{proof}

Com o Lema~\ref{KY:CtoA}, percebemos que a matriz~$A$ que representa o nosso problema atual é Monge convexa.

\begin{theo} \label{KY:TMtoKY}
Uma recorrência de intervalos~$A$ Monge convexa é Knuth-Yao otimizável.
\end{theo}

\begin{proof}
Seja~$A \in \B{Q}^{n \times n}$ uma matriz Monge convexa que é uma recorrência de intervalos com matriz de cortes ótimos~$P$. Queremos mostrar que, para todo~$i,j \in [n]$ com~$i < j$, vale que~${P[i][j-1] \leq P[i][j] \leq P[i+1][j]}$. Observe que o caso em que~${i + 1 = j}$ é trivial e assuma que~${i + 1 < j}$.

Sejam~${x,y \in[n]}$ tais que~$x < y$. Vamos mostrar que se~${A[i][x-1] + A[x][j] \leq A[i][y-1] + A[y][j]}$ então~${A[i][x-1] + A[x][j-1] \leq A[i][y-1] + A[y][j-1]}$. Assumindo a primeira desigualdade, vale que~${A[i][x-1] - A[i][y-1] \leq A[y][j] - A[x][j]}$ e, já que~${x < y}$ e~${j-1 < j}$, usando a Monge convexidade de~$A$, temos que~${A[x][j-1] - A[y][j-1] \leq A[x][j] - A[y][j]}$. Com isso, obtemos que~${A[i][x-1] + A[x][j-1] \leq A[i][y-1] + A[y][j-1]}$, o que conclui a prova da implicação proposta. A existência de algum par~$i,j \in [n]$ com~$i < j$ e tal que~${y = P[i][j-1] > P[i][j] = x}$ implica que~${A[i][x-1] + A[x][j] \leq A[i][y-1] + A[y][j]}$ e~${A[i][x-1] + A[x][j-1] > A[i][y-1] + A[y][j-1]}$, contrariando o que foi provado neste parágrafo. Concluímos que~${P[i][j-1] \leq P[i][j]}$. 

De forma simétrica, para mostrar que~${P[i][j] \leq P[i+1][j]}$, basta perceber que se~${x,y \in [n]}$ e~${x > y}$, então será verdade que a desigualdade~${A[i][x-1] + A[x][j] < A[i][y-1] + A[y][j]}$ implica que~${A[i+1][x-1] + A[x][j] < A[i+1][y-1] + A[y][j]}$, o que contradiz~${P[i][j] > P[i+1][j]}$, provando que~${P[i][j] \leq P[i+1][j]}$.
\end{proof}

Com o Lema~\ref{Monge:MCtoTM} e o Teorema~\ref{KY:TMtoKY}, vale que~$A$ é Knuth-Yao otimizável e podemos aplicar a otimização de Knuth-Yao para resolver o problema em tempo~$\Cl{O}(n^2)$.
